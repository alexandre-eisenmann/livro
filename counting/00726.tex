\begin{ex}
(Puc-SP) Uma urna contém bolas numeradas de 1 a 5. Sorteia-se uma bola, verifica-se o seu número e ela é reposta na urna. Num segundo sorteio, procede-se da mesma forma que no primeiro. A probabilidade de que o número da segunda bola seja estritamente maior que o da primeira é:
    \begin{enumerate}[(a)]
    \item $\frac{4}{5}$
    \item $\frac{2}{5}$
    \item $\frac{1}{5}$
    \item $\frac{1}{25}$
    \item $\frac{15}{25}$
    \end{enumerate}
      \begin{sol}
      resposta: b \\  
      Existem 25 possibilidades para as bolas serem tiradas. Se a primeira bola for 1, a segunda pode ser 2, 3, 4 ou 5 (4 possibilidades). Se for 2 , a segunda pode ser 3, 4 ou 5 (3 possibilidades); se for 3, a segunda pode ser 4 ou 5; se for 4 a segunda será 5. Total: $4+3+2+1=10 \Longrightarrow p=\frac{10}{25}=\frac{2}{5}$
      \end{sol}
\end{ex}