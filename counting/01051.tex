\begin{ex}
 	(Fuvest) Escolhido ao acaso um elemento do conjunto dos divisores positivos de 60, a probabilidade de que ele seja primo é:
    \begin{enumerate}[(a)]
    \item $\frac{1}{2}$
    \item $\frac{1}{3}$
    \item $\frac{1}{4}$
    \item $\frac{1}{5}$
    \item $\frac{1}{6}$
    \end{enumerate}
      \begin{sol}
        resposta: c \\
        divisores de 60 = \{1, 2, 3, 4, 5, 6, 10, 12, 15, 20, 30, 60\} = 12 divisores e 3 primos ( 2, 3, 5 )\\
        $\Longrightarrow\frac{3}{12}=\frac{1}{4}$
      \end{sol}
\end{ex}