\begin{ex}
Uma caixa contém três bolas vermelhas e cinco bolas brancas e outra caixa possui duas bolas vermelhas e três bolas brancas. Considerando-se que uma bola é transferida da primeira caixa para a segunda, e que uma bola é retirada da segunda caixa, podemos afirmar que a probabilidade de que a bola retirada seja de cor vermelha é:
   \begin{enumerate}[(a)]
   \item $\frac{18}{75}$
   \item $\frac{19}{45}$
   \item $\frac{19}{48}$
   \item $\frac{18}{45}$
   \item $\frac{19}{75}$
   \end{enumerate}
   \begin{sol}
    resposta: c \\
    $\frac{3}{8}=$ probabilidade de 1 bola vermelha ser transferida para 2ª caixa \\
    2ª caixa fica com 3 vermelhas e 3 brancas\\
    $\frac{3}{6}=$ probabilidade de sair uma vermelha da 2ª caixa \\
    $\frac{5}{8}=$ probabilidade de 1 bola branca ser transferida para 2ª caixa\\
    2ª caixa fica com 2 vermelhas e 4 brancas \\
    $\frac{2}{6}=$ probabilidade de sair 1 vermelha na 2ª caixa \\
    $\Longrightarrow \frac{3}{8}\cdot\frac{3}{6}+\frac{5}{8}\cdot\frac{2}{6}=\frac{19}{48}$
   \end{sol}
\end{ex}