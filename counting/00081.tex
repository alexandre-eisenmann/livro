\begin{ex}
  (Puc) Sabe-se que num dado momento, no caixa de um supermercado há 40 moedas, que totalizam a quantia de R\$ 3,75. Sabe-se também que: 
    \begin{itemize}  
        \item as moedas são apenas de três tipos: 5 centavos, 10 centavos e 25 centavos;
        \item o número de moedas de 10 centavos é o triplo da quantidade das de 25 centavos.
    \end{itemize}
   A probabilidade de retirar-se desse caixa, sucessivamente e sem reposição, três moedas em ordem crescente de valores é: 
    \begin{enumerate}  [(a)]
        \item $\frac{21}{988}$
        \item $\frac{23}{988}$
        \item $\frac{25}{988}$
        \item $\frac{27}{988}$
        \item $\frac{29}{988}$
    \end{enumerate}
      \begin{sol}
       resposta: c \\
       \textit{x} = nº moedas de 5 centavos;  \textit{y} = nºmoedas de 25 centavos; \textit{3y} = nº moedas de 10 centavos \\ 
       $
       \left \{
       \begin{array} {cl}
        x+3y+y=40 \\  
        0,05x+0,10\cdot3y+0,25y=3,75 \\
        \end{array}
        \right.
      $ $\Longrightarrow x=20,\hspace{0,2cm} y=5$ \\
      a probabilidade de serem retiradas da caixa, sucessivamente e sem reposição, 3 moedas em ordem crescente, é: $\frac{20}{40}\cdot\frac{15}{39}\cdot\frac{5}{38}=\frac{25}{988}$
      \end{sol}
 \end{ex}