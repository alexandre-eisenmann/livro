\begin{ex}
Numa dinâmica de grupo, uma psicóloga de RH (Recursos Humanos) relaciona de todas as formas possíveis dois participantes: ao primeiro faz a pergunta e ao segundo pede que comente a resposta do colega. Admita que a psicóloga não fará a mesma pergunta mais de uma vez.
   \begin{enumerate}[(a)]
   \item se 10 candidatos participam da dinâmica, qual é o número de perguntas feitas pela psicóloga?
   \item qual é o número  mínimo de candidatos que obriga a psicóloga a ter mais de 250 questões para realizar a dinâmica?
   \end{enumerate}
      \begin{sol}
      \phantom{A}
        \begin{enumerate} [(a)]
            \item $\mathrm{A}_{{10},2}=90$
            \item $\mathrm{A}_{n,2}>250 \Longrightarrow n>16\hspace{0,3cm} \therefore \hspace{0,3cm}n=17$
        \end{enumerate}
      \end{sol}
\end{ex}