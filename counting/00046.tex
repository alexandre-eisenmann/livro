\begin{ex}
  (Albert Einstein - medicina) Uma escola possui duas turmas que estão no terceiro ano, A e B. O terceiro ano A tem 24 alunos, sendo 10 meninas, e o terceiro ano B tem 30 alunos, sendo 16 meninas. Uma dessas turmas será escolhida aleatoriamente e, em seguida, um aluno da turma sorteada será aleatoriamente escolhido. A probabilidade de o aluno escolhido ser uma menina é:   
     \begin{enumerate}  [(a)]
         \item $\frac{13}{27}$
         \item $\frac{15}{32}$
         \item $\frac{19}{40}$
         \item $\frac{21}{53}$
     \end{enumerate}
       \begin{sol}
       resposta: c \\
       \begin{tabular}{|c|c|c|c|} \hline
       turma & meninas & meninos & total  \\  \hline
       A & 10 & 14 & 24  \\  \hline
       B & 16 & 14 & 30  \\  \hline
       \end{tabular}
       $\Longrightarrow p=\frac{1}{2}\cdot\frac{10}{24}+\frac{1}{2}\cdot\frac{16}{30}=\frac{19}{40}$
       \end{sol}
 \end{ex}