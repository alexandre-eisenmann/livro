\begin{ex}
 (FGV) Dez livros diferentes, incluindo dois de Português e 3 de Matemática, deverão ser colocados em uma estante, em qualquer ordem. Entretanto, os 2 livros de Português deverão estar juntos, o mesmo acontecendo com os 3 livros de Matemática. O número de diferentes maneiras de se fazer essa arrumação é:
    \begin{enumerate}[(a)]
    \item 3.628.800
    \item 60.480
    \item 5040
    \item 2520
    \item 1440
    \end{enumerate}
      \begin{sol}
        resposta: b \\
        $7!\cdot2!\cdot3!=60480$
      \end{sol}
\end{ex}