\begin{ex}
Numa prova de três questões verificou-se que 5 alunos acertaram as três questões; 15 alunos acertaram $Q_1$ e $Q_3$ ; 17 alunos acertaram $Q_1$ e $Q_2$; 12 alunos acertaram $Q_2$ e $Q_3$; 55 alunos acertaram $Q_1$; 55 acertaram $Q_2$; 64 acertaram $Q_3$ e 13 alunos erraram as três questões.
   \begin{enumerate}[(a)]
   \item quantos alunos há no total?
   \item quantos alunos acertaram apenas a $Q_2$ ?
   \item sorteando-se um aluno, qual a probabilidade dele ter acertado exatamente uma questão ?
   \item sorteando-se um aluno, qual a probabilidade dele ter acertado pelo menos duas questões ?
   \end{enumerate}
     \begin{sol}
       \phantom{A} \\ \\
        \begin{venndiagram3sets} [labelA=\(Q_1\),labelB=\(Q_2\),labelC=\(Q_3\),labelOnlyA=28,labelOnlyB=31,labelOnlyC=42, labelNotABC=13, labelABC=5, labelOnlyAB=12,labelOnlyBC=7, labelOnlyAC=10]
        \end{venndiagram3sets}
         \begin{enumerate} [(a)]
            \item $28+31+42+10+5+7+12+13=148$
            \item 31
            \item $28+31+42=101\rightarrow \frac{101}{148}\approx 68\%$
            \item $12+7+10+5=\frac{34}{148}\approx 23\%$
        \end{enumerate}
     \end{sol}
\end{ex}