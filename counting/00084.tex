\begin{ex}
  (Puc) Em um pote de vidro não transparente, foram colocados mini sabonetes, todos de mesmo tamanho, sendo 16 deles na cor amarela, 6 na cor verde e 4 na cor azul. Retirando-se aleatoriamente 3 desses mini sabonetes, um após o outro, sem reposição, a probabilidade de saírem pelo menos 2 deles na cor amarela, sabendo que o primeiro mini sabonete retirado era na cor amarela, é:
     \begin{enumerate} [(a)]
         \item $\frac{11}{20}$
         \item $\frac{13}{20}$
         \item $\frac{15}{20}$
         \item $\frac{17}{20}$
     \end{enumerate}
       \begin{sol}
        resposta: d \\
        2 amarelos e 1 verde ou 2 amarelos e 1 azul ou 1 amarelo e 1 amarelo\\
        $2\cdot\frac{15}{25}\cdot\frac{6}{24}+2\cdot\frac{15}{25}\cdot\frac{4}{24}+\frac{15}{25}\cdot\frac{14}{24}=\frac{510}{600}=\frac{17}{20}$
       \end{sol}
 \end{ex}