\begin{ex}
(Ita) São dados dois cartões, sendo que um deles tem ambos os lados na cor vermelha e o outro tem um lado na cor vermelha e o outro na cor azul. Um dos cartões é escolhido ao acaso e colocado sobre uma mesa. Se a cor exposta é vermelha, calcule a probabilidade de o cartão escolhido ter a outra cor também vermelha.
  \begin{sol}
   \phantom{A}\\
   Se uma face vermelha for exposta, sobram 2 vermelhas e 1 azul. \\
   A probabilidade de termos outra vermelha é $\frac{2}{3}$ 
  \end{sol}
\end{ex}