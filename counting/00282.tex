\begin{ex}
(Unaerp –SP) Em certa região metropolitana, 52\% da população  tem mais de 25 anos. Sabe-se ainda que 30\% das pessoas com mais de 25 anos tem menos de 35 anos.  Escolhendo-se ao acaso uma pessoa dessa região, qual a probabilidade de que seja alguém com 35 anos ou mais?
  \begin{sol}
    \phantom{A}\\
    30\% de 52\% = 15,6\% \\
  $52\%+15,6\%=67,6\% \Longrightarrow 100\%-67,6\%=32,4\%$
  \end{sol}
\end{ex}