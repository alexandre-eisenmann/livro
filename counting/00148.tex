\begin{ex}
Considere os números de três algarismos obtidos das permutações dos algarismos 5, 6 e 7. Ao ser sorteada uma dessas permutações, calcule a probabilidade de que o número obtido seja:
   \begin{enumerate}[(a)]
   \item par
   \item ímpar
   \item maior que 700
   \item menor quue 650
   \end{enumerate}
      \begin{sol}
        \phantom{A} \\
        espaço amostral = 6
         \begin{enumerate} [(a)]
             \item $\frac{\phantom{A}}{2}\frac{\phantom{A}}{1}\frac{6}{\phantom{A}}\rightarrow p=\frac{2}{6}=\frac{1}{3}$
             \item $\frac{\phantom{A}}{2}\frac{\phantom{A}}{1}\frac{\phantom{A}}{2} \rightarrow p=\frac{4}{6}=\frac{2}{3}$
             \item $\frac{7}{\phantom{A}}\frac{\phantom{A}}{2}\frac{\phantom{A}}{1} \rightarrow p=\frac{2}{6}=\frac{1}{3}$  \item $\frac{5}{\phantom{A}}\frac{\phantom{A}}{2}\frac{\phantom{A}}{1} \rightarrow p=\frac{2}{6}=\frac{1}{3}$
         \end{enumerate}
      \end{sol}
\end{ex}