\begin{ex}
Os 500 estudantes de uma escola responderam a uma pergunta sobre qual a sua área de conhecimento preferida. Dos 200 que preferiam exatas 120 eram homens; 100 homens preferiam biologia e dos 125 estudantes que preferiam humanas, 64\% eram mulheres.
   \begin{enumerate}[(a)]
   \item faça uma tabela com estes dados e complete-a com os que estão faltando.
   \item um estudante é escolhido ao acaso. Calcule a probabilidade do estudante escolhido ser homem ou ser pessoa que prefere humanas.
   \end{enumerate}
     \begin{sol}
       \phantom{A} 
        \begin{enumerate} [(a)]
            \item 
       \begin{tabular}{|c|c|c|c|c|}  \hline
           &exatas&biológicas&humanas&total  \\ \hline
   homem&120&100&45&265  \\  \hline
   mulher&80&75&80&235  \\  \hline
   total &200&175&125&500  \\  \hline 
       \end{tabular}
          \item $\frac{265}{500}+\frac{125}{500}-\frac{45}{500}=\frac{69}{100}$
       \end{enumerate}
     \end{sol}
\end{ex}