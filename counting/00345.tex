\begin{ex}
(FGV) Uma prova discursiva de Matemática deve conter 5 questões de Álgebra, 3 questões de Geometria e 2 de Trigonometria, num total de 10 questões. Para elaborar a prova, a banca dispõe de 8 questões de Álgebra, 6 de Geometria e 4 de Trigonometria. 
   \begin{enumerate}[(a)]
   \item com as informações dadas, quantas provas distintas, isto é, que tenham ao menos uma questão diferente, podem ser elaboradas?
   \item do total das 18 questões disponíveis, 14 são difíceis e 4 de Álgebra são médias. Qual a probabilidade de elaborar uma prova difícil, sabendo que ela deve conter pelo menos 7 questões difíceis?
   \end{enumerate}
     \begin{sol}
      \phantom{A}
        \begin{enumerate} [(a)]
            \item $\mathrm{C}_{6,3}\cdot\mathrm{C}_{4,2}\cdot\mathrm{C}_{8,5}=6720$
            \item $\mathrm{C}_{8,5}\rightarrow$ 5 questões de álgebra (médias M e difíceis D):
            2D e 3M \textbf{ou} 3D e 2M \textbf{ou} 4D e 1 M\\
            $(\mathrm{C}_{4,2}\cdot\mathrm{C}_{4,3}+\mathrm{C}_{4,3}\cdot\mathrm{C}_{4,2}+\mathrm{C}_{4,4}\cdot\mathrm{C}_{4,1})\cdot\mathrm{C}_{6,3}\cdot\mathrm{C}_{4,2}=52\cdot20\cdot6=6240 \Longrightarrow p=\frac{6240}{6720}=\frac{13}{14}$
        \end{enumerate}
     \end{sol}
\end{ex}