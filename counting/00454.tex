\begin{ex}
(Unesp) A eficácia de um teste de laboratório para checar certa doença nas pessoas que comprovadamente têm essa doença é de 90\%. Esse mesmo teste, porém, produz um falso positivo (acusa positivo em quem não tem comprovadamente a doença) da ordem de 1\%. Em um grupo populacional em que a incidência dessa doença é 0,5\%, seleciona-se uma pessoa ao acaso para fazer o teste. Qual é a probabilidade de que o resultado desse teste venha a ser positivo?
  \begin{sol}
   \phantom{A} \\
   Em uma população \textit{x}, tem-se: \\
   - $0,5\%\cdot x$ comprovadamente tem a doença, e em 90\% desse porcentual o resultado do teste é positivo.\\
   - $99,5\% \cdot x$  comprovadamente não tem a doença, e em 1\% desse porcentual o resultado do teste é positivo.\\ Assim, a probabilidade de que o resultado desse teste ser positivo é: \\
   $p=\frac{90\%\cdot0,5\%\cdot x +1\% \cdot99,5\% \cdot x}{x} \Longrightarrow p= 0,01445=\frac{1,445}{100}=1,445\%$
   
   
  \end{sol}
\end{ex}