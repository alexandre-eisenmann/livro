\begin{ex}
 (Unesp) Uma urna contém as letras: A, C, D, D, E, E, F, I, I e L.
  \begin{enumerate} [(a)]
      \item Se todas as letras forem retiradas da urna, uma após a outra, sem reposição, calcule a probabilidade de, na sequencia das retiradas, ser formada a palavra FELICIDADE.
      \item Se somente duas letras forem retiradas da urna, uma após a outra, sem reposição, calcule a probabilidade de serem retiradas duas letras iguais.
  \end{enumerate}
    \begin{sol}
    \phantom{A}
      \begin{enumerate} [(a)]
          \item as letras de FELICIDADE devem ser retiradas na ordem \\
          $p=\frac{1}{10}\cdot\frac{2}{9}\cdot\frac{1}{8}\cdot\frac{2}{7}\cdot\frac{1}{6}\cdot\frac{1}{5}\cdot\frac{2}{4}\cdot\frac{1}{3}\cdot\frac{1}{2}\cdot\frac{1}{1}= \frac{8}{10!}$
          \item 2 E ou 2 I ou 2 D \\
          $p=\frac{2}{10}\cdot\frac{1}{9}\cdot\frac{2}{10}\cdot\frac{1}{9}\cdot\frac{2}{10}\cdot\frac{1}{9}=\frac{6}{90}=\frac{1}{15}$
      \end{enumerate}
    \end{sol}
\end{ex}