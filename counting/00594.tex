\begin{ex}
 (Ibmec) A fase final de um processo de seleção de gerentes e supervisores para uma empresa é constituída de uma entrevista individual, com duração de uma hora para os candidatos a gerente e 30 minutos com duração para os candidatos a supervisor. 	
Nessa etapa, restam 10 candidatos, sendo 5 para cada um dos cargos. Todas as entrevistas serão realizadas no mesmo dia, sendo chamado um candidato por vez, e não havendo intervalo entre duas entrevistas consecutivas.
A ordem de chamada dos candidatos será definida por sorteio, e a primeira entrevista ocorrerá às 10 h. 
Márcia, uma das candidatas ao cargo de gerente, está preocupada, pois tem um compromisso nesse dia, precisando sair antes do término da última entrevista.
   \begin{enumerate}[(a)]
   \item calcule a probabilidade de que a entrevista de Márcia termine até as 11 h 30 min.
   \item calcule a probabilidade de que a entrevista de Márcia termine até as 12h. 
   \end{enumerate}
    \begin{sol}
     \phantom{A}
     \begin{enumerate} [(a)]
         \item para terminar às 11h30, Márcia deve ser a primeira \textbf{ou} a segunda a ser entrevistada \\
         $\frac{1}{10}+\frac{5}{10}\cdot\frac{1}{9}=\frac{7}{45}$
         \item para terminar às 12h, Márcia pode ser a 1ª ou 2ª entrevistada. Se Márcia for a 3ª entrevistada, as duas primeiras entrevistas devem ser para supervisor.\\
         $\frac{1}{10}+\frac{9}{10}\cdot\frac{1}{9}+\frac{5}{10}\cdot\frac{4}{9}\cdot\frac{1}{8}=\frac{41}{180}$
     \end{enumerate}
    \end{sol}
\end{ex}