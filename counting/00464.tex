\begin{ex}
(Fei) Numa urna foram colocadas 30  bolas: 10 bolas azuis numeradas de 1 a 10; 15 bolas brancas numeradas de 1 a 15 e 5 bolas cinzas numeradas de 1 a 5. Ao retirar-se aleatoriamente uma bola, a probabilidade de se obter uma bola par ou branca é:
   \begin{enumerate}[(a)]
   \item $\frac{29}{30}$
   \item $\frac{7}{15}$
   \item $\frac{1}{2}$
   \item $\frac{11}{15}$
   \item $\frac{13}{15}$
   \end{enumerate}
     \begin{sol}
      resposta: d \\
      temos:  15 bolas brancas, 14 bolas com número par, 7 bolas brancas e de número par. \\
      $\Longrightarrow \frac{14}{30}+\frac{15}{30}-\frac{7}{30}=\frac{11}{15}$
     \end{sol}
\end{ex}