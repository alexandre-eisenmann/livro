\begin{ex}
 (Fatec) Uma pessoa escreveu todos os anagramas da palavra FATEC, cada um em um pedacinho de papel, e colocou-os cada um em um recipiente vazio. Retirando-se um desses papéis do recipiente, ao acaso, a probabilidade de que o anagrama nele inscrito tenha as duas vogais juntas é:
    \begin{enumerate}[(a)]
    \item $\frac{7}{10}$
    \item $\frac{3}{10}$
    \item $\frac{2}{5}$
    \item $\frac{3}{5}$
    \item $\frac{1}{2}$
    \end{enumerate}
      \begin{sol}
      resposta: c  \\
         - anagramas de FATEC = 5! \\
         - vogais juntas =  $4!\cdot2$ (AE ou EA)\\
         $p=\frac{48}{120}=\frac{2}{5}$
      \end{sol}
\end{ex}