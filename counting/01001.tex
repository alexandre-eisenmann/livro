\begin{ex}
 	O horário de uma classe, num certo dia de semana , deve conter 10 aulas, sendo 3 de Matemática (uma de álgebra, uma de geometria e uma de trigonometria), duas de Física (uma de mecânica e uma de termologia), 3 de Português (uma de gramática, uma de literatura, uma de redação), duas de História ( geral e do Brasil). Determine o número de maneiras distintas de se formar um horário com:
    \begin{enumerate}[(a)]
    \item as 3 aulas de Matemática consecutivas e na ordem descrita acima;
    \item as 3 aulas de Matemática consecutivas, em qualquer ordem: 
    \item as 3 aulas de Matemática não são consecutivas.
    \end{enumerate}
      \begin{sol}
        \phantom{A} 
          \begin{enumerate} [(a)]
              \item $8!=40.320$
              \item $8!\cdot3!=241.920$
              \item $10! - 8!\cdot3!=3.386.880$
          \end{enumerate}
      \end{sol}
\end{ex}