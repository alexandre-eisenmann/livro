\begin{ex}
Qualquer símbolo utilizado na escrita de uma linguagem é chamado de caractere, por exemplo: letras, algarismos, sinais de pontuação, sinais de acentuação, sinais especiais, etc. Em computação, cada caractere é representado por uma sequência de 8 bits, e cada bit pode assumir dois estados, representados por 0 ou 1. Por exemplo, a sequência 01000111 representa a letra G. Assim, o número máximo de caracteres que podem ser representados por todas as sequências de 8 bits é:
   \begin{enumerate}[(a)]
   \item 16
   \item 32
   \item 64
   \item 128
   \item 256
   \end{enumerate}
     \begin{sol}
       resposta: e \\
       $2^8=256$
     \end{sol}
\end{ex}