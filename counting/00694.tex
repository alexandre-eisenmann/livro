\begin{ex}
 Numa eleição para prefeito de uma cidade concorreram somente os candidatos A e B. Em uma seção eleitoral votaram 250 eleitores. Do número total de votos dessa seção, 42\% foram para o candidato A, 34\% para o candidato B, 18\% foram anulados e os restantes estavam em branco. Tirando-se ao acaso, um voto dessa urna, a probabilidade de que seja um voto em branco é:
    \begin{enumerate}[(a)]
    \item $\frac{1}{100}$
    \item $\frac{3}{50}$
    \item $\frac{1}{50}$
    \item $\frac{1}{25}$
    \item $\frac{3}{20}$
    \end{enumerate}
      \begin{sol}
      resposta: b \\
       $ 42+34+18=94 \longrightarrow 100-94=6 \Longrightarrow 6\%=\frac{3}{50}$
       \end{sol}
\end{ex}