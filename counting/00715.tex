\begin{ex}
 (Ita) Quantos números de 6 algarismos distintos podemos formar usando os dígitos 1, 2, 3, 4, 5 e 6, nos quais o 1 e o 2 nunca ocupam posições adjacentes, mas o 3 e o 4 sempre ocupam posições adjacentes?
    \begin{enumerate}[(a)]
    \item 144
    \item 180
    \item 240
    \item 288
    \item 360
    \end{enumerate}
      \begin{sol}
       resposta: a
       \begin{enumerate} [--]
           \item o 3 e 4 estão juntos: $1\hspace{0.1cm} 2\hspace{0.15cm}  (3\hspace{0.15cm}  4)\hspace{0.15cm} 5\hspace{0.15cm}  6\rightarrow 5!\cdot2!=240$
           \item supondo o 1 e 2 juntos: $(1\hspace{0.15cm}2)\hspace{0.15cm}(3\hspace{0.15cm}4)\hspace{0.15cm} 5\hspace{0.15cm}6=4!\cdot2!\cdot2!=96$\\
           total (-) 1 e 2  juntos:
           240-96 = 144
       \end{enumerate}    
      \end{sol}
\end{ex}