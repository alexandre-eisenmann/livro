\begin{ex}
Um levantamento feito com 200 funcionários de uma empresa ( metade deles, homens, metade mulheres) mostrou que dos  80 fumantes, 70 eram homens.
   \begin{enumerate}[(a)]
   \item Complete a tabela:
   \begin{center}
   \begin{tabular}{|c|c|c|c|} \hline
       &  Homens  &  Mulheres  &  TOTAL  \\    \hline
   Fumantes    &          &            &         \\    \hline
   Não fumantes  &          &            &         \\    \hline
   TOTAL      &          &            &         \\    \hline
   \end{tabular}

   \end{center}    
   \item Sorteia-se um não fumante. Qual é probabilidade de que seja mulher?
   \item Sorteia-se uma mulher. Qual é a probabilidade de que seja não fumante?
   \item Qual é a probabilidade de se sortear ao acaso uma mulher não fumante?
   \item Ao se sortear alguém, qual é a probabilidade de ser homem ou fumante?
   \item Sorteando-se duas pessoas fumantes qual a probabilidade de haver nesse grupo, pelo        menos uma mulher? 
   \end{enumerate}
     \begin{sol}
      \phantom{A}
       \begin{enumerate}  [(a)]
           \item  \begin{tabular}{|c|c|c|c|} \hline
       &  Homens  &  Mulheres  &  TOTAL  \\    \hline
   Fumantes    & 70 & 10 & 80 \\    \hline
   Não fumantes  & 30  & 90 & 120 \\    \hline
   TOTAL      & 100  & 100  &  200   \\    \hline
   \end{tabular}
          \item $\frac{90}{120}=\frac{3}{4}$
          \item $\frac{9}{10}$
          \item $\frac{90}{200}=\frac{9}{20}$
          \item $\frac{10}{20}+\frac{8}{20}-\frac{7}{20}=\frac{11}{20}$
          \item $1-\frac{70}{80}\cdot\frac{69}{79}=\frac{149}{632}$
       \end{enumerate}
       
     \end{sol}
\end{ex}