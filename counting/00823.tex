\begin{ex}	
(Fuvest) Uma classe de Educação Física de um colégio é formada por 10 estudantes, todos com alturas diferentes. As alturas dos estudantes, em ordem crescente, serão designadas por $h_1$,$h_2$,……$h_{10}$  ($h_1<h_2<...<h_{10}).$
O professor vai escolher 5 desses estudantes para participar de uma demonstração na qual eles se apresentarão alinhados, em ordem crescente de suas alturas. Dos
$\binom {10} {5}= 252$ grupos que podem ser escolhidos, em quantos, o estudante cuja altura é $h_7$ ocupará a posição central durante a demonstração? 
    \begin{enumerate}[(a)]
    \item 7
    \item 10
    \item 21
    \item 45
    \item 60
    \end{enumerate}
      \begin{sol}
      resposta: d \\
     $h_1,h_2,h_3,h_4,h_5,h_6 \longrightarrow \mathrm{C }_{6,2}$\hspace{1cm} e\hspace{1cm} $h_8,h_9,h_{10} \longrightarrow \mathrm{C}_{3,2}$\\
     $\Longrightarrow \mathrm{C}_{6,2}\cdot\mathrm{C}_{3,2}=45$
     
      \end{sol}
\end{ex}