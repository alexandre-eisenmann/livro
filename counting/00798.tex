\begin{ex}
 (Puc - MG) O dispositivo que aciona a abertura do cofre de uma joalheria apresenta um teclado com 8 teclas, 4 delas identificadas pelos algarismos \{1, 2, 3,  4\} e quatro outras teclas pelas letras \{a, b, c, d\}. O segredo do cofre é uma sequência de 3 algarismos distintos seguido por uma sequência de duas letras distintas. A probabilidade de uma pessoa abrir esse cofre, numa única tentativa, feita ao acaso, é:
    \begin{enumerate}[(a)]
    \item $\frac{1}{256}$
    \item $\frac{1}{128}$
    \item $\frac{1}{444}$
    \item $\frac{1}{192}$
    \item$\frac{1}{288}$
    \end{enumerate}
      \begin{sol}
        resposta: e \\
       $ \underbrace{
        \frac{\phantom{A}}{4}\frac{\phantom{A}}{3}\frac{\phantom{A}}{2}}_\text{algar.}\underbrace{
        \frac{\phantom{A}}{4}\frac{\phantom{A}}{3}}_\text{letras}= 4\cdot3\cdot2\cdot4\cdot3=288\Longrightarrow p=\frac{1}{288}$
      \end{sol}
\end{ex}