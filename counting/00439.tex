\begin{ex}
Uma consumidora pegou, ao acaso, um tubo de creme dental da gôndola de um supermercado. A probabilidade de esse tubo conter 90g de creme é $\frac{2}{5}$, e a probabilidade de conter creme com própolis é $\frac{1}{3}$. Sabendo que a probabilidade de esse tubo conter 90g de creme ou conter creme com própolis é $\frac{3}{5}$ , calcule a probabilidade de ele conter 90g de creme com própolis.
  \begin{sol}
     \phantom{A} \\
     90g creme ou própolis: \hspace{0,3cm}$P(A \cup B)$ \\
     90g creme com própolis:  \hspace{0,3cm}$P(A \cap B)$ \\
     $\frac{3}{5}=\frac{2}{5}+\frac{1}{3}-P(A \cap B) \Longrightarrow P(A \cap B)=\frac{2}{15}$
  \end{sol}
\end{ex}