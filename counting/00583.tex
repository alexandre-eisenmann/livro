\begin{ex}
(Uerj) Com o intuito de separar o lixo para fins de reciclagem, uma instituição colocou em suas dependências, cinco lixeiras de diferentes cores, de acordo com o tipo de resíduo a que se destinam: vidro, metal, papel, plástico e orgânico.	Sem olhar para as lixeiras, João joga em uma delas uma embalagem plástica e, ao mesmo tempo, em outra, uma garrafa de vidro. A probabilidade de que ele tenha usado corretamente pelo menos uma lixeira é igual a:
   \begin{enumerate}[(a)]
   \item 25\%
   \item 30\%
   \item 35\%
   \item 40\%
   \end{enumerate}
     \begin{sol}
      resposta: c \\
      acertar o plástico e errar o vidro \textbf{ou} acertar o vidro e errar o plástico \textbf{ou} acertar o plástico e acertar o vidro:\\
      $\frac{1}{5}\cdot\frac{3}{4}+\frac{1}{5}\cdot\frac{3}{4}+\frac{1}{5}\cdot\frac{3}{4}=\frac{7}{20}=35\%$
     \end{sol}
\end{ex}