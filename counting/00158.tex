\begin{ex}
(Puc) Das 156 pessoas que participaram de um seminário sobre O Desenvolvimento de Projetos de Pesquisa no Brasil, sabe-se que: 
– 90 eram do sexo masculino;
– 75\% eram alunos da PUC; 
– 24 eram do sexo feminino e não eram alunos da PUC. 
Nessas condições, é correto afirmar que, entre os participantes:   
 \begin{enumerate} [(a)]
     \item 80 homens eram alunos da PUC.
     \item 45 mulheres eram alunas da PUC. 
     \item o número dos que não estudavam na PUC era igual a 42.
     \item o número de homens excedia o de mulheres em 34 unidades.
     \item  a razão entre o número de mulheres que não estudavam na PUC e o daquelas que lá estudavam, nesta ordem, é  4/7
 \end{enumerate}
   \begin{sol}
     resposta: e \\
     montar uma tabela: \\
     $75\%\cdot156=117$\hspace{0.6cm}$25\%\cdot156=39$\\
     
       \begin{tabular}{|c|c|c|c|} \hline
            & alunos PUC & não alunos PUC & Total\\ \hline
          masculino  & 75 & 15 & 90 \\  \hline
          feminino & 42 & 24 & 66  \\  \hline
          Total & 117 & 39 & 156 \\  \hline
       \end{tabular}
       $\Longrightarrow  \frac{24}{42}=\frac{4}{7}$
   \end{sol}
\end{ex}