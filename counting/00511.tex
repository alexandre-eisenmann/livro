\begin{ex}
(Cesgranrio) Numa caixa são colocados vários cartões, alguns amarelos, alguns verdes e os restantes pretos. Sabe-se que 50\% dos cartões são pretos, e que, para cada três cartões verdes, há 5 cartões pretos. Retirando-se ao acaso um desses cartões, a probabilidade de que seja amarelo é de:
   \begin{enumerate}[(a)]
   \item 10\%
   \item 15\%
   \item 20\% 
   \item 25\%
   \item 40\%
   \end{enumerate}
     \begin{sol}
      resposta: c \\
      3 verdes para 5 pretos =$\frac{3}{5}=0,6=60\%$ \hspace{0,4cm}
      verdes são 60\% dos pretos $\rightarrow  60\%\cdot50\%=30\%$\\
      $50\%+30\%=80\% \longrightarrow 20\% $ são amarelos.
     \end{sol}
\end{ex}