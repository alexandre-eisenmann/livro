\begin{ex}
 Num fim de semestre em uma determinada escola, constatou-se que 55\% dos alunos foram reprovados em Matemática, 65\% em Química, 31\% em Física, 16\% só em Matemática, 19\% em Matemática e Física, 20\% em Química e Física e 15\% em Matemática, Física e Química. Escolhendo um aluno ao acaso, qual a probabilidade:
    \begin{enumerate}[(a)]
    \item de não estar reprovado em nenhuma das três?
    \item de estar reprovado em Física ou em Matemática?
    \item de estar reprovado em Química, dado que não está reprovado em Matemática?
    \end{enumerate}
      \begin{sol}
        Diagrama \\
        \begin{venndiagram3sets}[labelA=\(M\),labelB=\(Q\),labelC=\(F\),labelOnlyA=16,labelOnlyB=25,labelOnlyC=7,labelNotABC=8,labelABC=15,labelOnlyAB=20,labelOnlyBC=5,labelOnlyAC=4,radius=1.2cm,tikzoptions={scale=1.5}]
        \end{venndiagram3sets}
        \begin{enumerate} [(a)]
            \item $65+11+16=92 \rightarrow 100-92=8\%$
            
            \item 100-(25+8)=67\% \hspace{0.2cm} ou \hspace{0.2cm} 55+31-19=67\%
            
            \item $p=\frac{30}{35}=\frac{2}{3}$
        \end{enumerate}
      \end{sol}
\end{ex}