\begin{ex}
 (UFPR) Um grupo de pessoas foi classificado quanto ao peso e pressão arterial, conforme mostrado no quadro a seguir:

\begin{center}
\begin{tabular}{|c|c|c|c|c|} \hline
\multirow {2} {*} {PRESSÃO} & \multicolumn {4}{c|}{PESO}\\
\cline {2-5}
& EXCESSO & NORMAL & DEFICIENTE & TOTAL\\  \hline 
ALTA & 0,10 & 0,08 & 0,02 & 0,20 \\ \hline
NORMAL & 0,15 & 0,45 & 0,20 & 0,80 \\ \hline
TOTAL & 0,25 & 0,53 & 0,22 & 1,00 \\ \hline
\end{tabular}
\label{tab:my_label}
\end{center}
Com bases nesses dados, considere as seguintes afirmativas:\\
1. A probabilidade de uma pessoa  escolhida ao acaso nesse grupo ter pressão alta é de 0,20.\\
2. Se se verifica que uma pessoa escolhida ao acaso, nesse grupo, tem excesso de peso, a probabilidade de ela ter também pressão alta é de 0,40.\\
3. Se se verifica que uma pessoa escolhida ao acaso, nesse grupo, tem pressão alta, a probabilidade  de ela ter também peso normal é de 0,08.\\
4. A probabilidade de uma pessoa escolhida ao acaso nesse grupo ter pressão normal e peso deficiente é de 0,20.\\
Assinale a alternativa correta:\\
    \begin{enumerate}[(a)]
    \item somente as afirmações 1, 2 e 3 são verdadeiras.
    \item somente as afirmações 1, 2 e 4 são verdadeiras.
    \item somente as afirmações 1 e 3 são verdadeiras.
    \item  somente as afirmações 2, 3 e 4 são verdadeiras.
    \item  somente as afirmações 2 e 3 são verdadeiras.
    \end{enumerate}
       \begin{sol}
        resposta: b \\
        1. $\frac{0,20}{1,00}=0,20\hspace{0.2cm} (\mathrm{V})$ \\
        2. $\frac{0,10}{0,25}=0,40\hspace{0.2cm} (\mathrm{F})$ ter pressão alta, tendo excesso de peso \\
        3. $\frac{0,08}{0,20}=0,40\hspace{0.2cm}(\mathrm{V})$ ter pressão normal, tendo pressão alta  \\
        4. $0,20 \hspace{0.2cm}(\mathrm{V})$ pressão normal e peso deficiente
       \end{sol}
\end{ex}