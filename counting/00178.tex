\begin{ex}
  Uma urna tem 6 bolas: 2 azuis, 2 brancas e 2 vermelhas. Uma segunda urna tem 9 bolas: 3 azuis, 3 brancas e 3 vermelhas. Uma pessoa vai escolher uma dessas urnas e dela vai retirar 3 bolas. Se ela deseja retirar 3 bolas de cores diferentes umas das outras, é preferível (de acordo com as probabilidades) que ela a retire da primeira urna, da segunda ou é indiferente escolher uma ou outra?
    \begin{sol}
     \phantom{A} \\
     1ª urna $\rightarrow 1\cdot\frac{4}{5}\cdot\frac{2}{4}=\frac{2}{5}$ \\
     2ª urna $\rightarrow 1\cdot\frac{6}{8}\cdot\frac{3}{7}=\frac{9}{28}$ \\
     Preferível a 1ª urna.
     
    \end{sol}
\end{ex}