\begin{ex}
    Uma dado é lançado três vezes. A pessoa A aposta que o número 6 sairá pelo menos uma vez. A pessoa B faz a aposta contrária,  afirmando que o número 6 não ocorrerá em qualquer dos lançamentos. Qual dessas pessoas tem maior probabilidade de ganhar a aposta?
      \begin{sol}
       \phantom{A} \\
       p(A):  1 menos sair 6 todas as vezes $\rightarrow 1-(\frac{5}{6})^3=\frac{91}{216}\simeq42,13\%$ \\
       p(B):   $\frac{5}{6}\cdot\frac{5}{6}\cdot\frac{5}{6}= \frac{125}{216}\simeq57,87\%$ \\
       B tem maior probabilidade de ganhar.
      \end{sol}
\end{ex}