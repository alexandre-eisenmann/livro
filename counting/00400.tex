\begin{ex}
Ao concluir as lições do dia um estudante deve guardar na estante 8 livros: Matemática (M), Física (F), Química (Q), História (H) , Biologia (B) , Português (P), Inglês (I) e Geografia (G), um ao lado do outro.
   \begin{enumerate}[(a)]
   \item em quantas sequências esses livros podem ser dispostos na prateleira de modo que os livros de exatas ( M, F e Q) fiquem juntos?
   \item em quantas sequências esses livros podem ser dispostos na prateleira de modo que F, Q e B fiquem juntos; P e I fiquem juntos e H e G fiquem juntos?
   \end{enumerate}
     \begin{sol}
       \phantom{A} 
        \begin{enumerate} [(a)]
            \item $6!\cdot3!=4320$
            \item $4!\cdot3!\cdot2!\cdot2!=576$
        \end{enumerate}
     \end{sol}
\end{ex}