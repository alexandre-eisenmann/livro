\begin{ex}
  (Puc) Um aluno prestou vestibular em apenas duas Universidades. Suponha que, em uma delas, a probabilidade de que ele seja aprovado é de 30\%, enquanto na outra, pelo fato de a prova ter sido mais fácil, a probabilidade de sua aprovação sobe para 40\%. Nessas condições, a probabilidade de que esse aluno seja aprovado em pelo menos uma dessas Universidades é de:
     \begin{enumerate}  [(a)]
         \item 70\%
         \item 68\%
         \item 60\%
         \item 58\%
         \item 52\%
     \end{enumerate}
       \begin{sol}
        resposta: d \\
        ser reprovado nas duas: $0,7\cdot0,6=0,42$ \\
        probabilidade de ser aprovado em pelo menos uma: $1-0,42=0,58=58\%$
       \end{sol}
 \end{ex}