\begin{ex}
 Numa gaveta de um armário, há 2 chaves do tipo A e uma chave do  tipo B. Noutra gaveta há um cadeado que é aberto pelas chaves do tipo A e 3 cadeados que são abertos pelas chaves do tipo B. Uma pessoa escolhe, ao acaso uma chave da primeira gaveta e um cadeado da segunda gaveta. Qual a probabilidade de o cadeado ser aberto pela chave escolhida?
    \begin{sol}
     Diagrama \\
        \begin{tikzpicture} [grow=right,sloped]
      \tikzstyle{level 1} = [sibling distance = 4.0 cm, level distance = 4.0 cm]
        \tikzstyle{level 2}=[sibling distance = 2.0 cm, level distance= 4.0 cm]
          \node{}
            child{
              node{B}
                 child{
                    node{chave B}
                  edge from parent
                  node[fill=white]{\(\frac{3}{4}\)}
                 }
                 child{
                    node{chave A}
                  edge from parent    
                   node[fill=white]{\(\frac{1}{4}\)}
                 }
            edge from parent
            node[fill=white]{\(\frac{1}{3}\)}
            }
          child{
             node{A}
                child{
                  node{chave B}
                  edge from parent
                   node[fill=white]{\(\frac{3}{4}\)}
                }
                child{
                  node{chave A }
                  edge from parent
                   node[fill=white]{\(\frac{1}{4}\)}
                }
            edge from parent
            node[fill=white]{\(\frac{2}{3}\)}
            };
        \end{tikzpicture}
        \\
        $\frac{2}{3}\cdot\frac{1}{4}$ ou $\frac{1}{3}\cdot\frac{3}{4}$ = $\frac{2}{12}+\frac{3}{12}=\frac{5}{12}$
    \end{sol}
\end{ex}