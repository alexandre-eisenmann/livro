\begin{ex}
Suponha que uma caixa possui 2 bolas pretas e 4 bolas verdes e, outra caixa possui uma bola preta e 3 bolas verdes. Passa-se uma bola da primeira para a segunda caixa, e retira-se uma bola da segunda caixa. Qual é a probabilidade da bola retirada da segunda caixa ser verde?
  \begin{sol}
   \phantom{A} \\
   $\frac{4}{6}=$ probabilidade de 1 bola verde ser transferida para 2ª caixa \\
   2ª caixa fica com 4 verdes e 1 preta \\
   $\frac{4}{5}=$ probabilidade de sair 1 verde da 2ª caixa \\
   $\frac{2}{6}=$ probabilidade de 1 bola preta ser transferida para 2ª caixa \\
   2ª caixa fica com 3 verdes e 2 pretas \\
   $\frac{3}{5}=$ probabilidade de sair 1 verde na 2ª caixa \\
   $\Longrightarrow p= \frac{4}{6}\cdot\frac{4}{5}+\frac{2}{6}\cdot\frac{3}{5}=\frac{11}{15}$
  \end{sol}
\end{ex}