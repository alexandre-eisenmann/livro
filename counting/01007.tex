\begin{ex}
 	Dois tenistas A e B iam disputar um prêmio de US\$ 800 000,00 em 5 jogos e seria considerado vencedor aquele que ganhasse 3 ou mais jogos. Em cada jogo, ambos tinham chances iguais de vencer. Após os dois primeiros jogos, que foram vencidos por A, um mau tempo impediu a continuação da disputa e, então, decidiu-se repartir o prêmio. Do ponto de vista probabilístico era justo que:
    \begin{enumerate}[(a)]
    \item cada um recebesse metade do prêmio.
    \item A recebesse US\$ 600 000,00 e B o restante.
    \item A recebesse US\$ 700 000,00 e B o restante.
    \item A recebesse US\$ 750 000,00 e B o restante.
    \item A recebesse o prêmio integralmente.
    \end{enumerate}
      \begin{sol}
        resposta: c \\
        - 3º jogo: se A ganha, é o vencedor, se  perde tem nova partida. \\
        - 4º jogo: se A ganha, é o vencerdor, se perde fica empatado. \\
        - o 5º jogo decide a partida.\\
        Para B ser vencedor é necessário ganhar o 3º, 4º e 5º jogos, então:\\ $p_B=\frac{1}{2}\cdot\frac{1}{2}\cdot\frac{1}{2}=\frac{1}{8}$ e $p_A=\frac{7}{8}$. Logo $800.000 \div 8 = 100.000 \\ \Longrightarrow $ A deve receber 700.000
        
      \end{sol}
\end{ex}