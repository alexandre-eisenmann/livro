\begin{ex}
 	Imagine a cena: 10 pessoas chegam correndo para formar uma fila para entrar na aula de Matemática, onde assuntos muito importantes serão discutidos. Pela porta de entrada da sala passa uma pessoa de cada vez. Dentre essas 10 pessoas, 4 são homens e 6 são mulheres. Severino é um dos homens e Magali é uma das mulheres. Calcule:
    \begin{enumerate}[(a)]
    \item o total das diferentes ordens em que as pessoas podem adentrar o recinto da aula, considerando que as mulheres entrarão primeiro, uma vez que, afinal, elas tem prioridade.
    \item a probabilidade de que Severino seja o primeiro a entrar e Magali a última.
    \item o total das diferentes ordens em que as pessoas podem entrar na aula se Severino e Magali entram sempre juntos, um atrás do outro.
    \item a probabilidade de que Magali e Severino não consigam entrar na aula considerando que o professor fecha a porta depois da entrada de 8 pessoas.
    \end{enumerate}
     \begin{sol}
      \phantom{A}
        \begin{enumerate} [(a)]
            \item $ 6!\cdot4!=17280$
            \item $\frac{1}{10}\cdot\frac{1}{9}=\frac{1}{90}$
            \item $9!\cdot2!=725760$
            \item $\frac{8}{10}\cdot\frac{7}{9}\cdot\frac{6}{8}\cdot\frac{5}{7}\cdot\frac{4}{6}\cdot\frac{3}{5}\cdot\frac{2}{4}\cdot\frac{1}{3}=\frac{1}{45}$
        \end{enumerate}
     \end{sol}
\end{ex}