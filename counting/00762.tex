\begin{ex}
 	(UFMG) Considere uma prova de matemática constituída de 4 questões de múltipla escolha, com 4 alternativas cada uma, das quais apenas uma é correta.	Um candidato decide fazer essa prova escolhendo, aleatoriamente, uma alternativa em cada questão.
Então, é correto afirmar que a probabilidade de esse candidato acertar, nessa prova, exatamente uma questão é:
    \begin{enumerate}[(a)]
    \item $\frac{27}{64}$
    \item $\frac{27}{256}$
    \item $\frac{9}{64}$
    \item $\frac{9}{256}$
    \end{enumerate}
      \begin{sol}
       reposta: a \\
       probabilidade de acertar = $\frac{1}{4}$ e a de errar =$\frac{3}{4}$ \\
       $\frac{1}{4}\cdot\frac{3}{4}\cdot\frac{3}{4}\cdot\frac{3}{4}\cdot4=\frac{27}{64}$
      \end{sol}
\end{ex}