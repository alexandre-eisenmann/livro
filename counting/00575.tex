\begin{ex}
(Enem) \begin{center}\textbf{A vida como ela é}
\end{center}		
O Ministério do Desenvolvimento Social e Combate à Fome (MDS) realizou, em parceria com a ONU, uma pesquisa nacional sobre a população que vive na rua, tendo ouvido 31.922 pessoas em 71 cidades brasileiras. Nesse levantamento, constatou-se que a maioria da população sabe ler e escrever (74\%), que apenas 15,1\% vivem de esmolas e que, entre os moradores de rua que ingressaram no ensino superior, 0,7\% se diplomou. Outros dados da pesquisa são apresentados nos quadros a seguir.
\begin{center}
\includegraphics[width=15cm]{imagens/enem_ex_969.png}
\end{center}
I. As informações apresentadas no texto são suficientes para se concluir que:
   \begin{enumerate}[(a)]
   \item	As pessoas que vivem na rua e sobrevivem de esmolas são aquelas que nunca estudaram.
   \item 	As pessoas que vivem na rua e cursaram o ensino fundamental, completo ou incompleto, são aquelas que sabem ler e escrever.
   \item Existem pessoas que declararam mais de um motivo para estarem vivendo na rua.
   \item	Mais da metade das pessoas que vivem na rua e que ingressaram no ensino superior se diplomou.
   \item 	As pessoas que declararam o desemprego como motivo para viver na rua também declararam a decepção amorosa.
   \end{enumerate}
II.   	No universo pesquisado, considere que P seja o conjunto das pessoas que vivem na rua por motivo de alcoolismo/drogas e Q seja o conjunto daquelas cujo motivo de viverem na rua é a decepção amorosa.  Escolhendo-se ao acaso uma pessoa no grupo pesquisado e supondo-se que seja igual a 40\% a probabilidade de que essa pessoa faça parte do conjunto P ou do conjunto Q, então a probabilidade de que ela faça parte do conjunto intersecção de P e Q é igual a:
   \begin{enumerate}[(a)]
   \item 12\%
   \item 16\%
   \item 20\%
   \item 36\%
   \item 52\%
   \end{enumerate}
\begin{sol}
 \phantom{A} 
  \begin{enumerate}[I]  
 \item c 
 \item $(P\cup Q)= (P)+(Q)-( P\cap Q) \hspace{0,7cm} P=36\% \hspace{0,4cm}  Q=16\% \hspace{0,4cm} (P\cap Q) =40\% \\ \rightarrow 40=36+14-(P\cap Q)  \Longrightarrow (P\cap Q)=52-40=12\%$
  \end{enumerate}
\end{sol}
\end{ex}