\begin{ex}
No cadastro de um cursinho pré-vestibular estão registrados 600 alunos assim distribuídos:
   \begin{itemize}
   \item 380 rapazes
   \item 105 moças que já concluíram o ensino médio
   \item 200 rapazes que estão cursando o curso médio
   \end{itemize}
Um nome do cadastro é selecionado ao acaso. Qual é a probabilidade de o nome escolhido ser de:
   \begin{enumerate}[(a)]
   \item uma moça ?
   \item um rapaz que já concluiu o ensino médio?
   \item um rapaz ou de alguém que está cursando o ensino médio?
   \end{enumerate}
     \begin{sol}
       \phantom{A} 
       \begin{enumerate} [(a)]
           \item $\frac{220}{600}=\frac{11}{30}$
           \item $\frac{180}{600}=\frac{3}{10}$
           \item $\frac{380}{600}+\frac{115}{600}=\frac{495}{600}=\frac{33}{40}$
       \end{enumerate}
       
     \end{sol}
\end{ex}