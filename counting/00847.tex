\begin{ex}
 Num setor de um hospital trabalham 5 médicos e 10 enfermeiras. Quantas equipes de plantão, de 5 pessoas, podem ser formadas, garantindo que haja sempre, no mínimo um médico e uma enfermeira?
   \begin{sol}
    \phantom{A} \\
    médicos (m); enfermeiras (e) \\
    1(m) e 4(e) ou 2(m) e 3(e) ou 3(m) e 2(e) ou 4(m) e 1(e) \\
    $\mathrm{C}_{5,1}\cdot\mathrm{C}_{{10},4}+\mathrm{C}_{5,2}\cdot\mathrm{C}_{{10},3}+\mathrm{C}_{5,3}\cdot\mathrm{C}_{{10},2}+\mathrm{C}_{5,4}\cdot\mathrm{C}_{{10},1}=2750$ 
   \end{sol}
\end{ex}