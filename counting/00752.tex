\begin{ex}
 (Uff) Búzios são pequenas conchas marinhas   que     em outras épocas foram usadas como dinheiro e hoje são empregadas como enfeites, inclusive em pulseiras, colares e braceletes ou como amuletos ou em jogos de búzios. No jogo de búzios se considera que cada búzio admite dois resultados possíveis (abertura para baixo - búzio fechado ou abertura para cima - búzio aberto).
Suponha que seis búzios idênticos foram lançados simultaneamente e que a probabilidade de um búzio ficar fechado ao cair, ou ficar aberto, é igual a $\frac{1}{2}$.
Pode-se afirmar que a probabilidade de que fiquem três búzios abertos e três búzios fechados ao cair, sem se levar em consideração a ordem em que eles tenham caído, é:
    \begin{enumerate}[(a)]
    \item $\frac{5}{16}$
    \item $\frac{9}{32}$
    \item $\frac{15}{64}$
    \item $\frac{9}{64}$
    \item $\frac{3}{32}$
    \end{enumerate}
      \begin{sol}
      resposta: a  \\
      $(\frac{1}{2})^3\cdot(\frac{1}{2})^3\cdot\mathrm{C}_{6,3}=\frac{5}{16}$
      \end{sol}
\end{ex}