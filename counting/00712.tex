\begin{ex}
 De uma urna com 6 bolas brancas e 4 pretas são retiradas 5 bolas, uma de cada vez.
    \begin{enumerate}[(a)]
    \item qual a probabilidade de sair bola branca em exatamente duas dessas retiradas, com reposição de bolas?
    \item idem, sem reposição?
    \item considere agora que as 5 bolas retiradas, sem reposição  foram colocadas em uma outra urna, contendo 2 bolas brancas e 3 pretas. Em seguida uma bola é retirada da segunda urna e devolvida à primeira. Qual a probabilidade de não ter bola preta na primeira urna no final desse processo?
    \end{enumerate}
       \begin{sol}
          \phantom{A}  
         \begin{enumerate} [(a)]
             \item $\binom{5}{2}\cdot(\frac{6}{10})^2\cdot(\frac{4}{10})^3=\frac{144}{625}$
             \item $\binom{5}{2}\cdot\frac{6}{10}\cdot\frac{5}{9}\cdot\frac{4}{8}\cdot\frac{3}{7}\cdot\frac{2}{6}=\frac{5}{21}$
             \item \textit{p}(não ter preta na primeira urna) = \textit{p}(saírem as pretas da urna e não voltar nenhuma preta)\\
             $\frac{4}{10}\cdot\frac{3}{9}\cdot\frac{2}{8}\cdot\frac{1}{7}\cdot\frac{6}{6}\cdot5\cdot\frac{3}{10}=\frac{1}{140}$
         \end{enumerate}
       \end{sol}


\end{ex}