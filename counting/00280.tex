\begin{ex}
(UF – RJ) Para testar a eficácia de uma campanha de anúncio de lançamento de um novo sabão S, uma agência de propaganda realizou uma pesquisa com 2000 pessoas. Por uma falha na equipe, a agência omitiu os dados dos campos \textit{x, y, z, w}  no seu relatório sobre a pesquisa, conforme a tabela a seguir:
\begin{center}
\begin{tabular}{|c|c|c|c|} \hline
Nº de pessoas  &  Adquiriram & Não adquiriram S & TOTAL\\  \hline
Viram o anúncio & \textit{x} & \textit{y} & 1500 \\  \hline
Não viram o anúncio  & 200 & \textit{z}  & 500 \\  \hline
TOTAL & 600 & \textit{w} & 2000 \\  \hline
\end{tabular}
\end{center}
   \begin{enumerate}[(a)]
   \item indique os valores de \textit{x, y, z, w}
   \item suponha que uma dessas 2000 pessoas entrevistadas seja escolhida ao acaso e que todas as pessoas tenham a mesma probabilidade de serem escolhidas. Determine a probabilidade de que esta pessoa tenha visto o anúncio da campanha e adquirido o sabão S.
   \end{enumerate}
     \begin{sol}
       \phantom{A}
       \begin{enumerate} [(a)]
        \item \textit{x}=400; \textit{y}=1100; 
        \textit{z}=300; \textit{w}=1400
        \item $\frac{400}{2000}=\frac{1}{5}=20\%$
       \end{enumerate}
      \end{sol}
\end{ex}