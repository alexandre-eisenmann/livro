\begin{ex}
Um experimento aleatório tem espaço equiprovável E , finito e não vazio. Classifique  cada uma das afirmações a seguir como verdadeira (V) ou falsa(F).
   \begin{enumerate}[(a)]
   \item Se um evento $A$ desse espaço amostral é tal que $P(A)=1$, então  $A  = E$.
   \item Se um evento $A$ desse espaço amostral é tal que $P(A)=0$, então $A = \varnothing$. 
   \item Se um evento $A$ desse espaço amostral é tal que $P(A)=\frac{n+3}{6}$, então $n$ pode assumir o valor 4.
   \item Se $A$ e $\overline A$ são eventos complementares de E, com $P(A)=0,8$, então  $P(\overline A)=0,1$.
   \end{enumerate}
     \begin{sol}
      \phantom{A}
       \begin{enumerate} [(a)]
           \item V
           \item V
           \item F
           \item F
       \end{enumerate}
     \end{sol}
\end{ex}