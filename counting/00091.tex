\begin{ex}
  (Fuvest) Em um experimento probabilístico, Joana retirará aleatoriamente 2 bolas de uma caixa contendo bolas azuis e bolas vermelhas. Ao montar-se o experimento, colocam-se 6 bolas azuis na caixa. Quantas bolas vermelhas devem ser acrescentadas para que a probabilidade de Joana obter 2 azuis seja 1/3?
    \begin{enumerate}   [(a)]
        \item 2
        \item 4
        \item 6
        \item 8
        \item 10
    \end{enumerate}
      \begin{sol}
       resposta: b \\
       a caixa contém 6 bolas azuis e \textit{x} bolas vermelhas. \\
       probabilidade de retirar 2 bolas azuis:\hspace{0,3cm}$\frac{1}{3}=\frac{6}{x+6}\cdot\frac{5}{x+5} \Longrightarrow x=4$ 
      \end{sol}
 \end{ex}