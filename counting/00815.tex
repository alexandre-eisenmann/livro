\begin{ex}
  Em uma urna são colocadas 15 bolas, sendo 6 brancas, 5 azuis e 4 amarelas. São retiradas, sucessivamente e sem reposição duas bolas dessa urna. Se a  primeira bola é branca, calcule a probabilidade de a segunda ser:
    \begin{enumerate}[(a)]
    \item azul
    \item amarela
    \item branca
    \item não ser branca
    \end{enumerate}
      \begin{sol}
       \phantom{A}  
       \begin{enumerate}  [(a)]
           \item $\frac{5}{14}$
           \item $\frac{4}{14}=\frac{2}{7}$
           \item $\frac{5}{14}$
           \item $\frac{9}{14}$
       \end{enumerate}
      \end{sol}
\end{ex}