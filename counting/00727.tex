\begin{ex}
 (Unifesp) Suponha que Moacir esqueceu o número de telefone de seu amigo. Ele tem apenas 2 fichas suficientes para dois telefonemas.
    \begin{enumerate}[(a)] 
    \item se Moacir só esqueceu os 2 últimos dígitos mas sabe que a soma desses 2 dígitos é 15, encontre o número de possibilidades para os 2 últimos dígitos.
    \item se Moacir só esqueceu o último dígito e decide escolher um dígito ao acaso, encontre a possibilidade de acertar o número do telefone, com as duas tentativas.
    \end{enumerate}
      \begin{sol}
         \phantom{A}  
        \begin{enumerate} [(a)]
            \item 4 possibilidades: 7+8 ou 8+7 ou 6+9 ou 9+6
            \item acertar na primeira tentativa ou errar na primeira e acertar na segunda.\\
            $\frac{1}{10}+\frac{9}{10}\cdot\frac{1}{9}=\frac{1}{5}$
        \end{enumerate}
      \end{sol}
\end{ex}