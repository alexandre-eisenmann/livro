\begin{ex}
(Ufpr) Um dado é lançado duas vezes. No primeiro lançamento obtém-se um número \textit{b}, e no segundo lançamento obtém-se um número \textit{c}. Qual é a probabilidade de o polinômio $x^2 + bx +c = 0 $ não ter raiz real?
   \begin{enumerate}[(a)]
   \item $\frac{1}{4}$
   \item $\frac{11}{36}$
   \item $\frac{17}{36}$
   \item $\frac{1}{2}$
   \item $\frac{1}{3}$
   \end{enumerate}
     \begin{sol}
      resposta: c \\
      espaço amostral =36 \\
      para a equação não ter raiz real: \hspace{0,4cm} $b^2 - 4 c <0 \rightarrow b^2 < 4c$ \\
       $b=1\rightarrow 4c>1 \Rightarrow c= \{1, 2, 3, 4, 5, 6\} $\\
       $b=2 \rightarrow 4-4c<0 \Rightarrow c = \{2, 3, 4, 5, 6\} $\\
       $b=3 \rightarrow 9-4c<0 \Rightarrow c = \{  3, 4, 5, 6\} $\\
       $b=4\rightarrow c =\{ 5, 6\} $ \\
       são 17 possibilidades $\Longrightarrow p=\frac{17}{36}$
     \end{sol}
\end{ex}