\begin{ex}
 (FGV) Um garçom anotou as encomendas de 4 fregueses. Cada um pediu uma sopa, um prato principal, uma bebida e uma sobremesa. O garçom não anotou quais clientes pediram quais encomendas, lembrando-se apenas que cada um pediu uma sopa diferente, um prato principal diferente, uma bebida diferente e uma sobremesa diferente. De quantas maneiras distintas ele poderá distribuir os pedidos entre os quatro clientes? 
    \begin{enumerate}[(a)]
    \item $(4!)^4$
    \item 4 \text{x} 4!
    \item 4! \text{x} 4!
    \item $4^{16}$
    \item $\frac{16!}{4! \text{x} 4!}$
    \end{enumerate}
      \begin{sol}
       resposta: a \\
       4 tipos de sopas, 4 tipos de prato principal, 4 tipos de bebidas e 4 tipos de sobremesa.\\
       $4!\cdot4!\cdot4!\cdot4! = (4!)^4$
      \end{sol}
\end{ex}