\begin{ex}
 (ITA) Retiram-se três bolas de uma urna que contém 4 bolas verdes, 5 bolas azuis e 7 bolas brancas. Se $ P_A $ é a probabilidade de não sair bola azul e \textit{P} é a probabilidade de todas as bolas saírem com a mesma cor, então a alternativa que mais se aproxima de $P_A$ +\textit{P} é:
    \begin{enumerate}[(a)]
    \item 0,21
    \item 0,25
    \item 0,28
    \item 0,35
    \item 0,40
    \end{enumerate}
      \begin{sol}
       resposta: e \\
       $P_A=\frac{11}{16}\cdot\frac{10}{15}\cdot\frac{9}{14}=\frac{990}{3360}$\\
       $P=\frac{4}{16}\cdot\frac{3}{15}\cdot\frac{2}{14}+\frac{5}{16}\cdot\frac{4}{15}\cdot\frac{3}{14}+\frac{7}{16}\cdot\frac{6}{15}\cdot\frac{5}{14}=\frac{294}{3360}$ \\
       $\Longrightarrow P_A+P=\frac{990+294}{3360}=\frac{1284}{3360}\approx0,38$
      \end{sol}
\end{ex}