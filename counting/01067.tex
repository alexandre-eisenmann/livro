\begin{ex}
 	 (Cesgranrio) Com os algarismos 1, 2, 3, 4, 5 e 6 formam-se números naturais de 6 algarismos distintos. Sabendo-se que neles não aparecem juntos dois algarismos pares nem 2 algarismos ímpares, então o número total de naturais assim formados é:
    \begin{enumerate}[(a)]
    \item 36
    \item 48
    \item 60
    \item 72
    \item 90
    \end{enumerate}
      \begin{sol}
        resposta: d \\
        $\frac{P}{3}\frac{I}{3}\frac{P}{2}\frac{I}{2}\frac{P}{1}\frac{I}{1}+\frac{I}{3}\frac{P}{3}\frac{I}{2}\frac{P}{2}\frac{I}{1}\frac{P}{1}= 3!\cdot3!\cdot2=72$
      \end{sol}
\end{ex}