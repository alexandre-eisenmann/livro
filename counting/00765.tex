\begin{ex}
 (UFSC) Qual é a soma dos números correspondentes às alternativas corretas?\\
(01) Considerando-se um hexágono regular e tomando-se ao acaso uma das retas determinadas pelos seus vértices, a probabilidade de que a reta passe pelo centro do hexágono é $\frac{1}{8}$ .\\
(02) Se 5 atletas disputam uma prova corrida de 800 metros, então o número de resultados possíveis para os dois primeiros lugares, sem que haja empates, é 10.\\
(04) Antonio, Cláudio, Carlos e Ivan montaram uma empresa de prestação de serviços e decidiram que o nome da empresa será a sigla formada pelas iniciais dos seus nomes, por exemplo, CACI. O número de siglas possíveis é 12.\\
(08) Quando 7 pessoas se encontram e todas se cumprimentam, o número de apertos de mão possível sem que os cumprimentos se repitam, é 42.
  \begin{sol}
    \phantom{A}  \\
   (01) F \hspace{0.3cm} (02) $5\cdot4=20$ F\hspace{0.3cm} (04) $\frac{4!}{2!}=12$  V \hspace{0.3cm} (08) $\mathrm{C}_{7,2}=21$ V\\
   $\Longrightarrow 8+4=12$
  \end{sol}
\end{ex}