\begin{ex}
(Unesp – SP) Numa cidade com 30 000 domicílios, 10 000 recebem regularmente o jornal da loja de eletrodomésticos X; 8 000 recebem o jornal do supermercado Y e metade do número de domicílios não recebe nenhum dos dois jornais. Determine:
   \begin{enumerate}[(a)]
   \item o número de domicílios que recebem os dois jornais;
   \item a probabilidade de um domicílio da cidade, escolhido ao acaso, receber o jornal da loja de eletrodomésticos X e não receber o jornal do supermercado Y.
   \end{enumerate}
     \begin{sol}
       \phantom{A} \\ \\
        \begin{venndiagram2sets} [labelA=\(X\),labelB=\(Y\),  labelOnlyA=7000,labelOnlyB=5000,labelAB=3000,labelNotAB=15000]
        \end{venndiagram2sets}
         \begin{enumerate} [(a)]
             \item $10000+8000+15000-(\mathrm{X}\cap\mathrm{Y})=30000 \Longrightarrow \mathrm{X}\cap\mathrm{Y}=3000$
             \item $\frac{7000}{30000}=\frac{7}{30}$
         \end{enumerate}
     \end{sol}
\end{ex}