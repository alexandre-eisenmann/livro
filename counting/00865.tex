\begin{ex}
 (UFSCar) Os resultados de 1200 lançamentos de um dado estão dispostos na listagem abaixo:

\begin{center}
\begin{tabular}{|c|c|c|c|c|c|c|}   \hline
Nº de uma face & 1   & 2   & 3   &   4 &   5 & 6  \\ \hline  
frequência     & 100 & 200 & 200 & 300 & 100 & 300 \\  \hline
\end{tabular} 
\end{center}

Admitindo-se, para dois novos  lançamentos  desse dado as mesmas condições experimentais anteriores , tem-se: 
    \begin{enumerate}[(a)]
    \item pelo menos uma ocorrência da face número 4;
    \item a ocorrência de uma face 4 ou 6;
    \item que a probabilidade da ocorrência de pelo menos uma face 6 é $\frac{7}{16}$;
    \item a ocorrência de uma face par;
    \item que a probabilidade de uma face par é $\frac{1}{2}$.
    \end{enumerate}
      \begin{sol}
       resposta: c \\
       probabilidades das faces : $f_1=\frac{1}{12};\hspace{0,3cm}
       f_2 =\frac{1}{6};\hspace{0,3cm}
       f_3 =\frac{1}{6};\hspace{0,3cm}
       f_4 = \frac{1}{4};\hspace{0,3cm}
       f_5 = \frac{1}{12};\hspace{0,3cm}
       f_6 = \frac{1}{4}$ \\
        ítem c: ocorrer uma face 6 num lançamento e no outro qualquer face ou ocorrer face 6 nos 2 lançamentos:
       $\frac{1}{4}\cdot\frac{3}{4}\cdot2+\frac{1}{4}\cdot\frac{1}{4}=\frac{7}{16}$
       
      \end{sol}
\end{ex}