\begin{ex}
   (Fatec) Um aprendiz de feiticeiro, numa experiência investigativa, tem a sua disposição cinco substâncias distintas entre as quais deverá escolher três distintas para fazer uma poção. No entanto, duas dessas cinco substâncias, quando misturadas, anulam qualquer efeito reativo. A probabilidade do aprendiz obter uma poção sem efeito reativo é:
     \begin{enumerate} [(a)]
         \item 20\%
         \item 30\%
         \item 40\%
         \item 50\%
         \item 60\%
     \end{enumerate}
      \begin{sol}
       resposta: b \\
       Das 5 substâncias, 2 anulam o efeito reativo, então usamos somente 3 substâncias.\\
        $\Longrightarrow p=\frac{3}{\mathrm{C}_{5,3}}=\frac{3}{10}=30\%$
      \end{sol}
  \end{ex}