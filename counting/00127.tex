\begin{ex}
(Fuvest) Um apreciador deseja adquirir, para sua adega, 10 garrafas de vinho de um lote constituído por 4 garrafas da Espanha, 5 garrafas da Itália e 6 garrafas da França, todas de diferentes marcas.
   \begin{enumerate}[(a)]
   \item  de quantas maneiras é possível escolher 10 garrafas deste lote?
   \item de quantas maneiras é possível escolher 10 garrafas do lote, sendo duas garrafas da Espanha, 4 da Itália e 4 da França?
   \item qual é a probabilidade de que escolhidas ao acaso, 10 garrafas do lote, haja exatamente 4 garrafas da Itália e, pelo menos, uma garrafa, de cada  um dos outros países?
   \end{enumerate}
     \begin{sol}
       \phantom{A}
       \begin{enumerate} [(a)]
           \item total de garrafas: 4+5+6 =15  $\Longrightarrow \mathrm{C}_{{15},{10}}=3003$
           \item $\mathrm{C}_{4,2}\cdot\mathrm{C}_{5,4}\cdot\mathrm{C}_{6,4}=450$
           \item 
            4 da Itália e pelo menos 1 dos outros países: $\mathrm{C}_{5,4}\cdot(\mathrm{C}_{{10},6}-1)=1045$ \\
           $ \Longrightarrow p=\frac{1045}{3003}=\frac{95}{273}$
       \end{enumerate}
     \end{sol}
     
\end{ex}