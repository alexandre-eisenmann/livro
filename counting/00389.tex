\begin{ex}
(Puc-SP) Os 36 cães existentes em um canil são apenas de três raças: poodle, dálmata e boxer. Sabe-se que o total de cães das raças poodle e dálmata excede o número de cães da raça boxer em 6 unidades, enquanto que o total de cães das raças dálmata e boxer é o dobro do número dos cães de raça poodle. Nessas condições, escolhendo-se ao acaso, um cão desse canil, a probabilidade de ele ser da raça poodle é:
   \begin{enumerate}[(a)]
   \item $\frac{1}{4}$
   \item $\frac{1}{3}$
   \item $\frac{5}{12}$
   \item $\frac{1}{2}$
   \item $\frac{2}{3}$
   \end{enumerate}
     \begin{sol}
       resposta: b \\
       P+D=B+6 \hspace{0,5cm} D+B=2P \hspace{0,5cm} P+D+B=36 \hspace{0,5cm} então P=12 \\ 
       $\Longrightarrow p=\frac{12}{36}=\frac{1}{3}$
     \end{sol}
\end{ex}