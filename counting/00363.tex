\begin{ex}
(Unesp) Através de fotografias de satélites de certa região da floresta amazônica, pesquisadores fizeram um levantamento das áreas de floresta (F) e não floresta(D) dessa região, nos anos de 2004 e 2006. Com base nos dados levantados, os pesquisadores elaboraram a seguinte matriz de probabilidade:\\
\begin{center}
    \includegraphics[height=3cm]{imagens/IMG_8380.jpg}
\end{center}
Por exemplo: a probabilidade de uma área de não floresta (D) no ano de 2004 continuar a ser não floresta (D) no ano de 2006 era de 0,98. Outro exemplo: a probabilidade de uma área de floresta (F) em 2004 passar a área de não floresta(D) em 2006 era de 0,05. Supondo que a matriz de probabilidade se manteve a mesma do ano de 2006 para o ano de 2008, determine a probabilidade de uma área de floresta(F) dessa região em 2004 passar a ser de não floresta (D) em 2008.
 \begin{sol}
  \phantom{A} \\
  A probabilidade (P) de uma área F em 2004 passar a ser D em 2008 pode acontecer de 2 maneiras:
  $\text{F}(2004)\rightarrow \text{F}(2006) \rightarrow \text{D}(2008)= 0,95\cdot0,05=0,0475\\
  \textbf{ou}\hspace{0,4cm} \text{F}(2004)\rightarrow \text{D}(2006) \rightarrow \text{D}(2008)= 0,05\cdot0,98=0,049$ \\
  $\Longrightarrow \text{P}=0,0475+0,049=0,0965=9,65\%$
 \end{sol}
\end{ex}