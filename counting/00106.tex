\begin{ex}
  (Unicamp) Uma escola com 960 alunos decidiu renovar seu mobiliário. Para decidir quantas cadeiras de canhotos será necessário comprar, fez-se um levantamento do número de alunos canhotos em cada turma. A tabela abaixo indica, na segunda linha, o número de turmas com o total de canhotos indicado na primeira linha. \\
    \begin{center}
        \begin{tabular}{|c|c|c|c|c|c|c|}  \hline
            nº total de alunos canhotos &0&1&2&3&4&5  \\ \hline
           nº de turmas  &1&2&5&12&8&2 \\  \hline
        \end{tabular}
    \begin{enumerate} [(a)]
        \item qual a probabilidade de que uma turma escolhida ao acaso tenha pelo menos 3 alunos canhotos?
        \item qual a probabilidade de que um aluno escolhido ao acaso na escola seja canhoto? 
    \end{enumerate}
      \begin{sol}
       \phantom{A}
       \begin{enumerate} [(a)]
           \item pelo menos 3 canhotos: 12 turmas com 3 canhotos, 8 turmas com 4 e 2 turmas com 5 canhotos, num total de 30 turmas.
           $\Longrightarrow p=\frac{22}{30}=73,33\%$
           \item número de cachotos: $1\cdot2+5\cdot2+12\cdot3+8\cdot4+2\cdot5=90$\hspace{0,4cm} $\Longrightarrow p=\frac{90}{960}=\frac{3}{32}=9,375\%$
       \end{enumerate}
      \end{sol}
    \end{center}
\end{ex}