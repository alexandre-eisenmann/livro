\begin{ex}
 (UE-RJ) Os números naturais de 1  a 10 foram escritos , um a um sem repetição, em 10 bolas de pingue- pongue. Se duas delas forem escolhidas ao acaso, o valor mais provável da soma dos números sorteados é igual a :
    \begin{enumerate}[(a)]
    \item 9
    \item 10
    \item 11
    \item 12
    \end{enumerate}
      \begin{sol}
        resposta: c \\
        somas possíveis: 3, 4, 5, 6, 7, 8, 9, 10, 11\hspace{0,2cm} ou\hspace{0,2cm}5, 6, 7, 8, 9, 10, 11, 12\hspace{0,2cm} ou \\
        7, 8, 9 , 10, 11, 12, 13 \hspace{0,2cm}ou\hspace{0,2cm} 9, 10, 11, 12, 13, 14\hspace{0,2cm}ou\hspace{0,2cm}11, 12, 13, 14, 15\hspace{0,2cm}ou\hspace{0,2cm}13, 14, 15, 16
        apareceram 5 números 11
      \end{sol}
\end{ex}