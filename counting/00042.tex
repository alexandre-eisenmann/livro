\begin{ex}
  (Enem) Doze times se inscreveram em um torneio de futebol amador. O jogo de abertura do torneio foi escolhido da seguinte forma: primeiro foram sorteados 4 times para compor o Grupo A. Em seguida, entre os times do Grupo A, foram sorteados 2 times para realizar o jogo de abertura do torneio, sendo que o primeiro deles jogaria em seu próprio campo, e o segundo seria o time visitante. \\
  A quantidade total de escolhas possíveis para o Grupo A e a quantidade total de escolhas dos times do jogo de abertura podem ser calculadas através de:
    \begin{enumerate}  [(a)]
        \item uma combinação e um arranjo, respectivamente.
        \item um arranjo e uma combinação, respectivamente.
        \item um arranjo e uma permutação, respectivamente.
        \item duas combinações.
        \item dois arranjos.
    \end{enumerate}
      \begin{sol}
      resposta: a \\
      Primeiro foram sorteados 4 times  entre os 12 (combinação).
      Depois devem ser sortedos 2 times entre 4 de modo que o primeiro joga no seu campo e o segundo, como visitante ( importa a ordem = arranjo).
      \end{sol}
 \end{ex}