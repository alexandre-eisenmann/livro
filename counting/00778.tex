\begin{ex}
 (Cescem) De um total de 100 alunos que se inscreveram para os vestibulares das faculdades de Matemática, Física e Química, sabe-se que:\\
- 30 inscreveram-se para prestar Matemática e destes, 20 são do sexo masculino.\\
- o total de alunos do sexo masculino é 50, dos quais 10 inscreveram-se em Química.\\
- existem 10 moças que se inscreveram para o vestibular de Química.\\
Nestas condições, sorteando-se um aluno e sabendo-se que é do sexo feminino, a probabilidade de que ele seja inscrito para o vestibular de Matemática é de:
    \begin{enumerate}[(a)]
    \item $\frac{1}{5}$
    \item $\frac{1}{4}$
    \item $\frac{1}{3}$
    \item $\frac{1}{2}$
    \item 1
    \end{enumerate}
      \begin{sol}
       resposta: a \\
       10 mulheres inscritas em Química no total de 50 mulheres $\Longrightarrow \frac{10}{50}=\frac{1}{5}$
      \end{sol}
\end{ex}