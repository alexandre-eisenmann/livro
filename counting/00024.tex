\begin{ex}
  (Enem) Para cadastrar-se em um site, uma pessoa precisa escolher uma senha composta por quatro caracteres, sendo dois algarismos e duas letras (maiúsculas ou minúsculas). As letras e os algarismos podem estar em qualquer posição. Essa pessoa sabe que o alfabeto é composto por vinte e seis letras e que uma letra maiúscula difere da minúscula em uma senha.
  O número total de senhas possíveis para o cadastramento nesse site é dado por:
    \begin{enumerate} [(a)]
        \item ${10}^2 . {26}^2$
        \item ${10}^2 . {52} ^2$
        \item ${10}^2 . {52} ^2 . \frac{4!}{2!}$
        \item ${10}^2 . {26} ^2 . \frac{4!}{2! . 2!}$
        \item ${10}^2 . {52}^2 . \frac{4!}{2! . 2!}$
    \end{enumerate}
     \begin{sol}
      resposta: e \\
      A = algarismo, L = letras \\
      total de letras: 26 minúsculas e 26 maiusculas= 52 letras \\
      A A L L ( permutação com repetição ) =  $\frac{4!}{2!2!}$ \\
      número total de senhas: $10^2\cdot52^2\cdot\frac{4!}{2!21}$
     \end{sol}
 \end{ex}