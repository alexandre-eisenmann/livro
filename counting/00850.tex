\begin{ex}
 Um baralho tem 12 cartas distintas, das quais 4 são “reis”. Retirando-se sucessivamente 3 cartas ao acaso, sem reposição, qual a probabilidade de sair:
    \begin{enumerate}[(a)]
    \item um "rei", apenas na primeira carta retirada?
    \item apenas um "rei", entre as cartas retiradas?
    \item pelo menos um "rei" entre as cartas retiradas?
    \end{enumerate}
      \begin{sol}
       \phantom{A}
       \begin{enumerate} [(a)]
           \item $\frac{4}{12}\cdot\frac{8}{11}\cdot\frac{7}{10}=\frac{28}{165}$
           \item $\frac{4}{12}\cdot\frac{8}{11}\cdot\frac{7}{10}\cdot3=\frac{28}{55}$
           \item sem os "reis":  $\frac{8}{12}\cdot\frac{7}{11}\cdot\frac{6}{10}= \frac{14}{55}$ \\
           total  (-)  sem os reis $\Longrightarrow 1-\frac{14}{55}= \frac{41}{55}$
           
       \end{enumerate}
      \end{sol}
\end{ex}