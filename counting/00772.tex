\begin{ex}
  (UFMG) Um clube resolve fazer uma Semana de Cinema. Para isso, os organizadores escolhem 7 filmes, que serão exibidos um por dia. Porém, ao elaborar a programação, eles decidem que três desses filmes, que são de ficção científica, devem ser exibidos em dias consecutivos. Nesse caso, o número de maneiras diferentes de se fazer a programação dessa semana é: 
    \begin{enumerate}[(a)]
    \item 144
    \item 576
    \item 720
    \item 1040
    \item 288
    \end{enumerate}
      \begin{sol}
       resposta: c \\
         \begin{tikzpicture}
         \tracos{7}{0.5cm}
         \juntos{1}{3}{5}
         \tcima{1}{0.5cm}{F}
         \tcima{2}{0.5cm}{F}
         \tcima{3}{0.5cm}{F}
         \tbaixo{4}{0.5cm}{4}
         \tbaixo{5}{0.5cm}{3}
         \tbaixo{6}{0.5cm}{2}
         \tbaixo{7}{0.5cm}{1}
         \end{tikzpicture}
         \\
         $\Longrightarrow5!\cdot3!=720$
      \end{sol}
\end{ex}