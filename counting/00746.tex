\begin{ex}
 (FGV) Dois jogadores de pingue-pongue X e Y jogaram entre si, no passado, muitas partidas e cada um ganhou metade das partidas disputadas. Na rodada final de um torneio recente, os mesmos jogadores, X e Y disputam o prêmio de R\$ 600,00. Segundo as regras, partidas serão realizadas até que um dos jogadores consiga 3 vitórias, sendo declarado o melhor do torneio. Entretanto, quando X tinha duas  vitórias e Y tinha uma, faltou luz no local e a rodada foi interrompida. Na impossibilidade de adiar para outro dia, o diretor do torneio determinou que o prêmio fosse dividido entre os dois finalistas. Qual é a forma correta de dividir o prêmio entre os dois jogadores?
   \begin{sol}
     \phantom{A} \\
     Y ganha se vencer duas partidas em seguida: $p(\mathrm{Y})=\frac{1}{2}\cdot\frac{1}{2}=\frac{1}{4}\Longrightarrow p(\mathrm{X})=\frac{3}{4}$ \\
     $600,00 \div4=150,00\Longrightarrow$ X deve ganhar R\$ 450,00 e Y R\$ 150,00
   \end{sol}
\end{ex}