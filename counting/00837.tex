\begin{ex}
 (Enem) Em um concurso de televisão, apresentam-se ao participante 3 fichas voltadas para baixo, estando representadas em cada uma delas as letras T, V e E. As fichas encontram-se alinhadas em uma ordem qualquer. O participante deve ordenar as fichas a seu gosto, mantendo as letras voltadas para baixo, tentando obter a sigla TVE. Ao desvirá-las, para cada letra que esteja na posição correta ganhará um prêmio de R\$ 200,00. A probabilidade do participante não ganhar qualquer prêmio é igual a :
    \begin{enumerate}[(a)]
    \item 0
    \item $\frac{1}{3}$
    \item $\frac{1}{4}$
    \item $\frac{1}{2}$
    \item $\frac{1}{6}$
    \end{enumerate}
      \begin{sol}
        resposta: b \\
        6 possibilidades: TVE, TEV, VTE, EVT, VET, ETV \\
        não ganham nada: VET e ETV \\
        $p=\frac{2}{6}=\frac{1}{3}$
      \end{sol}
\end{ex}