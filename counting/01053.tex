\begin{ex}
 (UCDB-MS) Um grupo de 100 pessoas apresenta as seguintes características:
\begin{center}
\begin{tabular}{|c|c|c|} \hline
& Mulheres & Homens \\ \hline
Estudantes & 15 & 35 \\ \hline
Não Estudantes & 5 & 45 \\ \hline
\end{tabular}
\end{center}
Sendo escolhida ao acaso uma pessoa desse grupo, a probabilidade de ser mulher estudante ou homem não estudante é:
    \begin{enumerate}[(a)]
    \item $\frac{7}{10}$
    \item $\frac{1}{2}$
    \item $\frac{3}{5}$
    \item $\frac{1}{5}$
    \item $\frac{3}{20}$
    \end{enumerate}
      \begin{sol}
        resposta: c \\
        $\frac{15}{100}+\frac{45}{100}=\frac{60}{100}=\frac{3}{5}$
      \end{sol}
\end{ex}