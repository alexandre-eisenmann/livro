\begin{ex}
(UERJ) Um pesquisador possui em seu laboratório um recipiente contendo 100 exemplares de “Aedes aegypti”, cada um deles contaminado com apenas um dos tipos de vírus, de acordo com a tabela abaixo. Retirando-se ao acaso 2 mosquitos desse recipiente, qual é a probabilidade de que pelo menos um esteja contaminado como tipo DEN 3 ?\\
\begin{center}
   \begin{tabular}{|c|c|} \hline
      Tipo   & Quantidade mosquitos \\  \hline
       DEN 1 &  30 \\  \hline
       DEN 2 &  60 \\  \hline
       DEN 3 &  10 \\  \hline
   \end{tabular}
\end{center}
 \begin{sol}
  \phantom{A} \\
  pelo menos um contaminado = 1 - \textit{p}(nenhum contaminado) \\
  $p=1-\frac{90}{100}\cdot\frac{89}{99}=\frac{21}{110}$
 \end{sol}
\end{ex}