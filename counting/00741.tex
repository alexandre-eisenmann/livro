\begin{ex}
 (PUCUSP) Um marceneiro pintou de azul todas as faces de um bloco maciço de madeira e, em seguida dividiu-o totalmente em pequenos cubos de 10 cm de aresta. Considerando-se que as dimensões do bloco eram de 140 cm por 120 cm por 90 cm, então a probabilidade de escolher-se aleatoriamente um dos cubos obtidos após a divisão e nenhuma das faces estar pintada de azul é:
    \begin{enumerate}[(a)]
    \item $\frac{1}{3}$
    \item $\frac{5}{9}$
    \item $\frac{2}{3}$
    \item $\frac{5}{6}$
    \item $\frac{8}{9}$
    \end{enumerate}
      \begin{sol}
        resposta: b \\
        número total de bloquinhos=  $\frac{140\cdot120\cdot90}{10\cdot10\cdot10}=1512$ \\
        número de cubos sem pintura: subtrai-se 2 de cada lado $\rightarrow 12\cdot10\cdot7=840$  \\
        $p=\frac{840}{1512}=\frac{5}{9}$
        
      \end{sol}
\end{ex}