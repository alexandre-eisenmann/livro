\begin{ex}
 (Enem) Um rapaz estuda em uma escola que fica longe de sua casa, e por isso precisa utilizar o transporte público. Como é muito observador, todos os dias ele anota a hora exata (sem considerar os segundos) em que o ônibus passa pelo ponto de espera. Também notou que nunca consegue chegar ao ponto de ônibus antes de 6h 15min da manhã. Analisando os dados coletados durante o mês de fevereiro, o  qual teve 21 dias letivos, ele concluiu que 6 h 21min foi o que mais se repetiu, e que a mediana do conjunto de dados é 6 h 22min.
 A probabilidade de que, em algum dos dias letivos de fevereiro, esse rapaz tenha apanhado o ônibus antes de 6 h 21min da manhã é, no máximo:
    \begin{enumerate} [(a)]
        \item $\frac{4}{21}$
        \item $\frac{5}{21}$
        \item $\frac{6}{21}$
        \item $\frac{7}{21}$
        \item $\frac{8}{21}$
    \end{enumerate}
      \begin{sol}
      resposta: d \\
      \begin{tikzpicture}
         \tracos{11}{1.1cm}
         \tcima{1}{1.1cm}{6h15}
         \tcima{2}{1.1cm}{6h16}
         \tcima{3}{1.1cm}{6h17}
         \tcima{4}{1.1cm}{6h18}
         \tcima{5}{1.1cm}{6h19}
         \tcima{6}{1.1cm}{6h20}
         \tcima{7}{1.1cm}{6h20}      \tcima{8}{1.1cm}{6h21}
         \tcima{9}{1.1cm}{6h21}
         \tcima{10}{1.1cm}{6h21}
         \tcima{11}{1.1cm}{6h22}
      \end{tikzpicture}
      \\
      6h22min é a mediana e 6h21min é a moda ( o nº que mais se repete)\\
      a probabilidade do rapaz ter apanhado o ônibus antes de 6h21min é:\hspace{0,3cm}  $p=\frac{7}{21}$ 
      \end{sol}
 \end{ex}