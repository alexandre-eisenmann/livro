\begin{ex}
(Puc) Um jogo de crianças consiste em lançar uma caixa de fósforos sobre uma mesa. Ganha quem conseguir fazer com que a caixa fique apoiada sobre sua menor face. Suponha que a probabilidade de uma face ficar apoiada sobre a mesa é proporcional à sua área e que a constante de proporcionalidade é a mesma para cada face. Se a dimensões da caixa são: 2 cm, 4 cm e 8 cm, qual é a probabilidade de a caixa ficar apoiada sobre a sua face menor?
  \begin{sol}
    \phantom{A} \\
    Áreas: $8cm^2; 16cm^2; 32 cm^2$
    $\Longrightarrow p=\frac{8}{8+16+32}=\frac{8}{56}=\frac{1}{7}$
  \end{sol}
\end{ex}