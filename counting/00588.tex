\begin{ex}
Carlos e Tobias são candidatos às duas vagas existentes no departamento de recursos humanos de uma empresa. Após a divulgação do resultado do teste realizado, em que outros candidatos além dos dois foram aprovados, Carlos avaliou que a probabilidade de ser ele um dos escolhidos é 60\%, e Tobias avaliou que a probabilidade de ser ele o escolhido é 70\%. Admitindo que essas avaliações estejam corretas, a probabilidade de pelo menos um dos dois de ser escolhido é:
   \begin{enumerate}[(a)]
   \item 46\%
   \item 38\%
   \item 54\%
   \item 88\%
   \item 36\%
   \end{enumerate}
    \begin{sol}
     resposta: d \\
     pelo menos 1 dos 2: Carlos (sim) e Tobias (não) ou Carlos (não) e Tobias (sim) ou Carlos (sim) e Tobias (sim) \\
     $0,60\cdot0,30+0,40\cdot0,70+0,60\cdot0,70=0,88=88\%$
    \end{sol}
\end{ex}