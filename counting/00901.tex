\begin{ex}
 	(Unesp) Na convenção de um partido para lançamento da candidatura de uma chapa ao governo de certo estado havia 3 possíveis candidatos a governador, sendo 2 homens e uma mulher, e 6 possíveis candidatos a vice-governador, sendo 4 homens e 2 mulheres. Ficou estabelecido que a chapa governador/vice-governador, seria formada por duas pessoas de sexos opostos. Sabendo que os 9 candidatos são distintos, o número de maneiras possíveis de se formar a chapa é:
    \begin{enumerate}[(a)]
    \item 18
    \item 12
    \item 8
    \item 6
    \item 4
    \end{enumerate}
      \begin{sol}
       resposta: c \\
       governador/vice-governador: 1 homem e 1 mulher ou 1 mulher e 1 homem \\
       $\mathrm{C}_{2,1}\cdot\mathrm{C}_{2,1}+\mathrm{C}_{1,1}\cdot\mathrm{C}_{4,1}=8$
      \end{sol}
\end{ex}