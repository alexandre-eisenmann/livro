\begin{ex}
	(Fatec) Com os dígitos 1, 2, 3, 4 e 5 deseja-se formar números com 5 algarismos não repetidos, de modo que o 1 sempre preceda o 5. A quantidade de números assim constituídos é:
    \begin{enumerate}[(a)]
    \item 66
    \item 54
    \item 78
    \item 50
    \item 60
    \end{enumerate}
     \begin{sol}
       resposta: e \\
       colocando o número 1 antes do 5, temos:\\
       $\frac{1}{\phantom{A}}\frac{\phantom{A}}{\phantom{A}}\frac{\phantom{A}}{\phantom{A}}\frac{\phantom{A}}{\phantom{A}}\frac{\phantom{A}}{\phantom{A}}$ : 4 opções para o 5  \hspace{0,5 cm} $\frac{\phantom{A}}{\phantom{A}}\frac{1}{\phantom{A}}\frac{\phantom{A}}{\phantom{A}}\frac{\phantom{A}}{\phantom{A}}\frac{\phantom{A}}{\phantom{A}}$ : 3 opções para o 5 \\
       $\frac{\phantom{A}}{\phantom{A}}\frac{\phantom{A}}{\phantom{A}}\frac{1}{\phantom{A}}\frac{\phantom{A}}{\phantom{1}}\frac{\phantom{A}}{\phantom{A}}$ : 2 opções para o 5  \hspace{0,5 cm}
       $\frac{\phantom{A}}{\phantom{A}}\frac{\phantom{A}}{\phantom{A}}\frac{\phantom{A}}{\phantom{A}}\frac{1}{\phantom{A}}\frac{\phantom{A}}{\phantom{A}}$ : 1 opção para o 5 \\
       $4+3+2+1= 10$ opções com os algarismo 1 e 5 e  para os algarismos 2, 3 e 4 temos $3!$ de opções\\
       $\Longrightarrow 10\cdot3!=60$
     \end{sol}
\end{ex}