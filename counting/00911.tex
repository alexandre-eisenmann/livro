\begin{ex}
 (Puccamp) Em uma escola, 10 alunos  (6 rapazes e 4 garotas) apresentam-se para compor a diretoria do Grêmio Estudantil, que deverá ter os seguintes membros: um presidente, um vice-presidente e dois secretários. Os nomes dos candidatos são colocados em uma urna, da qual serão sorteados os membros que comporão a diretoria. A probabilidade de que na equipe sorteada o presidente ou o vice-presidente sejam do sexo masculino  é:
    \begin{enumerate}[(a)]
    \item $\frac{1}{3}$
    \item $\frac{4}{5}$
    \item $\frac{5}{6}$
    \item $\frac{13}{15}$
    \item $\frac{27}{30}$
    \end{enumerate}
      \begin{sol}
       resposta: d \\
       total - (presidente e vice mulheres) 
       $ \Longrightarrow1- (\frac{4}{10}\cdot\frac{3}{9})=\frac{13}{15}$
      \end{sol}
\end{ex}