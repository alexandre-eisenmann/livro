\begin{ex}
(Unicamp) Três candidatos A,B e C concorrem à presidência de um clube. Uma pesquisa aponta que dos sócios entrevistados, 150 não pretendem votar. Dentre os entrevistados que estão dispostos a participar da eleição, 40 sócios votariam apenas no candidato A, 70 votariam apenas em B e 100 votariam apenas em C. Além disso, 190 disseram que não votariam em A, 110 disseram que não votariam em C, e 10 sócios estão na dúvida e podem votar tanto em A como em C, mas não em B. Finalmente, a pesquisa revelou que 10 entrevistados votariam em qualquer candidato. Com base nesses dados, pergunta-se:
   \begin{enumerate}[(a)]
   \item quantos sócios entrevistados estão em dúvida entre votar em B ou em C, mas não votariam em A?
   \item dentre os sócios consultados que pretendem participar da eleição, quantos não votariam em B?
   \item quantos sócios participaram da eleição?
   \item suponha que a pesquisa represente as intenções de voto de todos os sócios do clube. Escolhendo um sócio ao acaso, qual a probabilidade de que ele vá participar da eleição, mas ainda não tenha se decidido por um único candidato?
   \end{enumerate}
    \begin{sol}
     \phantom{A} \\ \\
       \begin{venndiagram3sets} [ labelOnlyA=40,labelOnlyB=70,labelOnlyC=100,labelNotABC=150,labelABC=10,labelOnlyAB=0,labelOnlyBC=20,labelOnlyAC=10]
       
       \end{venndiagram3sets}
       \begin{enumerate} [(a)]
           \item 20
           \item 150
           \item $40+10+10+70+20+100+150=400$
           \item $\frac{250}{400}\cdot\frac{40}{250}=10\%$
       \end{enumerate}
    \end{sol}
\end{ex}