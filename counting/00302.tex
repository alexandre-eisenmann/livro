\begin{ex}
 (Ucsal – BA) Uma escola de línguas tem somente alunos de inglês e espanhol, nenhum deles estudando as duas línguas. Do total de alunos 20\% estudam espanhol, 65\% são do sexo feminino e 30\% são do sexo masculino e estudam inglês. Se escolhermos ao acaso um aluno dessa escola, a probabilidade de ele ser do sexo feminino  e estudar inglês é:
   \begin{enumerate}[(a)]
   \item $\frac{1}{2}$
   \item $\frac{7}{20}$
   \item $\frac{3}{10}$
   \item $\frac{3}{20}$
   \item $\frac{1}{20}$
   \end{enumerate}
     \begin{sol}
       resposta: a \\
       20\% estudam espanhol, logo 80\% estudam inglês e 30\% do sexo masculino estudam inglês. \\
       Sendo \textit{x} = sexo feminino que estuda inglês, temos:
       $80\% = 30\% +x \Longrightarrow x= 50\%$ 
     \end{sol}
\end{ex}