\begin{ex}
(Enem)  A queima de cana aumenta a concentração de dióxido de carbono e de material particulado na atmosfera, causa alteração do clima e contribui para o aumento das doenças respiratórias. A tabela abaixo apresenta números relativos a pacientes internados em um hospital no período da queima da cana.
   \begin{center}
       \includegraphics[width=10cm]{imagens/ex_855.jpg}
   \end{center}
Escolhendo-se aleatoriamente um paciente internado nesse hospital por problemas respiratórios causados pelas queimadas, a probabilidade de que ele seja uma criança é igual a:
   \begin{enumerate}[(a)]
   \item 0,26, o que sugere a necessidade de implementação de medidas que reforcem a atenção ao idoso internado com problemas respiratórios.
   \item 0,50, o que comprova ser de grau médio a gravidade dos problemas respiratórios que atinge a população nas regiões das queimadas.
   \item 0,63, o que mostra que nenhum aspecto relativo à saúde infantil pode ser negligenciado
   \item 0,67, o que indica a necessidade de campanhas de conscientização que objetivem a eliminação das queimadas.
   \item 0,75, o que sugere a necessidade de que, em áreas atingidas pelos efeitos das queimadas, o atendimento hospitalar no setor de pediatria seja reforçado.
   \end{enumerate}
    \begin{sol}
     resposta: e \\
    $\frac{150}{200}=\frac{3}{4}=0,75$
    \end{sol}
\end{ex}