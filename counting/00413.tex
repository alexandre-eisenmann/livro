\begin{ex}
Quantos números naturais maiores que 4500 e de quatro algarismos distintos podemos representar com os algarismos 2, 3, 4, 5, 6 e 7
  \begin{sol}
    \phantom{A}
     $\frac{4}{\phantom{A}}\frac{5}{\phantom{A}}\frac{\phantom{A}}{4}\frac{\phantom{A}}{3}$\hspace{0,4cm}(+)\hspace{0,4cm}$\frac{4}{\phantom{A}}\frac{6}{\phantom{A}}\frac{\phantom{A}}{4}\frac{\phantom{A}}{3}$\hspace{0,4cm}(+)\hspace{0,4cm}$\frac{4}{\phantom{A}}\frac{7}{\phantom{A}}\frac{\phantom{A}}{4}\frac{\phantom{A}}{3}$\hspace{0,4cm}(+)\hspace{0,4cm}$\frac{\phantom{A}}{3}\frac{\phantom{A}}{5}\frac{\phantom{A}}{4}\frac{\phantom{A}}{3}\hspace{0,4cm} \Longrightarrow 12\cdot3+180=216$ 
  \end{sol}
\end{ex}