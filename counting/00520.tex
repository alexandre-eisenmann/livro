\begin{ex}
Todos os clientes de um banco dispõem de uma senha de acesso à conta bancária por meio do caixa eletrônico. Cada senha é uma sequência formada por 3 algarismos entre os 10 algarismos de 0 a 9, seguidos de 2 letras, escolhidas entre 26 letras do alfabeto. Se dois clientes não podem ter a mesma senha, pode-se afirmar que o maior número possível de clientes que esse banco pode ter é:
   \begin{enumerate}[(a)]
   \item 700.000
   \item 464.000
   \item 676.000
   \item 580.000
   \item 386.000
   \end{enumerate}
     \begin{sol}
     resposta:  c\\
     $10^3\cdot26^2=676.000$
     \end{sol}
\end{ex}