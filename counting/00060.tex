\begin{ex}
  (Ita) Numa certa brincadeira, um menino dispõe de uma caixa contendo quatro bolas, cada qual marcada com apenas uma destas letras: N, S, L e O. Ao retirar aleatoriamente uma bola, ele vê a letra correspondente e devolve a bola à caixa. Se essa letra for N, ele dá um passo na direção Norte; se S, em direção Sul, se L, na direção Leste e se O, na direção Oeste. Qual a probabilidade de ele voltar para a posição inicial no sexto passo?
    \begin{sol}
    \phantom{A} \\
    \hspace{0,5cm}
    $p= \frac{\text{ número de casos favoráveis}}{\text{número de casos possíveis}}$ \\
    \\
    I) 3 bolas N e 3 bolas S  $ \Rightarrow P^{3,3}_6=20$ \\
    II) 3 bolas L e 3 bolas O  $\Rightarrow P^{3,3}_6=20$ \\
    III) 2 bolas N, 2 bolas S,1 bola L e 1 O $\Rightarrow P^{2,2}_6=180$ \\
    IV) 1 bola N, 1 bola S, 2 bolas L e 2 bolas O  $\Rightarrow P^{2,2}_6=180$ \\
    nº de casos favoráveis: $20+20+180+180=400$ \hspace{0,4cm}
    nº de casos possíveis: $4^6=4096$\\
    $\Longrightarrow p= \frac{400}{4096}=\frac{25}{256}$ 
    \end{sol}
  
 \end{ex}