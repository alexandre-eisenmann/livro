\begin{ex}
  (Santa Casa) Em uma urna há 15 bolas, diferenciáveis apenas por suas cores, sendo 6 pretas, 5 brancas e 4 vermelhas, de modo que todas têm igual probabilidade de serem sorteadas. Uma pessoa vai até a urna, sorteia uma bola, não a mostra a ninguém e a mantém consigo. Em seguida, uma segunda pessoa vai até a urna e retira uma nova bola. A probabilidade de as duas bolas sorteadas terem a mesma cor é um valor :
    \begin{enumerate}   [(a)]
        \item entre 15\% e 25\%
        \item entre 25\% e 35\%
        \item entre 35\% e 45\%
        \item inferior a 15\%
        \item superior a 45\%
    \end{enumerate}
      \begin{sol}
       resposta: b \\
       preta e preta ou branca e branca ou vermelha e vermelha. \\
       $\frac{6}{15}\cdot\frac{5}{14}+\frac{5}{15}\cdot\frac{4}{14}+\frac{4}{15}\cdot\frac{3}{14}=\frac{62}{210}=\frac{31}{105}\approx29,5\%$
      \end{sol}
 \end{ex}