\begin{ex}
   (Enem) Numa escola com 1200 alunos foi realizada uma pesquisa sobre o conhecimento desses em duas línguas estrangeiras, inglês e espanhol. Nessa pesquisa constatou-se que 600 alunos falam inglês, 500 falam espanhol e 300 não falam qualquer um desses idiomas. Escolhendo-se um aluno dessa escola ao acaso e sabendo-se que ele não fala inglês qual a probabilidade de que esse aluno fale espanhol?
     \begin{enumerate} [(a)]
         \item $\frac{1}{2}$
         \item $\frac{5}{8}$
         \item $\frac{1}{4}$
         \item $\frac{5}{6}$
         \item $\frac{5}{14}$
     \end{enumerate}
      \begin{sol} 
      resposta: a \\
       $600+500+300=1400-1200=200$ \hspace{0,2cm}
       falam inglês e espanhol.\\ 
       300 falam  espanhol e outros 300 não falam nem inglês nem espanhol \\
         \begin{venndiagram2sets} [labelA=\(I\),labelB=\(E\),labelOnlyA=400,labelOnlyB=300,labelAB=200,labelNotAB=300]
         \end{venndiagram2sets}
         $ \Longrightarrow p=\frac{300+300}{1200}=\frac{600}{1200}=\frac{1}{2}$
      \end{sol}
  \end{ex}