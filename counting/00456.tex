\begin{ex}
(Ufscar) Em uma urna foram colocadas cem bolas, numeradas de 1 a 100. Para um sorteio aleatório de uma bola, o jogador A apostou no número 35, o jogador B no número 63 e o jogador C no número 72. A, B e C foram os únicos jogadores da partida. Depois de escolhidos os números apostados, o organizador do evento divulgou a seguinte regra:  ganhará o prêmio quem acertar o número sorteado e, não havendo acertador, ganhará aquele que mais se aproximar do número sorteado. Se houver empate entre dois jogadores, ganhará aquele que vencer uma partida de cara ou coroa realizada com uma moeda honesta. 
   \begin{enumerate}[(a)]
   \item qual é a probabilidade de que A seja o ganhador do prêmio?
   \item qual é a probabilidade de que B seja o ganhador do prêmio?
   \end{enumerate}
     \begin{sol}
      \phantom{A}
        \begin{enumerate}  [(a)]
            \item jogador A ganha se o nº sorteado for: 1 ou 2 ou 3 ou ....ou 48 e 50\% de chance se o nº for 49 $\Longrightarrow p = \frac{48}{100}+\frac{1}{2}\cdot \frac{1}{100}=$48,5\%
            \item jogador B ganha se o nº sorteado for: 50 ou 51 ou 52 ou.....ou 67 e 50\% de chance se o nº for 49 $\Longrightarrow p=\frac{18}{100}+\frac{1}{2}\cdot \frac{1}{100}= 18,5\%$
        \end{enumerate}
     \end{sol}
\end{ex}