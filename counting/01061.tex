\begin{ex}
 (Ufscar) Um computador registra em sua memória informações em código usando duas letras não repetidas, seguidas de 4 algarismos distintos. Duas dessas informações $x_1 x_2 a_1 a_2 a_3 a_4$ e $y_1 y_2 b_1 b_2 b_3 b_4$  são iguais se, e somente se $ x_i=y_i $ , $ i=1,2 $ e $ a_j=b_j $ ; $ j=1,2,3,4 $. Usando as letras A, B, C, D, E, F, G, H, I, J e os algarismos 1, 2, 3, 4, o número máximo de informações distintas registráveis será:
    \begin{enumerate}[(a)]
    \item 3200
    \item 5040
    \item 1080
    \item 2670
    \item 2160
    \end{enumerate}
      \begin{sol}
        resposta: e \\
        $\frac{\phantom{A}}{10}\frac{\phantom{A}}{9}\frac{\phantom{A}}{4}\frac{\phantom{A}}{3}\frac{\phantom{A}}{2}\frac{\phantom{A}}{1}= 90\cdot24=2160$
      \end{sol}
\end{ex}