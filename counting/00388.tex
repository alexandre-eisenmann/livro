\begin{ex}
(Unb) Em um trajeto urbano, existem 7 semáforos de cruzamento, cada um deles podendo estar vermelho (R) ; verde (V) ou amarelo (A). Denomina-se percurso a uma sequência de estados desses sinais com que um motorista se depararia ao percorrer o trajeto. Por exemplo: (R, V, A, A, R, V, R) é um percurso. Supondo que todos os percursos tenham a mesma probabilidade de ocorrência, julgue os seguintes itens:
   \begin{itemize}
   \item  [(1)] O número de possíveis percursos é 7!
   \item  [(2)] A probabilidade de ocorrer o percurso (R, V, A, A, R, V, R ) é :   $(\frac{1}{3})^3 + (\frac{1}{3})^2 + (\frac{1}{3})^2$
   \item  [(3)] A probabilidade de que o primeiro semáforo esteja verde é $\frac{1}{3}$
   \item  [(4)] A probabilidade de que à exceção do primeiro, todos os demais semáforos estejam vermelhos é inferior a 0,0009
   \item  [(5)] A probabilidade de que apenas um semáforo esteja vermelho é inferior a 0,2
   \end{itemize}
     \begin{sol}
       \phantom{A} 
       \begin{itemize}
           \item [(1)] $3^7 \rightarrow$ F 
           \item [(2)] $(\frac{1}{3})^7 \rightarrow$ F
           \item [(3)] V
           \item [(4)] $\frac{2}{3}\cdot(\frac{1}{3})^6=0,0009144>0,0009 \rightarrow$  F
           \item [(5)] $\frac{1}{3}\cdot(\frac{2}{3})^6\cdot \mathrm{C}_{6,1}=\frac{448}{2187}=0,2 \rightarrow$ F
       \end{itemize}
     \end{sol}
\end{ex}