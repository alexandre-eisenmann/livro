\begin{ex}
Uma urna contém precisamente 10 bolas, sendo : 3 verdes, 2 pretas e 5 azuis. Retirando 3 bolas da urna, uma de cada vez e com reposição, calcule a probabilidade de saírem:
   \begin{enumerate}[(a)]
   \item a primeira bola verde, a segunda preta e a terceira azul.
   \item 3 bolas de cores diferentes.
   \item 3 bolas azuis.
   \end{enumerate}
    \begin{sol}
       \phantom{A}
       \begin{enumerate} [(a)]
           \item $\frac{3}{10}\cdot\frac{2}{10}\cdot\frac{5}{10}=\frac{3}{100}$
           \item $\frac{3}{10}\cdot\frac{2}{10}\cdot\frac{5}{10}\cdot3!=\frac{9}{50}$
           \item $\frac{5}{10}\cdot\frac{5}{10}\cdot\frac{5}{10}=\frac{1}{8}$
       \end{enumerate}
    \end{sol}
\end{ex}