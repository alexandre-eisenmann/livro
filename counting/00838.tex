\begin{ex}
 (Fatec) Considere todos os números de cinco algarismos distintos obtidos pela permutação dos algarismos 4, 5, 6, 7 e 8. Escolhendo-se um desses números, ao acaso, a probabilidade dele ser um número ímpar é:
    \begin{enumerate}[(a)]
    \item 1
    \item $\frac{1}{2}$
    \item $\frac{2}{5}$
    \item $\frac{1}{4}$
    \item $\frac{1}{5}$
    \end{enumerate}
     \begin{sol}
       resposta: c \\
       espaço amostral = 5! \\
       $\frac{\phantom{A}}{4}\frac{\phantom{A}}{3}\frac{\phantom{A}}{2}\frac{\phantom{A}}{1}\frac{\phantom{A}}{2}=2\cdot4!
       \Longrightarrow p =\frac{2\cdot4!}{5!}=\frac{2}{5}$
     \end{sol}
\end{ex}