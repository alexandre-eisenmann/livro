\begin{ex}
	(Mack) Os números dos telefones de uma cidade são constituídos de 6 dígitos. Sabendo que o primeiro dígito nunca pode ser zero, se os números de telefones passarem a ser de 7 dígitos, o aumento possível na quantidade de telefones será:
    \begin{enumerate}[(a)]
    \item 81 x $10^3$
    \item 90 x $10^3$
    \item 81 x $10^4$
    \item 81 x $10^5$
    \item 90 x $10^5$
    \end{enumerate}
      \begin{sol}
        resposta: d \\
    $9\cdot{10}^6-9\cdot{10}^5=81\cdot10^5$
      \end{sol}
\end{ex}