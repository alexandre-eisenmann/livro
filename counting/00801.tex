\begin{ex}
 (UFMG) Um aposentado realiza diariamente, de segunda à sexta-feira, estas atividades:
    \begin{itemize}
    \item [--] leva seu neto Pedrinho às 13h para a escola,
    \item [--] pedala 20 minutos na bicicleta ergométrica, 
    \item [--] passeia com o cachorro da família,
    \item [--] pega seu neto Pedrinho, às 17 h, na escola,
    \item[--] rega as plantas do jardim de sua casa .
    \end{itemize}
Cansado, porém de fazer essas atividades sempre na mesma ordem, ele resolveu que a cada dia, vai realizá-las em uma ordem diferente.
Nesse caso, o número de maneiras possíveis de ele realizar essas 5 atividades, em ordem diferente, é:
    \begin{enumerate}[(a)]
    \item 60
    \item 48
    \item 120
    \item 72
    \item 24
    \end{enumerate}
      \begin{sol}
        resposta: a \\
        $5\cdot4\cdot3\cdot2\cdot1=120$  \\
        como as atividades (a) e (d) não podem mudar, a resposta é: $120\div 2 = 60$
      \end{sol}
\end{ex}