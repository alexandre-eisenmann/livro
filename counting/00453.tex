\begin{ex}
(Ufscar) Um espaço amostral é um conjunto cujos elementos representam todos os resultados possíveis de um experimento. Chamamos de evento ao conjunto de resultados do experimento correspondente a algum subconjunto de um espaço amostral.
   \begin{enumerate}[(a)]
   \item descreva o espaço amostral correspondente ao lançamento simultâneo de um dado e de uma moeda.
   \item determine a probabilidade que no experimento descrito ocorram os eventos:
   Evento A: resulte cara na moeda e um número par no dado.
   Evento B: resulte 1 ou 5 no dado.
   \end{enumerate}
     \begin{sol}
      \phantom{A}
        \begin{enumerate} [(a)]
            \item espaço amostral= (1,k) (1,c) (2,k) (2,c) (3,k) (3,c) (4,k) (4,c) (5,k) (5,c) (6,k) (6,c)
            \item $p(A)=\frac{3}{12}=\frac{1}{4}$\hspace{0,6cm} $p(B)=\frac{4}{12}=\frac{1}{3}$
        \end{enumerate}
     \end{sol}
\end{ex}