\begin{ex}
  (Enem) José, Paulo e Antônio estão jogando dados não viciados, nos quais, em cada uma das seis faces, há um número de 1 a 6. Cada um deles jogará dois dados simultaneamente. José acredita que, após jogar seus dados, os números das faces voltadas para cima lhe darão uma soma igual a 7. Já Paulo acredita que sua soma será igual a 4 e Antônio acredita que sua soma será igual a 8. Com essa escolha, quem tem a maior probabilidade de acertar sua respectiva soma é:
    \begin{enumerate} [(a)]
        \item Antônio, já que sua soma é a maior de todas as escolhidas.
        \item José e Antônio, já que há 6 possibilidades tanto para a escolha de José quanto para a escolha de Antônio, e há apenas 4 possibilidades para a escolha de Paulo.
        \item José e Antônio, já que há 3 possibilidades tanto para a escolha de José quanto para a escolha de Antônio, e há apenas 2 possibilidades para a escolha de Paulo.
        \item José, já que há 6 possibilidades para formar sua soma, 5 possibilidades para formar a soma de Antônio e apenas 3 possibilidades para formar a soma de Paulo.
        \item Paulo, já que sua soma é a menor de todas.
    \end{enumerate}
      \begin{sol}
       resposta: d \\
       - 6 possibilidades de soma 7: (1,6) (2,5) (3,4) (4,3) (5,2) (6,1) \\
       - 3 possibilidades de soma 4: (1,3) (2,2) (3,1) \\
       - 5 possibilidades de soma 8:
       (2,6) (3,5) (4,4) (5,3) (6,2) \\
       José tem soma 7; há 5 possibilidades para formar a soma de Antônio e 3 possibilidades para a soma de Paulo.
       \end{sol}
 \end{ex}