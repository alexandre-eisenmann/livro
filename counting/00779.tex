\begin{ex}
 Com os algarismos de 1 a 9 vamos formar números inteiros de três algarismos. Quantos são ímpares e maiores que 800: \\
   \begin{enumerate} [(a)]
       \item considerando os três algarismos distintos;
       \item podendo repetir ( ou não ) os algarismos.
   \end{enumerate}
     \begin{sol}
      \phantom{A}
       \begin{enumerate} [(a)]
           \item $\frac{8}{1}\frac{\phantom{A}}{7}\frac{\mathrm{I}}{5}+\frac{9}{1}\frac{\phantom{A}}{7}\frac{\phantom{A}}{4}=35+28=63$
           \item $\frac{8}{1}\frac{8}{1}\frac{\mathrm{I}}{5}+\frac{8}{1}\frac{\mathrm{I}}{5}\frac{\mathrm{I}}{5}+\frac{9}{1}\frac{9}{1}\frac{\mathrm{I}}{5}+\frac{9}{1}\frac{\mathrm{I}}{5}\frac{\mathrm{I}}{5}=5+25+5+25+63=123$
       \end{enumerate}
     
     \end{sol}

\end{ex}