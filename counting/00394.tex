\begin{ex}
No cadastro de um cursinho, estão registrados 600 alunos, sendo 380 homens. Sabendo que 105 mulheres já se formaram no ensino médio e 200 homens não se formaram no ensino médio, pergunta-se:
   \begin{enumerate}[(a)]
   \item quantas mulheres não se formaram no ensino médio?
   \item qual é a probabilidade de sortear um homem formado?
   \item sabendo que é mulher, qual é a probabilidade de ser uma  formada?
   \item qual é a probabilidade de sortear uma mulher ou alguém que não se formou?
   \end{enumerate}
     \begin{sol}
       \phantom{A} \\ \\
       \begin{tabular}{|c|c|c|c|} \hline
            & homens & mulheres &total \\  \hline 
         formados   & 180 & 105 & 285 \\ \hline
         não formados & 200 & 115 & 315 \\ \hline
         total & 380 & 220 & 600 \\ \hline
       \end{tabular}
         \begin{enumerate} [(a)]
             \item 115
             \item $\frac{180}{600}=\frac{3}{10}$
             \item $\frac{105}{220}=\frac{21}{44}$
             \item $\frac{380}{600}+\frac{315}{600}-\frac{200}{600}=\frac{495}{600}=\frac{33}{40}$
         \end{enumerate}
     \end{sol}
\end{ex}