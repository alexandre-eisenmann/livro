\begin{ex}
(Unicamp) Um casal convidou 6 amigos para assistirem uma peça teatral. Chegando ao teatro, descobriram que, em cada fila da sala, as poltronas eram enumeradas em ordem crescente. Assim, por exemplo, a poltrona 1 de uma fila era sucedida pela poltrona 2 da mesma fila, que por sua vez, era sucedida pela poltrona 3, e assim por diante. Suponha que as 8 pessoas receberam ingressos com numeração consecutiva de uma mesma fila e que os ingressos foram distribuídos entre elas de forma aleatória. Qual a probabilidade de o casal ter recebido ingressos de poltronas vizinhas?
  \begin{sol}
    \phantom{A} \\
    total de jeitos dos 8 sentarem = 8! \\
    total de jeitos em que o casal ocupam poltronas vizinhas = $7!\cdot2$  \\
    $\Longrightarrow p =\frac{7!\cdot2}{8!}=\frac{1}{4}$
    
    
  \end{sol}
\end{ex}