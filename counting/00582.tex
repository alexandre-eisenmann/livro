\begin{ex}
Uma pessoa tem no bolso exatamente duas moedas de R\$ 1,00, 4 moedas de R\$ 0,50 e 3 moedas de R\$ 0,10. Essa pessoa retira, simultaneamente, 3 moedas do bolso. Calcule a probabilidade de:
   \begin{enumerate}[(a)]
   \item as moedas retiradas terem valores diferentes entre si.
   \item saírem duas moedas de R\$ 0,50 e uma de R\$ 0,10.
   \item as moedas retiradas totalizarem R\$ 1,20.
   \end{enumerate}
     \begin{sol}
      \phantom{A}
        \begin{enumerate} [(a)]
            \item $\frac{2}{9}\cdot\frac{4}{8}\cdot\frac{3}{7}\cdot3!=\frac{2}{7}$
            \item $\frac{4}{9}\cdot\frac{3}{8}\cdot\frac{3}{7}\cdot3=\frac{3}{14}$
            \item $\frac{2}{9}\cdot\frac{3}{8}\cdot\frac{2}{7}\cdot3=\frac{1}{14}$
            \end{enumerate}
     \end{sol}
\end{ex}