\begin{ex}
(FGV) Numa grande cidade a probabilidade de que um carro de certo modelo seja roubado, no período de um ano é $\frac{1}{20}$. Se considerarmos uma amostra aleatória de 10 destes carros;
   \begin{enumerate}[(a)]
   \item qual a probabilidade de que nenhum seja roubado no período de um ano?
   \item  qual a probabilidade de que exatamente um carro seja roubado no período de um ano?
   \end{enumerate}
Nota: admitir independência entre os eventos associados aos roubos de cada carro.
 \begin{sol}
   \phantom{A}
     \begin{enumerate} [(a)]
         \item $\mathrm{C}_{{10},0}\cdot(\frac{1}{20})^0\cdot(\frac{19}{20})^{10}=59,8\%$
         \item $\mathrm{C}_{{10},1}\cdot(\frac{1}{20})^1\cdot(\frac{19}{20})^9=31,5\%$
     \end{enumerate}
 \end{sol}
\end{ex}