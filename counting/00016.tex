\begin{ex}
  (Enem) Um morador de uma região metropolitana tem 50\% de probabilidade de atrasar-se para o trabalho quando chove na região; caso não chova, sua probabilidade de atraso é de 25\%. Para um determinado dia, o serviço de meteorologia estima em 30\% a probabilidade da ocorrência de chuva nessa região. Qual é a probabilidade de esse morador se atrasar para o serviço no dia para o qual foi dada a estimativa de chuva?
    \begin{enumerate} [(a)]
        \item 0,075
        \item 0,150
        \item 0,325
        \item 0,600
        \item 0,800
    \end{enumerate}
    \begin{sol}
     resposta: c \\
     chover e atrasar \textbf{ou} não chover e atrasar 
     $\Longrightarrow30\%\cdot50\%+70\%\cdot25\%=32,5\%=0,325$
    \end{sol}
 \end{ex}