\begin{ex}
 (FGV) Em uma gaveta de armário de um quarto escuro há 6 camisetas vermelhas, 10 brancas e 7 pretas. Qual é o número mínimo de camisetas que se deve retirar da gaveta, sem que se vejam as cores, para que:
    \begin{enumerate}[(a)]
    \item se tenha certeza de ter retirado duas camisetas de cores diferentes?
    \item se tenha certeza de ter retirado duas camisetas de mesma cor?
    \item se tenha certeza de ter retirado pelo menos uma camiseta de cada cor?
    \end{enumerate} 
     \begin{sol}
        \phantom{A}  \\
        \begin{enumerate} [(a)]
            \item 11 porque as 10 primeiras podem ser brancas.
            \item 4 porque as 3 primeiras podem ter cores diferentes.
            \item 18 porque se as 10 primeiras forem brancas e as 7 seguintes forem pretas, a 18ª é vermelha.
        \end{enumerate}
     \end{sol}
\end{ex}