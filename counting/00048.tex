\begin{ex}
  (Enem) Rafael mora no Centro de uma cidade e decidiu se mudar, por recomendações médicas, para uma das regiões: Rural, Comercial, Residencial Urbano ou Residencial Suburbano. A principal recomendação médica foi com as temperaturas das “ilhas de calor” da região, que deveriam ser inferiores a 31ºC. Tais temperaturas são apresentadas no gráfico: 
     \begin{center}
         \includegraphics[width=8cm]{imagens/enem_ex_1048.png}
     \end{center}
  Escolhendo, aleatoriamente, uma das outras regiões para morar, a probabilidade de ele escolher uma região que seja adequada às recomendações médicas é:
    \begin{enumerate}  [(a)]
        \item $\frac{1}{5}$
        \item $\frac{1}{4}$
        \item $\frac{2}{5}$
        \item $\frac{3}{5}$
        \item $\frac{3}{4}$
    \end{enumerate}
      \begin{sol}
      resposta: e \\
      Observando o gráfico nota-se que as regiões que estão abaixo de $31^{\circ}$C são: Rural, Residencial Urbana e Residencial Suburbana, isto é 3 entre 4 $\Longrightarrow p=\frac{3}{4}$
      \end{sol}
 \end{ex}