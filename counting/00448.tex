\begin{ex}
(Fgv) Uma moeda é viciada de tal forma que os resultados possíveis, cara e coroa são tais, que a probabilidade de sair cara num lançamento é o triplo de sair coroa.
   \begin{enumerate}[(a)]
   \item lançando-se uma vez a moeda qual é a probabilidade de sair cara?
   \item  lançando-se três vezes a moeda, qual a probabilidade de sair exatamente uma cara?
   \end{enumerate}
     \begin{sol}
      \phantom{A}  \\
      cara = \textit{c}, coroa = \textit{k} ; \textit{c=3k}
       \begin{enumerate} [(a)]
       \item $3k+k=1 \rightarrow k = \frac{1}{4} \therefore c=\frac{3}{4}$
       \item \textit{ckk} \textbf{ou} \textit{kck} \textbf{ou} \textit{kkc} $\Longrightarrow \frac{3}{4}\cdot\frac{1}{4}\cdot\frac{1}{4}+\frac{1}{4}\cdot\frac{3}{4}\cdot\frac{1}{4}+\frac{1}{4}\cdot\frac{1}{4}\cdot\frac{3}{4}=\frac{9}{64}$
       \end{enumerate}
      
     \end{sol}
\end{ex}