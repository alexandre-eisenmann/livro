\begin{ex}
 (ITA) Um general possui \textit{n} soldados para tomar uma posição inimiga. Desejando efetuar um ataque com 2 grupos, um frontal com \textit{r} soldados e outro de retaguarda com \textit{s}  soldados ( \textit{r}+\textit{s}=\textit{n}), ele poderá dispor seus homens de:
    \begin{enumerate}[(a)]
    \item $\frac{n!}{(r+s)!}$ maneiras distintas neste ataque.
    \item $\frac{n!}{r!s!}$ maneiras distintas neste ataque.
    \item $\frac{n!}{(rs)!}$ maneiras distintas neste ataque.
    \item $\frac{2n!}{(r+s)!}$ maneiras distintas neste ataque.
    \item $\frac{2n!}{r!s!}$ maneiras distintas neste ataque.
    \end{enumerate}
      \begin{sol}
       resposta: b \\
       $\mathrm{C}_{n,r}=\frac{n!}{r!\cdot(n-r)!}=\frac{n!}{r!\cdot s!}$
      \end{sol}
\end{ex}