\begin{ex}
 	(Ufmg) Vinte alunos de uma escola, entre os quais Gabriel, Mateus e Roger, formam uma fila aleatoriamente.
    \begin{enumerate}[(a)]
    \item determine a probabilidade de essa fila ser formada de tal modo que Gabriel, Mateus e Roger apareçam juntos em qualquer ordem.
    \item determine a probabilidade de essa fila ser formada de tal modo que, entre Gabriel e Mateus, haja exatamente, 5 outros alunos.
    \end{enumerate}
      \begin{sol}
        \phantom{A} 
          \begin{enumerate} [(a)]
              \item $\frac{18!\cdot3!}{20!}=\frac{3}{190}$
              \item 5 alunos entre Gabriel e Mateus; depois os outros 13 alunos.\\
              $\frac{\mathrm{A}_{{18},5}\cdot14!\cdot2}{20!}=\frac{7}{95}$
          \end{enumerate}
      \end{sol}
\end{ex}