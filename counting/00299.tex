\begin{ex}
(Fuvest) São efetuados lançamentos sucessivos e independentes de uma moeda perfeita \\ (a probabilidade de cara e de coroa são iguais) até que apareça cara pela segunda vez.
   \begin{enumerate}[(a)]
   \item qual é a probabilidade de que a segunda cara apareça no oitavo lançamento?
   \item sabendo-se que a segunda cara apareceu no oitavo lançamento, qual é a probabilidade condicional de que a primeira cara tenha aparecido no terceiro?
   \end{enumerate}
    \begin{sol}
      \phantom{A}  \\
           (k= cara c= coroa)\\
           7 lançamentos: (kcccccck) (ckccccck) (cckcccck) (ccckccck) (cccckcck) (ccccckck) (cccccckk).
          A  probabilidade de cada  lançamento é: $(\frac{1}{2})^8$
     \begin{enumerate} [(a)]
           \item 
          $7\cdot(\frac{1}{2})^8=\frac{7}{256}$
          \item 
          $\frac{(\frac{1}{2})^8}{\frac{7}{256}}=\frac{1}{7}$
     \end{enumerate}
      
    \end{sol}
\end{ex}