\begin{ex}
  (ITA) Uma amostra de estrangeiros, em que 18\% são proficientes em inglês, realizou um exame para classificar a sua proficiência nesta língua. Dos estrangeiros que são proficientes em inglês, 75\% foram classificados como proficientes. Entre os não proficientes em inglês, 7\% foram classificados como proficientes. Um estrangeiro desta amostra, escolhido ao acaso, foi classificado como proficiente em inglês. A probabilidade deste estrangeiro ser efetivamente proficiente nesta língua é de aproximadamente:
    \begin{enumerate}[(a)]
    \item 73\%
    \item 70\%
    \item 68\%
    \item 65\%
    \item 64\%
    \end{enumerate}
      \begin{sol}
        \phantom{A} \\
        18\% estrangeiros proficientes em inglês;  75\% dos 18\% = 13,5\%  classificados como proficientes.\\
        82\% estrangeiros não proficientes em inglês;  7\% de 82\% = 5,74\%  classificados como proficientes em inglês\\
        $\Longrightarrow p = \frac{13,5}{13,5+5,74}\simeq0,7 = 70\%$
      \end{sol}
\end{ex}