\begin{ex}
(Fatec) Suponha que na região em que ocorreu  a passagem do furacão Katrina, somente ocorrem 3 grandes fenômenos destrutivos da natureza, dois a dois mutuamente exclusivos:
   \begin{itemize}
   \item [--]  os hidromiteorológicos (A)
   \item [--]  os geofísicos (B)
   \item [--]  os biológicos (C)
   \end{itemize}
Se a probabilidade de ocorrer A é cinco vezes a de ocorrer B e esta corresponde a 50\% da probabilidade de ocorrência de C, então a probabilidade de ocorrer:
   \begin{enumerate}[(a)]
   \item  A é igual a duas vezes a de ocorrer C.
   \item  C é igual à metade de ocorrer B.  
   \item  B ou C é igual a 42,5\%.
   \item  A ou B é igual a 75\%.
   \item  A ou C é igual a 92,5\%.
   \end{enumerate}
     \begin{sol}
       resposta: d \\
       $2,5\cdot p(\mathrm{C})+0,5\cdot p(\mathrm{C})+p(\mathrm{C})=1\rightarrow p(\mathrm{C})=\frac{1}{4};\hspace{0,3cm} p(\mathrm{B})=\frac{1}{8};\hspace{0,3cm} p(\mathrm{A})=\frac{5}{8}$\\
       $\Longrightarrow p(\mathrm{A})+p(\mathrm{B})=\frac{6}{8}=75\%$
     \end{sol}
\end{ex}