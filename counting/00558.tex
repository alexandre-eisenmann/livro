\begin{ex}
(Enem) Um médico está estudando um novo medicamento que combate um tipo de  câncer em estágios avançados. Porém, devido ao forte efeito dos seus componentes, a cada dose administrada há uma chance de 10\% de que o paciente sofra algum dos efeitos colaterais observados no estudo, tais como dores de cabeça, vômitos ou mesmo agravamento dos sintomas da doença. O médico oferece tratamentos compostos por 3, 4, 6, 8 ou 10 doses do medicamento, de acordo com o risco que o paciente pretende assumir.  Se um paciente considera aceitável um risco de até 35\% de chances que ocorra algum dos efeitos colaterais durante o tratamento, qual é o maior número admissível de doses para esse paciente?
   \begin{enumerate}[(a)]
   \item 3 doses
   \item 4 doses 
   \item 6 doses
   \item 8 doses
   \item 10 doses
   \end{enumerate}
     \begin{sol}
      resposta: b \\
      chance de não acontecer: 90\% a cada dose\hspace{0,6cm} risco de até 35\% \\
      probabilidade de não ter sintomas:\\
      - tomando 1 dose: 90\% \\
      - tomando 2 doses: 81\% \\
      - tomando 3 doses: $(0,9)^3=72,9\% \rightarrow 100-72,9=27,1\%$ \\
      - tomando 4 doses: $(0,9)^4=65,61\% \rightarrow 100 - 65,1=34,39\%$ \\
        logo, o número admissível de doses é 4
     \end{sol}
\end{ex}