\begin{ex}
(Ufrj) O setor de controle de qualidade de uma pequena confecção fez um levantamento das peças produzidas classificando-as como aproveitáveis ou não aproveitáveis. As porcentagens de peças aproveitáveis estão na tabela abaixo.
\begin{center}
\begin{tabular}{|c|c|}  \hline
\textbf{Peça}  & \textbf{Aproveitável}  \\  \hline
Camiseta  &  96\%  \\ \hline
Bermuda   &  98\%  \\  \hline
Calça     &  90\%   \\  \hline
\end{tabular}
\end{center}
Um segundo levantamento verificou que 75\% das camisetas aproveitáveis, 90\% das bermudas aproveitáveis e 85\% das calças aproveitáveis são de primeira qualidade.	Escolhendo-se aleatoriamente uma calça e uma camiseta dessa confecção calcule a probabilidade \textit{p} de as condições a seguir serem ambas satisfeitas: a camiseta ser de primeira qualidade e a calça não aproveitável.
 \begin{sol}
  \phantom{A} \\
   $75\%\cdot96\%\cdot10\%= 0,072=7,2\%$
 \end{sol}
\end{ex}