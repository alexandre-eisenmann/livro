\begin{ex}
(Enem) O diretor de uma escola convidou os 280 alunos de terceiro ano a participarem de uma brincadeira. Suponha que existem 5 objetos e 6 personagens numa casa de 9 cômodos; um dos personagens esconde um dos objetos em um dos cômodos da casa. O objetivo da brincadeira é adivinhar qual objeto foi escondido por qual personagem e em qual cômodo da casa o objeto foi escondido. Todos os alunos decidiram participar. A cada vez um aluno é sorteado e dá a sua resposta. As respostas devem ser sempre distintas das anteriores, e um mesmo aluno não pode ser sorteado mais de uma vez. Se a resposta do aluno estiver correta, ele é declarado vencedor e a brincadeira é encerrada. O diretor sabe que algum aluno acertará a resposta porque há:
   \begin{enumerate}[(a)]
   \item 10 alunos a mais do que possíveis respostas distintas.
   \item 20 alunos a mais do que possíveis respostas distintas.
   \item 119 alunos a mais do que possíveis respostas distintas. 
   \item 260 alunos a mais do que possíveis respostas distintas. 
   \item 270 alunos a mais do que possíveis respostas distintas. 
   \end{enumerate}
     \begin{sol}
      resposta: a \\
      possibilidades= $5\cdot6\cdot9= 270 \hspace{0,4cm} \Longrightarrow 280 - 270=10$
     \end{sol}
\end{ex}