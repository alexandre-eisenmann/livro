\begin{ex}
 Com os algarismos 1, 2, 3, 4, 5 e 6 formam-se números naturais de seis algarismos distintos. Sorteando-se um deles, a probabilidade  que nele não apareçam juntos dois algarismos pares nem dois algarismos ímpares, é:
    \begin{enumerate}[(a)]
    \item $\frac{1}{20}$
    \item $\frac{1}{15}$
    \item $\frac{1}{12}$
    \item $\frac{1}{10}$
    \item $\frac{1}{8}$
    \end{enumerate}
      \begin{sol}
       resposta: d \\
       P I P I P I ou I P I P I P $\Longrightarrow \frac{3!\cdot3!\cdot2}{6!}=\frac{1}{10}$
       
      \end{sol}
\end{ex}