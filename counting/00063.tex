\begin{ex}
(Insper) Um antigo game show da televisão brasileira consistia em um apresentador fazer perguntas para um participante indicar, entre 4 alternativas, a resposta correta. Ao longo do programa, quando o participante não sabia qual era a resposta correta, ele podia recorrer a um tipo de auxílio, chamado “ajuda das cartas”, no qual ele escolhia aleatoriamente uma entre quatro cartas, podendo ser beneficiado com a exclusão de 0, 1, 2 ou 3 alternativas erradas. Suponha que um participante decida responder uma pergunta em que, para ele, todas as alternativas são igualmente prováveis de ser a correta. Se ele optar pela “ajuda das cartas”, a probabilidade de ele escolher a alternativa correta será:
 \begin{enumerate} [(a)]
     \item entre 40\% e 45\%.
     \item superior a 50\%.
     \item inferior a 35\%.
     \item entre 35\% e 40\%.
     \item entre 45\% e 50\%.
 \end{enumerate}
   \begin{sol}
   resposta: b \\
   Probabilidade de sortear cada uma das cartas: $\frac{1}{4}$.  \\
   Tendo sorteado a carta, a probabilidade de escolher a alternativa completa: zero é $\frac{1}{4}$; um é $\frac{1}{3}$; dois é $\frac{1}{2}$ e três é $1$ \\
   $\Longrightarrow p=\frac{1}{4}\cdot(\frac{1}{4}+\frac{1}{3}+\frac{1}{2}+1)=\frac{25}{48}\approx 0,52=52\%$
   \end{sol}
\end{ex}