\begin{ex}
(Fuvest) Um dado cúbico, não viciado, com faces numeradas de 1 a 6 é lançado 3 vezes. Em cada lançamento, anota-se o número obtido na face superior do dado, formando-se uma sequência (a, b, c). Qual é a probabilidade de que b seja sucessor de a ou que c seja sucessor de b?
   \begin{enumerate}[(a)]
   \item $\frac{4}{27}$
   \item $\frac{11}{54}$
   \item $\frac{7}{27}$
   \item $\frac{10}{27}$
   \item $\frac{23}{54}$
   \end{enumerate}
     \begin{sol}
      resposta: c \\
      $\frac{1}{\phantom{A}}\frac{2}{\phantom{A}}\frac{\phantom{A}}{6}\hspace{0,4cm}\frac{2}{\phantom{A}}\frac{3}{\phantom{A}}\frac{\phantom{A}}{6}\hspace{0,4cm}\frac{3}{\phantom{A}}\frac{4}{\phantom{A}}\frac{\phantom{A}}{6}\hspace{0,4cm}\frac{4}{\phantom{A}}\frac{5}{\phantom{A}}\frac{\phantom{A}}{6}\hspace{0,4cm}\frac{5}{\phantom{A}}\frac{6}{\phantom{A}}\frac{\phantom{A}}{6}
      \hspace{0,4cm}\frac{\phantom{A}}{6}\frac{1}{\phantom{A}}\frac{2}{\phantom{A}} \hspace{0,4cm}\frac{\phantom{A}}{6}\frac{2}{\phantom{A}}\frac{3}{\phantom{A}} \hspace{0,4cm}\frac{\phantom{A}}{6}\frac{3}{\phantom{A}}\frac{4}{\phantom{A}} \hspace{0,4cm}\frac{\phantom{A}}{6}\frac{4}{\phantom{A}}\frac{5}{\phantom{A}} \hspace{0,4cm}\frac{\phantom{A}}{6}\frac{5}{\phantom{A}}\frac{6}{\phantom{A}}$\\
      4 repetidos: 123, 234, 345, 456 . Total: 30 + 30 - 4 = 56 \\
      $p=\frac{56}{6\cdot6\cdot6}=\frac{56}{216}=\frac{7}{27}$
     \end{sol}
\end{ex}