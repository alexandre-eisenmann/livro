\begin{ex}
(Puc –SP) Uma moeda é viciada de modo que a probabilidade de ocorrer cara numa jogada é 30\% a mais do que a de ocorrer coroa. Se essa moeda é jogada duas vezes consecutivamente, a probabilidade de ocorrência de cara nas duas jogadas é:
   \begin{enumerate}[(a)]
   \item 49\%
   \item 42,25\%
   \item 64\%
   \item 64,25\%
   \item 15\%
   \end{enumerate}
    \begin{sol}
     resposta: b \\
     $x+x+\frac{30}{100}=1\hspace{0,5cm} x = 0,35$\hspace{0,5cm} logo:  coroa = 0,35 ; cara = 0,65 \\
      $\Longrightarrow 0,65\cdot0,65= 42,25\%$
    \end{sol}
\end{ex}