\begin{ex}
(UF – RJ) Os cavalos X, Y, Z disputam uma prova ao final da qual não poderá ocorrer empate. Sabe-se que a probabilidade de X vencer é igual ao dobro da probabilidade de Y vencer. Da mesma forma, a probabilidade de Y vencer é igual ao dobro da probabilidade de Z vencer. Calcule a probabilidade de:
   \begin{enumerate}[(a)]
   \item X vencer;
   \item Y vencer;
   \item Z vencer.
   \end{enumerate}
     \begin{sol}
       \phantom{A} \\
       Z = \textit{n}; Y = 2\textit{n}; X = 4\textit{n} $\Rightarrow 4n+2n+n=1 \therefore n=\frac{1}{7}$
       \begin{enumerate} [(a)]
           \item $\frac{4}{7}$
           \item $\dfrac{2}{7}$
           \item $\frac{1}{7}$
       \end{enumerate}
     \end{sol}
\end{ex}