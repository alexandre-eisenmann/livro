\begin{ex}
(Uenf-RJ) Uma pesquisa realizada em um hospital indicou que a probabilidade de um paciente morrer no prazo de um mês, após determinada operação de câncer, é igual a 20\%.
Se três pacientes são submetidos a essa operação, calcule a probabilidadede, nesse prazo:
   \begin{enumerate}[(a)]
   \item todos sobreviverem.
   \item apenas dois sobreviverem.
   \end{enumerate}
    \begin{sol}
     \phantom{A}
       \begin{enumerate} [(a)]
           \item $(0,80)^3=0,512=51,2\%$
           \item $(0,80)^2\cdot0,20\cdot \mathrm{C}_{3,2}=0,384=38,4\%$
       \end{enumerate}
    \end{sol}
\end{ex}