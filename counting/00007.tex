\begin{ex}
  (Fuvest) Um jogo educativo possui 16 peças nos formatos: círculo, triângulo, quadrado e estrela, e cada formato é apresentado em 4 cores: amarelo, branco, laranja e verde. Dois jogadores distribuem entre si quantidades iguais dessas peças, de forma aleatória. O conjunto de 8 peças que cada jogador recebe é chamado de coleção.
      \begin{enumerate} [(a)]
          \item quantas são as possíveis coleções que um jogador pode receber?
          \item qual  é a probabilidade de que os dois jogadores recebam a mesma quantidade de peças amarela?
          \item a regra do jogo estabelece pontuações para as peças, da seguinte forma: círculo = 1 ponto, triângulo = 2 pontos, quadrado = 3 pontos e estrela = 4 pontos. Quantas são as possíveis coleções que valem 26 pontos ou mais ?
      \end{enumerate}
        \begin{sol}
         \phantom{A}
          \begin{enumerate} [(a)]
              \item $\mathrm{C}{{16},8}=12870
              $
              \item $\frac{\mathrm{C}_{4,2}\cdot\mathrm{C}_{{12},6}}{\mathrm{C}_{{16},8}}=\frac{28}{65}$
              \item coleções possíves:\\
              - 4 estrelas e 4 quadrados (28 pontos) $\Longrightarrow \mathrm{C}_{4,4}\cdot\mathrm{C}_{4,4}=1$ \\
              - 4 estrelas, 3 quadrados e 1 triângulo (27 pontos) $\Longrightarrow \mathrm{C}_{4,4}\cdot\mathrm{C}_{4,3}\cdot\mathrm{C}_{4,1}=16$\\
              - 4 estrelas, 3 quadrados e 1 círculo (26 pontos) $\Longrightarrow \mathrm{C}_{4,4}\cdot\mathrm{C}_{4,3}\cdot\mathrm{C}_{4,1}=16$ \\
              - 4 estrelas, 2 quadrados e 2 triângulos  (26 pontos) $\Longrightarrow \mathrm{C}_{4,4}\cdot\mathrm{C}_{4,2}\cdot\mathrm{C}_{4,2}=36$\\
              - 3 estrelas, 4 quadrados e 1 triângulo
              (26 pontos)$\Longrightarrow \mathrm{C}_{4,3}\cdot\mathrm{C}_{4,4}\cdot\mathrm{C}_{4,1}=16$\\
              $\therefore \hspace{0,5cm} 1+16+16+36+16=85$
          \end{enumerate}
        \end{sol}
  \end{ex}