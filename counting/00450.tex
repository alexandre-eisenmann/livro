\begin{ex}
 (Uerj) Uma pesquisa realizada em um hospital indicou que a probabilidade de um paciente morrer num prazo de um mês, após determinada operação de câncer, é igual a 20\%. Se três pacientes são submetidos a essa operação, calcule a probabilidade de, nesse prazo:
   \begin{enumerate}[(a)]
   \item todos sobreviverem;
   \item apenas dois sobreviverem.
   \end{enumerate}
     \begin{sol}
      \phantom{A}  
       \begin{enumerate} [(a)]
           \item $80\%\cdot80\%\cdot80\%=0,512=51,2\%$
           \item $0,8\cdot0,8\cdot0,2\cdot\mathrm{C}_{3,2}=0,384=38,4\%$
       \end{enumerate}
     \end{sol}
\end{ex}