\begin{ex}
 (Mack) Num tribunal, 10 réus devem ser julgados isoladamente num mesmo dia. Três são paulistas, 2 são mineiros, 3 são gaúchos e 2 são baianos. O número de formas de não julgar consecutivamente 3 paulistas é:
    \begin{enumerate}[(a)]
    \item $P_7$
    \item $P_8$
    \item $P_{10} - P_8$
    \item $P_{10} - P_3$
    \item $P_{10} - P_8$  \text{x}  $P_3 $
    \end{enumerate}
      \begin{sol}
        resposta: e \\
        total (-) os 3 paulistas juntos:\hspace{0,5cm}
        $\mathrm{P}_{10}-\mathrm{P}_{8}\times\mathrm{P}_3$
      \end{sol}
\end{ex}