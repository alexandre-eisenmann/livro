\begin{ex}
 (Ita) Considere os números de 2 a 6 dígitos distintos utilizando-se apenas 1, 2, 4, 5, 7 e 8. Quantos desses números são ímpares e começam com um dígito par?
    \begin{enumerate}[(a)]
    \item 375
    \item 465
    \item 545
    \item 585
    \item 625
    \end{enumerate}
      \begin{sol}
     resposta: d \\
     números com:  
     2 dígitos: $\frac{3}{\mathrm{P}}\frac{3}{\mathrm{I}}= 9$\hspace{0.2cm} ou \hspace{0.2cm}      3 dígitos:  $\frac{3}{\mathrm{P}}\frac{4}{\phantom{A}}\frac{3}{\mathrm{I}}=36$\hspace{0.2cm} ou 
     4 dígitos: $\frac{3}{\mathrm{P}}\frac{4}{\phantom{A}} \frac{3}{\phantom{A}}\frac{3}{\mathrm{I}}=108$\hspace{0.2cm} ou\hspace{0.2 cm} 5 dígitos: $\frac{3}{\mathrm{P}}\frac{4}{\phantom{A}}\frac{3}{\phantom{A}}\frac{2}{\phantom{A}}\frac{3}{\mathrm{I}}= 216$\hspace{0.2cm}ou\hspace{0.2cm}$\frac{3}{\mathrm{P}}\frac{4}{\phantom{A}}\frac{3}{\phantom{A}}\frac{2}{\phantom{A}}\frac{1}{\phantom{A}}\frac{3}{\mathrm{I}}= 216$\\
     $9+36+108+216+216 = 585$
      \end{sol}
\end{ex}