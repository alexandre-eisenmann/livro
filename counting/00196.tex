\begin{ex}
 (Cescem) Em uma sala de aula existem 5 crianças: uma brasileira, uma italiana, uma japonesa, uma inglesa e uma francesa. Em uma urna existem 5 bandeiras correspondentes aos países de origem dessas crianças: Brasil, Itália, Japão, Inglaterra e França. Uma criança e uma bandeira são selecionadas ao acaso, respectivamente da sala e de uma urna. A probabilidade de que a criança sorteada não receba a sua bandeira vale:
   \begin{enumerate}[(a)]
   \item $\frac{1}{25}$
   \item $\frac{5}{25}$
   \item $\frac{25}{25}$
   \item $\frac{20}{25}$
   \item $\frac{5}{20}$
   \end{enumerate}
    \begin{sol}
     resposta: d \\
     total (-) uma criança receber a sua bandeira 
     $\rightarrow  1-1\cdot\frac{1}{5}=\frac{4}{5}=\frac{20}{25}$
    \end{sol}

O texto a seguir refere-se aos exercícios: 591 e 592\\
(Cescem)  Sabendo-se que os erros de impressão tipográfica, por página impressa, se distribuem de acordo com as seguintes probabilidades:
\begin{center}
\begin{tabular}{|c|c|} \hline
Nº erros/página & Probabilidade  \\  \hline
0  & 0,70 \\  \hline
1  & 0,15 \\  \hline
2  & 0,10 \\  \hline
3  & 0,02  \\ \hline
4  & 0,02  \\ \hline
5 ou + & 0,01 \\ \hline
\end{tabular}
\end{center}
\end{ex}