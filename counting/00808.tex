\begin{ex}
 	Numa caixa X há 8 bolas pretas e 4 bolas brancas e em outra caixa Y há 8 bolas brancas e 4 bolas pretas. Todas as bolas são idênticas com exceção da cor. Vamos retirar duas bolas de cada uma. Calcule o total de possibilidades em que todas as bolas tenham uma mesma cor.
 	  \begin{sol}
 	  \phantom{A}  \\
 	  2 pretas da X e 2 pretas da Y ou 2 brancas da X e 2 pretas da Y \\
 	  $\frac{8}{12}\cdot\frac{7}{11}\cdot\frac{4}{12}\cdot\frac{3}{11}+\frac{4}{12}\cdot\frac{3}{11}\cdot\frac{8}{12}\cdot\frac{7}{11}=2\cdot\frac{8}{12}\cdot\frac{7}{11}\cdot\frac{4}{12}\cdot\frac{3}{11}=\frac{28}{363}$ \\ \\
 	  ou usando combinação $\frac{\mathrm{C}_{8,2}}{\mathrm{C}_{{12},2}}\cdot\frac{\mathrm{C}_{4,2}}{\mathrm{C}_{{12},2}}+\frac{\mathrm{C}_{4,2}}{\mathrm{C}_{{12},2}}\cdot\frac{\mathrm{C}_{8,2}}{\mathrm{C}_{{12},2}}=2\cdot\frac{\mathrm{C}_{8,2}}{\mathrm{C}_{{12},2}}\cdot\frac{\mathrm{C}_{4,2}}{\mathrm{C}_{{12},2}}=\frac{28}{363}$ 
 	  
 	  \end{sol}
\end{ex}