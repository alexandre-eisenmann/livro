\begin{ex}
Um grupo de pesquisadores estudou a relação entre a presença de um gene A em indivíduos e a chance desse indivíduo desenvolver uma doença X, que tem tratamento mas não apresenta cura. Os dados do estudo mostraram que 8\% da população é portadora do gene A e 10\% da população sofre da doença X. Além disso, 88\% da população não é portadora do gene A nem sofre da doença X. De acordo com esses dados, se uma pessoa sofre da doença X, qual é a probabilidade de que seja portadora do gene A?
  \begin{sol}
    \phantom{A} \\
    \begin{tabular}{|c|c|c|c|} \hline
         & A & não A & total   \\ \hline
        X & 6\% & 4\% & 10\% \\  \hline
     não X& 2\% & 88\% & 90\% \\ \hline
     total& 8\% & 92\% &100\%  \\ \hline
     
    \end{tabular}
    $\Longrightarrow \frac{6\%}{10\%}=60\%$
  \end{sol}
\end{ex}