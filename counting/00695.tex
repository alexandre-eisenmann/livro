\begin{ex}
 Em uma pesquisa realizada em uma faculdade foram feitas duas perguntas aos alunos. 120 alunos responderam "sim" a ambas, 300 responderam "sim" à primeira, 250 responderam "sim" à segunda e 200 responderam "não" a ambas. Se um aluno for escolhido ao acaso, qual a probabilidade de ele ter respondido "não" à primeira pergunta?
    \begin{enumerate}[(a)]
    \item $\frac{1}{7}$
    \item 50\%
    \item $\frac{3}{8}$
    \item $\frac{11}{21}$
    \item $\frac{4}{25}$
    \end{enumerate}
      \begin{sol}
      resposta: d \\
      Diagrama\\ \\
         \begin{venndiagram2sets}[labelA= $\mathrm{P}_1$,labelB=$\mathrm{P}_2$,labelOnlyA=180,labelOnlyB=130,labelAB=120,labelNotAB=$\mathrm{200}$] 
          \end{venndiagram2sets}
          \\
      número total de alunos= \(180+120+130+200=630\)\\
       \(630-300=330\) responderam não à primeira pergunta $\longrightarrow p=\frac{330}{630}=\frac{11}{21}$
      \end{sol}
\end{ex}