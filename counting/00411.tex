\begin{ex}
No Brasil as placas de automóvel são formadas por uma sequência de três letras seguidas de uma sequência de quatro algarismos.
   \begin{enumerate}[(a)]
   \item quantas placas diferentes podem ser formadas com as letras A,  B, C e D e com os algarismos 1, 2, 3, 4 e 5?
   \item quantas placas podem ser formadas com as letras A, B, C e D e com os algarismos 1, 2, 3, 4 e 5?
   \item quantas placas diferentes podem ser formadas com pelo menos um algarismo não nulo empregando-se as 26 letras do alfabeto e os 10 algarismos do sistema decimal?
   \end{enumerate}
    \begin{sol}
      \phantom{A} 
        \begin{enumerate} [(a)]
            \item $4^3\cdot5^4=40000$
            \item $4\cdot3\cdot2\cdot5\cdot4\cdot3\cdot2=2880$
            \item todas (-) um algarismo nulo $\Longrightarrow 26^3\cdot10^4-26^3=175.742.424$
        \end{enumerate}
    \end{sol}
\end{ex}