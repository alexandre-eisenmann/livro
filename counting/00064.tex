\begin{ex}
 (Ita) Considere os seguintes resultados relativamente ao lançamento de uma moeda:
   \begin{enumerate} [I.]
       \item 	Ocorrência de duas caras em dois lançamentos. 
       \item Ocorrência de três caras e uma coroa em quatro lançamentos.
       \item Ocorrência de cinco caras e três coroas em oito lançamentos. 
   \end{enumerate}
  Pode-se afirmar que :
   \begin{enumerate} [(a)]
       \item dos três resultados, I é o mais provável.
       \item dos três resultados, II é o mais provável. 
       \item dos três resultados, III é o mais provável.
       \item os resultados I e II são igualmente prováveis.
       \item os resultados II e III são igualmente prováveis.
   \end{enumerate}
     \begin{sol}
     resposta: d \\
     $p_I= \frac{1}{2}\cdot\frac{1}{2}=\frac{1}{4}=0,25$ \\
     $p_{II}=\mathrm{C}_{4,3}\cdot(\frac{1}{2})^3\cdot(\frac{1}{2})^1=\frac{1}{4}=0,25$ \\
     $p_{III}=\mathrm{C}_{8,5}\cdot(\frac{1}{2})^5\cdot(\frac{1}{2})^3=\frac{7}{2^5}=\frac{7}{32}\approx 0,21$
     \end{sol}
\end{ex}