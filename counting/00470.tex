\begin{ex}
(Uel) Contra certa doença podem ser aplicadas as vacinas I ou II. A vacina I falha em 10\% dos casos e a vacina II em 20\% dos casos, sendo esses eventos totalmente independentes. Nessas condições, se todos os habitantes de uma cidade receberem doses adequadas das duas vacinas, a probabilidade de um indivíduo NÃO estar imunizado contra a doença é:
   \begin{enumerate}[(a)]
   \item 30\%
   \item 10\%
   \item 3\%
   \item 2\%
   \item 1\%
   \end{enumerate}
     \begin{sol}
      resposta: d \\
      $10\% \cdot 20\% = 2\%$
     \end{sol}
\end{ex}