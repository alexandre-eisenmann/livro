\begin{ex}
Em uma classe do segundo ano do ensino médio, precisamente 64\% dos alunos leem jornal, 48\% leem revistas e 10\% não leem jornal nem revista. Escolhendo um desses alunos ao acaso, qual é a probabilidade de que ele seja leitor de jornal e de revista ?
  \begin{sol}
   \phantom{A} \\ \\
    \begin{venndiagram2sets} [labelA=\(J\),labelB=\(R\),labelOnlyA=64-p,labelOnlyB=48-p,labelAB=p,labelNotAB=10] 
    \end{venndiagram2sets}
    $64-p+p+48-p+10=100 \Longrightarrow p=22\%$
    
  \end{sol}
\end{ex}