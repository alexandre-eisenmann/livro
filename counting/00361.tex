\begin{ex}
(Pucc – adaptado) No jogo da Mega Sena, um apostador pode assinalar entre 6 e 15 números de um total de 60 opções disponíveis. O valor da aposta é igual a R\$ 2,00 multiplicado pelo número de sequências  de 6 números que são possíveis, a partir daqueles números assinalados pelo apostador.
Por exemplo: se o apostador assinala 6 números tem apenas uma sequência favorável e paga R\$ 2,00 pela aposta.  Se o apostador assinala 7 números , tem 7 sequências favoráveis ou seja, é possível formar 7 sequências de 6 números a partir dos 7 números escolhidos: $\frac{7.6.5.4.3.2.1}{6!} = 7$. Neste caso, o valor da aposta é R\$ 14,00. Considerando que se trata de uma aplicação de matemática, sem apologia a qualquer tipo de jogo, responda:
   \begin{enumerate}[(a)]
   \item	é verdade que apostar dois cartões com 10 números assinalados ou 5 cartões com 9 números assinalados, são opções equivalentes em termos de custo? Justifique através de cálculos.
   \item 	assinalando 7 números, qual a probabilidade do apostador acertar somente 4 destes?
   \end{enumerate}
     \begin{sol}
      \phantom{a}
       \begin{enumerate} [(a)]
           \item (2 cartões) $\rightarrow 2\cdot\frac{10.9.8.7.6.5}{6.5.4.3.2.1}=120$\hspace{0,2cm}reais\\
            (5 cartões) $\rightarrow 5\cdot\frac{9.8.7.6.5.4}{6.5.4.3.2.1}=120$\hspace{0,2cm}reais \\
            resposta: sim
           \item probabilidade de acertar um número $= \frac{1}{60};$\hspace{0,2cm} probabilidade de errar $=\frac{59}{60}$ \\
           $\Longrightarrow (\frac{1}{60})^4\cdot(\frac{59}{60})^2=\frac{59^2}{60^6}$
           
       \end{enumerate}
     \end{sol}
\end{ex}