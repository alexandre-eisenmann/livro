\begin{ex}
  (Mack) Antônio, José, Pedro, Maria e Renata foram comemorar o aniversário de Antônio em uma churrascaria da cidade. O garçom que os recebeu acomodou-os prontamente em uma mesa redonda para 5 pessoas e assim que todos se sentaram Antônio percebeu que, sem querer, haviam sentado em volta da mesa por ordem de idade, isto é, a partir do segundo mais novo até o mais velho, cada um tinha como vizinho do mesmo lado, o colega imediatamente mais novo. A probabilidade de isso ocorrer se os cinco amigos sentassem aleatoriamente é:
    \begin{enumerate}  [(a)]
        \item $\frac{1}{2}$
        \item $\frac{1}{4}$
        \item $\frac{1}{6}$
        \item $\frac{1}{12}$
        \item $\frac{1}{24}$
    \end{enumerate}
      \begin{sol}
       resposta: d \\
       permutação circular: $P_n=(n-1)!$ \\
       nº de casos possíveis: $P_5=4!= 24$ agrupamentos \\
       nº casos favoráveis: 2 (sentido horário e anti-horário) 
       $\Longrightarrow  p=\frac{2}{24}=\frac{1}{12}$
      \end{sol}
 \end{ex}