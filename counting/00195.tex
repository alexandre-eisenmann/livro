\begin{ex}
Numa urna há 14 bolas numeradas de 1 a 14. Dela, serão sorteadas ao acaso duas bolas. Você está no local do sorteio, e antes que ele se realize, um amigo lhe diz: “ Adivinhe quantas das duas bolas sorteadas terão números de 1 a 10". Para dar a resposta que tem maior probabilidade de estar certa, o que você deve responder: que os números de 1 a 10 aparecerão nas 2 bolas sorteadas, só em uma delas ou em nenhuma?
 \begin{sol}
  \phantom{A} \\
 - nas duas bolas:\hspace{0,2cm}$\frac{10}{14}\cdot\frac{9}{13}=\frac{49}{91}$\\
 - em uma delas:\hspace{0,2cm} $\frac{10}{14}\cdot\frac{4}{13}=\frac{20}{91}$ \\
 - nenhuma:\hspace{0,2cm} $\frac{4}{14}\cdot\frac{3}{13}=\frac{15}{91}$\\
 resposta: nas duas
 \end{sol}
 \end{ex}