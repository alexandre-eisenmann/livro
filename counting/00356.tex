\begin{ex}
(Unemat – adaptado) Numa das salas de concurso de vestibular, há 40 candidatos do sexo masculino e feminino, concorrendo aos cursos de Matemática e de Computação, distribuídos conforme o quadro abaixo:
\begin{center}
\begin{tabular}{|c|c|c|}  \hline
   &  MATEMÁTICA  &   COMPUTAÇÃO     \\  \hline
MASCULINO &     15       &     10           \\  \hline
FEMININO  &     10       &     05           \\  \hline   
\end{tabular}
\end{center}
Antes do início da prova, será sorteado um candidato para abrir o envelope lacrado.	Com base na distribuição do quadro acima, associe V ou F:
   \begin{itemize}
   \item [(    )]  A probabilidade do candidato sorteado ser da Computação e Feminino é $\frac{1}{8}$.
   \item [(   )] A probabilidade do candidato sorteado ser da Matemática ou Feminino é $\frac{3}{4}$.
   \item [(   )] A probabilidade do candidato sorteado ser de Matemática é $\frac{5}{4}$.
   \item [(   )]  A probabilidade de o candidato sorteado ser da Computação sabendo que é Masculino é $\frac{1}{4}$.
   \end{itemize}
     \begin{sol}
      \phantom{A} \\
      V - V - F - F
     \end{sol}
\end{ex}