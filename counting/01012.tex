\begin{ex}
 (Puc) Numa caixa há 100 bolas, numeradas de 1 a 100. Retiram-se, simultaneamente duas bolas.  Qual é a probabilidade de se obterem números consecutivos?
    \begin{enumerate}[(a)]
    \item $\frac{1}{2}$
    \item $\frac{1}{50}$
    \item $\frac{9}{100}$
    \item $(\frac{1}{100})^2$
    \item $(\frac{99}{100})^2$
    \end{enumerate}
      \begin{sol}
        resposta: b \\
        espaço amostral : $\mathrm{C}_{{100},2}$ \\
        -podemos formar : (1,2) (2,3) (3,4) .....(99,100) = 99 pares de números consecutivos \\
        $\Longrightarrow p=\frac{99}{\mathrm{C}_{{100},2}}=\frac{1}{50}$
      \end{sol}
    
\end{ex}