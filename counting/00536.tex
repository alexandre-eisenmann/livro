\begin{ex}
(Uel) Na formação de uma Comissão Parlamentar de Inquérito (CPI), cada partido indica um certo número de membros, de acordo com o tamanho de sua representação no Congresso Nacional. Faltam apenas dois partidos, A e B, para indicar seus membros. O partido A tem 40 deputados e deve indicar 3 membros, enquanto o partido B tem 15 deputados e deve indicar 1 membro. Assinale a alternativa que apresenta o número de possibilidades diferentes para a composição dos membros desses dois partidos nessa CPI.
   \begin{enumerate}[(a)]
   \item 55
   \item $(40 - 3)\cdot(15 - 1)$
   \item $\frac{40!}{37!.3!}\cdot 15$
   \item $40\cdot39\cdot38\cdot15$
   \item $40!\cdot37!\cdot15!$
   \end{enumerate}
     \begin{sol}
      resposta: c \\
      $\mathrm{C}_{{40},3}\cdot \mathrm{C}_{{15},1}=\frac{40!}{37!\cdot3!}\cdot15$
     \end{sol}
\end{ex}