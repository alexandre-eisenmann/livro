\begin{ex}
 (Unesp) Um colégio possui duas salas, A e B, de determinada série. Na sala A, estudam 20  alunos e na B, 30 alunos. Dois amigos, Pedro e João, estudam na sala  A. Um aluno é sorteado da sala A  e transferido para a B. Posteriormente, um aluno é sorteado e transferido da sala B  para a sala A .
   \begin{enumerate} [(a)]
       \item No primeiro sorteio, qual a probabilidade de qualquer um dos dois amigos ser transferido da sala A para a B?
       \item Qual a probabilidade, no final das transferências, de os amigos ficarem na mesma sala?
   \end{enumerate}
     \begin{sol}
     \phantom{A}
       \begin{enumerate} [(a)]
           \item $\frac{1}{20}+\frac{1}{20}=\frac{2}{20}=\frac{1}{10}=10\%$
           \item os 2 amigos terminarão na mesma sala se nenhum dos 2 for transferido no 1º sorteio ou se o mesmo amigo for transferido nos 2 sorteios: $\frac{18}{20}+\frac{2}{20}\cdot\frac{1}{31}=\frac{280}{310}=\frac{28}{31}$
       \end{enumerate}
     \end{sol}
\end{ex}