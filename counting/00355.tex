\begin{ex}
Para que um número de 4 algarismos seja par é preciso que ele “termine” por um número par, ou, em outras palavras que o algarismo das unidades seja: 0, 2, 4, 6 ou 8. Então nesse caso, por exemplo: 3542; 6134 e 9200 são números pares.
   \begin{enumerate}[(a)]
   \item quantos números pares de 4 algarismos existem?
   \item quantos números ímpares de 4 algarismos distintos existem?
   \end{enumerate}
     \begin{sol}
      \phantom{A}
        \begin{enumerate} [(a)]
            \item $\frac{\phantom{A}}{9}\frac{\phantom{A}}{10}\frac{\phantom{A}}{10}\frac{\phantom{A}}{5}=4500$
            \item $\frac{\phantom{A}}{8}\frac{\phantom{A}}{9}\frac{\phantom{A}}{8}\frac{\phantom{A}}{5}=2880$
        \end{enumerate}
     \end{sol}
\end{ex}