\begin{ex}
 (Unicamp) Seja \textit{S} o conjunto dos números naturais cuja representação decimal é formada apenas pelos algarismos 0, 1, 2, 3 e 4.
    \begin{enumerate}[(a)]
    \item seja \textit{x} um número de 10 algarismos pertencente a S, cujos dois últimos algarismos tem igual probabilidade de assumir qualquer valor inteiro de 0 a 4. Qual a probabilidade de que \textit{x} seja divisível por 15?
    \begin{center}
\begin{tabular} {|c|c|c|c|c|c|c|c|c|c|} \hline
2 & 0 & 3 & 4 & 1 & 3 & 2 & 1 & ? & ? \\ \hline
\end{tabular}
\end{center}
    \item  quantos números menores que um bilhão e múltiplos de 4 pertencem ao conjunto \textit{S}?
    \end{enumerate}

  \begin{sol}
    \phantom{A} 
      \begin{enumerate} [(a)]
          \item número divisível por 15 deve ser divisível por 3 e 5 ao mesmo tempo, logo o último algarismo deve ser 0.\\
          Para achar o penúltimo algarismo, somamos todos os algarismos de \textit{S}.\\Resultado : 16 , logo o penúltimo algarismo é 2
          $\Longrightarrow p=\frac{1}{5}\cdot\frac{1}{5}=\frac{1}{25}$
          \item número múltiplo de 4: dezenas e unidades devem ser múltiplos de 4 \\ 00  04  12  20  24  32  40 e 44  (8 opções) \\
          $\Longrightarrow5^7\cdot8 = 5^7\cdot2^3 =5^4\cdot{10}^3=625000$
      \end{enumerate}
  \end{sol}
\end{ex}