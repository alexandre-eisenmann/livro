\begin{ex}
(Vunesp) Uma pesquisa publicada pela revista Veja de 7 de junho de 2006 sobre os hábitos alimentares dos brasileiros mostrou que, no almoço, aproximadamente 70\% dos brasileiros comem carne bovina e que, no jantar, esse índice cai para 50\%. Supondo que a probabilidade condicional de uma pessoa comer carne bovina no jantar dado que ela comeu carne bovina no almoço seja 0,6, determine a probabilidade de a pessoa comer carne bovina no almoço ou no jantar.
  \begin{sol}
   \phantom{A}\\
   $P(B/A)=\frac{P(B\cap A)}{P(A)}
   \Longrightarrow \frac{6}{10}=\frac{P(B \cap A)}{\frac{7}{10}}\Longrightarrow P(B \cap A)=\frac{42}{100} $\\
   $P(A \cup B)=P(A) +P(B) -P(B\cap A) \Longrightarrow \frac{7}{10}+\frac{5}{10}-\frac{42}{100}=\frac{78}{100}=78\%$
  \end{sol}
\end{ex}