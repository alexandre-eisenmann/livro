\begin{ex}
Calcule a quantidade de números naturais compreendidos entre 300 e 3000 que podemos representar utilizando somente os algarismos 1, 2, 3, 5, 7 e 8 de modo que não haja algarismos repetidos (Sugestão: separe a resolução em dois casos).
  \begin{sol}
    \phantom{A} \\
    $\frac{3}{\phantom{A}}\frac{\phantom{A}}{5}\frac{\phantom{A}}{4}=20\cdot4=80$ \hspace{0,6cm} (+)\hspace{0,6cm}
    $\frac{1}{\phantom{A}}\frac{\phantom{A}}{5}\frac{\phantom{A}}{4}\frac{\phantom{A}}{3}=  60\cdot2=120$ \hspace{0,3cm}
    $\Longrightarrow 80 +120=200$
  \end{sol}
\end{ex}