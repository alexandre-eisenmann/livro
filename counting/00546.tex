\begin{ex}
(Enem) A população mundial está ficando mais velha, os índices de natalidade diminuíram e a expectativa de vida aumentou. No gráfico seguinte, são apresentados dados obtidos por pesquisa realizada pela Organização das Nações Unidas (ONU), a respeito da quantidade de pessoas com 60 anos ou mais em todo o mundo. Os números da coluna da direita representam as faixas percentuais. Por exemplo, em 1950, havia 95 milhões de pessoas com 60 anos ou mais nos países desenvolvidos, número entre 10\% e 15\% da população total nos países desenvolvidos. 
\begin{center}
\includegraphics[width=10cm]{imagens/enem_ex_940.png}
\end{center}
Em 2050, a probabilidade de se escolher, aleatoriamente, uma pessoa com 60 anos ou mais de idade, na população dos países desenvolvidos, será um número mais próximo de:
   \begin{enumerate}[(a)]
   \item $\frac{1}{2}$
   \item $\frac{7}{20}$
   \item $\frac{8}{25}$
   \item $\frac{1}{5}$
   \item $\frac{3}{25}$
   \end{enumerate}
     \begin{sol}
      resposta: c \\
      a resposta está na linha vermelha (países desenvolvidos) \\
      um número entre 30 e 35 $\Longrightarrow \frac{8}{25}=32\%$
     \end{sol}
\end{ex}