\begin{ex}
(UFPE) O vírus $X$ aparece nas variantes $X_1$ e $X_2$. Se um indivíduo tem esse vírus, a probabilidade de ser a variante  $X_1$ é $\frac{3}{5}$  . Se o indivíduo tem o vírus $X_1$, a probabilidade de esse sobreviver é de $\frac{2}{3}$ ; mas se o indivíduo tem o vírus $X_2$, a probabilidade de ele sobreviver é de $\frac{5}{6}$ . Nessas condições, qual é a probabilidade de o indivíduo portador do vírus $X$ sobreviver?
   \begin{enumerate}[(a)]
   \item $\frac{1}{3}$
   \item $\frac{7}{15}$
   \item $\frac{3}{5}$
   \item $\frac{2}{3}$
   \item $\frac{11}{15}$
   \end{enumerate}
     \begin{sol}
      resposta: e \\
      $\frac{3}{5}\cdot\frac{2}{3}+\frac{2}{5}\cdot\frac{5}{6}=\frac{11}{15}$
     \end{sol}
\end{ex}