\begin{ex}
(UFMG) Lílian possui sete pares de meias brancas, quatro pares de meias cinza, três pares de
meias pretas e cinco pares de meias azuis.
Sabe-se que as meias de mesma cor são idênticas.
Suponha que todas essas meias estão embaralhadas em uma gaveta e que Lílian retira
dela, aleatoriamente, certo número de meias.
Considerando essas informações, determine:
 \begin{enumerate}  [(a)]
     \item o número mínimo de pés de meia que Lílian deve retirar dessa gaveta para ter certeza de ter, pelo menos, um par de meias de uma mesma cor.
     \item a probabilidade de Lílian, ao retirar exatamente dois pés de meia dessa gaveta, obter um par de meias de uma mesma cor.
     \item a probabilidade de Lílian, ao retirar quatro pés de meia dessa gaveta, obter, pelo menos,um par de meias de uma mesma cor.
 \end{enumerate}
   \begin{sol}
    \phantom{A}
      \begin{enumerate} [(a)]
          \item são 4 cores diferentes, logo ela deve retirar 5 meias.
          \item mesma cor: 2 brancas ou 2 cinzas ou 2 pretas ou 2 azuis:\\
          $p=\frac{14}{38}\cdot\frac{13}{37}+\frac{8}{38}\cdot\frac{7}{37}+\frac{6}{38}\cdot\frac{5}{37}+\frac{10}{38}\cdot\frac{9}{37}=\frac{179}{703}$
          \item pelo menos um par de meias de mesma cor = 1 - probabilidade de retirar pares de meia de cores diferentes: \hspace{0,4cm}
          $p= 1-(4!\cdot\frac{14}{38}\cdot\frac{8}{37}\cdot\frac{6}{36}\cdot\frac{10}{35})=\frac{639}{703}$
      \end{enumerate}
   \end{sol}
 \end{ex}