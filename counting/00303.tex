\begin{ex}
(UF – RJ)”Protéticos e dentistas dizem que a procura por dentes postiços não aumentou. Até declinou um pouquinho. No Brasil, segundo a Associação Brasileira de Odontologia (ABO), há 1,4 milhão de pessoas sem nenhum dente na boca e 80\% dela já usam dentadura. Assunto encerrado” (Adaptado de Veja, out 1997).\\
Considere que a população brasileira seja de 160 milhões de habitantes. Escolhendo ao acaso um desses habitantes, a probabilidade de que ele não possua nenhum dente na boca e use dentadura, de acordo com a ABO, é de:
   \begin{enumerate}[(a)]
   \item 0,28\%
   \item 0,56\%
   \item 0,70\%
   \item 0,80\%
   \item 0,60\%
   \end{enumerate}
     \begin{sol}
       resposta: c \\
       $80\% \cdot 1.400.000= 1.120.000 \Longrightarrow p=\frac{1.120.000}{160.000.000}=0,007=0,7\%$
     \end{sol}
\end{ex}