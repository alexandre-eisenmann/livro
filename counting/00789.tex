\begin{ex}
 (Cesgranrio) Um dado comum (não viciado) teve 4 de suas faces pintadas de vermelho e as outras duas, de azul. Se esse dado for lançado três vezes, a probabilidade de que, em no mínimo dois lançamentos a face voltada para cima seja azul, será, aproximadamente de:
    \begin{enumerate}[(a)]
    \item 52,6\%
    \item 44,4\%
    \item 66,7\%
    \item 25,9\%
    \item 22,2\%
    \end{enumerate}
     \begin{sol}
     resposta: d  \\
     probabilidade de face azul=$\frac{1}{3}$; probabilidade de face vermelha=$\frac{2}{3}$\\
     2 ou 3 faces azuis: $\Longrightarrow\frac{1}{3}\cdot\frac{1}{3}\cdot\frac{2}{3}\cdot\mathrm{C}_{3,2}+\frac{1}{3}\cdot\frac{1}{3}\cdot\frac{1}{3}= \frac{7}{27}=25,9\%$
     \end{sol}
\end{ex}