\begin{ex}
Uma caixa contém 3 bolas vermelhas, 2 pretas e uma azul. Outra caixa contém 5 bolas amarelas. Uma bola é retirada da primeira caixa e colocada na segunda. Depois, uma bola é retirada da segunda e colocada na primeira. Qual a probabilidade de que as bolas pretas fiquem separadas?
 \begin{sol}
   \phantom{A} \\
   probabilidade de bola preta ser retirada da 1ª caixa e colocada na 2ª caixa: $\frac{2}{6}$  \\
   probabilidade de bola amarela da 2ª caixa ser colocada na 1ª caixa: $\frac{5}{6}$ \\
   $\Longrightarrow p = \frac{2}{6}\cdot\frac{5}{6}= \frac{5}{18}$
 \end{sol}
\end{ex}