\begin{ex}
 (Enem) O tênis é um esporte em que a estratégia de jogo a ser adotada depende, entre outros fatores, de o adversário ser canhoto ou destro. Um clube tem um grupo de 10 tenistas, sendo que 4 são canhotos e 6 são destros. O técnico do clube deseja realizar uma partida de exibição entre dois desses jogadores, porém, não poderão ser ambos canhotos. Qual o número de possibilidades de escolha dos tenistas para a partida de exibição?
    \begin{enumerate}[(a)]
    \item  $\frac{10!}{2!\cdot8!}-\frac{4!}{2!\cdot2!}$
    \item  $\frac{10!}{8!}-\frac{4!}{2!}$
    \item  $\frac{10!}{2!\cdot8!}-2$ 
    \item  $\frac{6!}{4!}+4\cdot4$
    \item  $\frac{6!}{4!}+6\cdot4$
    \end{enumerate}
      \begin{sol}
       resposta: a \\
       todas as duplas menos as que tem 2 canhotos juntos \\
       $\mathrm{C}_{{10},2}-\mathrm{C}_{4,2}\Longrightarrow\frac{10!}{2!\cdot8!}-\frac{4!}{2!\cdot2!}$
      \end{sol}
    
\end{ex}