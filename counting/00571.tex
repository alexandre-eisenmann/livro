\begin{ex}
(Ufpe) Uma fábrica usa, nos seus produtos, um sistema de codificação cujos códigos são sequências formadas com uma das 26 letras do alfabeto (incluídas K, W e Y) seguidas de dois dígitos de 0 a 9 (exemplos: S90, K23). Calcule a probabilidade de um código desse sistema, escolhido aleatoriamente, ter uma vogal ou dois dígitos iguais.
  \begin{sol}
   \phantom{A} \\
   $P(A\cup B)=P(A)+P(B)-P(A\cap B) \\
   P(A\cup B)= \frac{5}{26}+\frac{10}{100}-\frac{5}{26}\cdot\frac{1}{10} \Longrightarrow \frac{71}{260}$
  \end{sol}
\end{ex}