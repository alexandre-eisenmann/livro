\begin{ex}
 (UF-RN) "Blocos lógicos"é uma coleção de peças utilizadas no ensino da Matemática. São 48 peças construídas combinando-se 3 cores (azul, vermelha e amarela), 4 formas (triangular, quadrada, retangular e circular), 2 tamanhos (grande e pequeno) e 2 espessuras (grossa e fina). Cada peça tem apenas uma cor, uma forma, um tamanho e uma espessura. Se uma criança pegar uma peça, aleatoriamente, a probabilidade dessa peça ser amarela e grande, é:
    \begin{enumerate}[(a)]
    \item $\frac{1}{2}$
    \item $\frac{1}{6}$
    \item $\frac{1}{3}$
    \item $\frac{1}{12}$
    \end{enumerate}
      \begin{sol}
         resposta: b  \\
         3 cores, 4 formas, 2 tamanhos e 2 espessuras = $3\cdot4\cdot2\cdot2=48$ possibilidades \\
         amarela e grande (pode ter qualquer formato e qualquer espessura) $\longrightarrow 4\cdot2 =8$ \\
         $\Longrightarrow p=\frac{8}{48}=\frac{1}{6}$  
         \end{sol}
\end{ex}