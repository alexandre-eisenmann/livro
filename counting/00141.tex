\begin{ex}
As letras da palavra PRISMA foram escritas em cartões de igual tamanho, uma letra diferente em cada cartão. Em seguida, os cartões foram dobrados, misturados e, finalmente, ao acaso, enfileirados e desvirados. Qual é a probabilidade de que a palavra formada pela sequência das letras escritas nos cartões:
   \begin{enumerate}[(a)]
   \item tenha todas as vogais juntas?
   \item não tenha as letras P, R e I juntas?
   \end{enumerate}
     \begin{sol}
       \phantom{A} 
   \begin{enumerate} [(a)]
       \item $\frac{2\cdot5!}{6!}=\frac{1}{3}$
       \item 1- (PRI) juntas: $\Longrightarrow (1-\frac{4!\cdot3!}{6!})=\frac{4}{5}$
   
       
   \end{enumerate}
     \end{sol}
\end{ex}