\begin{ex}
Um mágico colocou em sua cartola 4 cartas de copas, 3 de paus e 2 de espadas. A seguir, pediu a uma criança que retirasse simultaneamente 3 cartas da cartola. Calcule a probabilidade de a criança ter retirado:
   \begin{enumerate}[(a)]
   \item 3 cartas de copas.
   \item 2 cartas de copas e 1 de paus.
   \item 3 cartas de naipes diferentes.
   \end{enumerate}
     \begin{sol}
        \phantom{A} 
        \begin{enumerate} [(a)]
            \item $\frac{\mathrm{C}_{4,3}}{\mathrm{C}_{9,3}}=\frac{4}{84}=\frac{1}{21}$\hspace{0,3cm} \textbf{ou}\hspace{0,3cm} $\frac{4}{9}\cdot\frac{3}{8}\cdot\frac{2}{7}=\frac{1}{21}$
            \item $\frac{\mathrm{C}_{4,2}\cdot\mathrm{C}_{3,1}}{\mathrm{C}_{9,3}}=\frac{3}{14}$\hspace{0,3cm} \textbf{ou} \hspace{0,3cm} $\frac{4}{9}\cdot\frac{3}{8}\cdot\frac{3}{7}\cdot3=\frac{3}{14}$
            \item $\frac{\mathrm{C}_{4,1}\cdot\mathrm{C}_{3,1}\cdot\mathrm{C}_{2,1}}{\mathrm{C}_{9,3}}=\frac{2}{7}$\hspace{0,3cm}\hspace{0,3cm} \textbf{ou} \hspace{0,3cm} $\frac{4}{9}\cdot\frac{3}{8}\cdot\frac{2}{7}\cdot3!=\frac{2}{7}$
        \end{enumerate}
     \end{sol}
\end{ex}