\begin{ex}
   (Enem) Os estilos musicais preferidos pelos jovens brasileiros são o samba, o rock e a MPB. O quadro a seguir registra o resultado de uma pesquisa relativa à preferência musical de um grupo de 1 000 alunos de uma escola. Alguns alunos disseram não ter preferência por nenhum desses três estilos.
   \begin{center}
       \begin{tabular}{|c|c|c|c|c|} \hline
          preferência musical & rock & samba & MPB & rock e samba\\  \hline
           número de alunos & 200 & 180 & 200 & 70 \\  \hline 
        \end{tabular} 
    \end{center}
    \begin{center}
       \begin{tabular}{|c|c|c|c|c|} \hline 
           preferência musical & rock e MPB  & samba e MPB & rock, samba e MPB\\  \hline
            número de alunos & 60 & 50 & 20 \\ \hline
       \end{tabular}
   \end{center}
   Se for selecionado ao acaso um estudante no grupo pesquisado, qual é a probabilidade de ele preferir somente MPB?
     \begin{enumerate} [(a)]
    \item 2\%
    \item 5\%
    \item 6\%
    \item 11\%
    \item 20\%
    \end{enumerate}
     \begin{sol}
     resposta: d \\
     Diagrama de Venn\\
       \begin{venndiagram3sets}[labelA=\(R\),labelB=\(S\),labelC=\(MPB\),labelOnlyA=90,labelOnlyB=80,labelOnlyC=110,labelABC=20,labelOnlyAB=50,labelOnlyAC=40,labelOnlyBC=30]
       \end{venndiagram3sets}
      $\Longrightarrow p = \frac{110}{1000}=11\%$
     \end{sol}
 \end{ex}