\begin{ex}
(Enem) Numa avenida existem 10 semáforos. Por causa de uma pane no sistema, os semáforos ficaram sem controle durante uma hora, e fixaram suas luzes unicamente em Verde ou Vermelho. Os semáforos funcionam de forma independente; a probabilidade de acusar a cor verde é de 2/3 e a de acusar a cor vermelha é de 1/3. Uma pessoa percorreu a pé toda essa avenida durante o período da pane, observando a cor da luz de cada um desses semáforos. Qual a probabilidade de que essa pessoa tenha observado exatamente um sinal na cor verde?
  \begin{enumerate} [(a)]
      \item $\frac{10\textbf{x}2}{3^{10}}$
      \item $\frac{10\textbf{x}2^9}{3^{10}}$
      \item $\frac{2^{10}}{3^{100}}$
      \item $\frac{2^{90}}{3^{100}}$
      \item $\frac{2}{3^{10}}$
  \end{enumerate}
   \begin{sol}
    resposta: a \\
    supondo o 1º semáforo verde e os demais vermelhos, temos:\hspace{0,1cm} $p=\frac{2}{3}\cdot(\frac{1}{3})^9$\\
    mas, qualquer um dos 10 semáforos pode estar verde, então temos: \hspace{0,2cm} $p=\frac{2}{3}\cdot(\frac{1}{3})^9\cdot10= \frac{10\textbf{x}2}{3^{10}}$
   \end{sol}
  \end{ex}