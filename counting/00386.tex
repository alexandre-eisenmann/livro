\begin{ex}
(Uel) Devido à ameaça de uma epidemia de sarampo e rubéola, os 400 alunos de uma escola foram consultados sobre as vacinas que já haviam tomado. Do total 240 haviam sido vacinados contra sarampo e 100 contra rubéola, sendo que 80 não haviam tomado nenhuma dessas vacinas. Tomando-se ao acaso um aluno dessa escola, a probabilidade dele ter tomado as duas vacinas é:
   \begin{enumerate}[(a)]
   \item 2\%
   \item 5\%
   \item 10\%
   \item 15\%
   \item 20\%
   \end{enumerate}
     \begin{sol}
      resposta : b \\
      $240+100+80=420$ \hspace{0,3cm} (20 tomaram as duas)
      $\Longrightarrow \frac{20}{400}=\frac{1}{20}=5\%$
     \end{sol}
\end{ex}