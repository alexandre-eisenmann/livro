\begin{ex}
 	(Fatec) Um grupo formado por quatro rapazes e uma senhorita vai visitar uma exposição de arte. Um dos rapazes é um perfeito cavalheiro e, não passa pela porta da sala de exposições sem que a senhorita já o tenha feito. O número de modos pelos quais eles podem entrar, consecutivamente, no recinto é:
    \begin{enumerate}[(a)]
    \item 120
    \item 60
    \item 48
    \item 64
    \item 6
    \end{enumerate}
      \begin{sol}
      resposta: b \\
      senhorita em primeiro lugar:\hspace{0,3cm}
      $\frac{\mathrm{S}}{\phantom{A}}\frac{\phantom{A}}{4}\frac{\phantom{A}}{3}\frac{\phantom{A}}{2}\frac{\phantom{A}}{1}=24$ \\
      senhorita em segundo lugar (o cavalheiro só entra depois dela, logo só 3 rapazes entram antes dela):\hspace{0,3cm}
      $\frac{\phantom{A}}{3}\frac{\mathrm{S}}{\phantom{A}}\frac{\phantom{A}}{3}\frac{\phantom{A}}{2}\frac{\phantom{A}}{1}=18$ \\
      usa-se o mesmo raciocínio para a senhorita em terceiro e quarto lugar:\hspace{0,3cm}
      $\frac{\phantom{A}}{3}\frac{\phantom{A}}{2}\frac{\mathrm{S}}{\phantom{A}}\frac{\phantom{A}}{2}\frac{\phantom{A}}{1}=12$ \\
      senhorita em quarto lugar: \hspace{0,3cm}
      $\frac{\phantom{A}}{3}\frac{\phantom{A}}{2}\frac{\phantom{A}}{1}\frac{\mathrm{S}}{\phantom{A}}=6$ \\
      $\Longrightarrow  24+18+12+6=60$
      
      \end{sol}
    
\end{ex}