\begin{ex}
Uma moeda é lançada 5 vezes. Determine a probabilidade de que:
   \begin{enumerate}[(a)]
   \item os 3 primeiros lançamentos deem cara?
   \item exatamente 3 lançamentos deem cara?
   \item pelo menos 3 lançamentos deem cara?
   \item se obtenha cara no segundo lançamento, sabendo que houve duas caras e 3 coroas?
   \end{enumerate}
     \begin{sol}
       \phantom{A} 
        \begin{enumerate} [(a)]
            \item $(\frac{1}{2})^3\cdot(\frac{2}{2})^2=\frac{1}{8}$
            \item $(\frac{1}{2})^5\cdot\mathrm{C}_{5,3}=\frac{5}{16}$
            \item 3 caras e 2 coroas ou 4 caras e 1 coroa ou 5 caras:\\
            $(\frac{1}{2})^5\cdot\mathrm{C}_{5,3}+(\frac{1}{2})^5\cdot\mathrm{C}_{5,4}+(\frac{1}{2})^5=\frac{1}{2}$
            \item colocando uma cara no 2º lançamento, a outra cara pode estar em 4 posições: $\frac{4}{\mathrm{C}_{5,2}}=\frac{2}{5}$
        \end{enumerate}
     \end{sol}
   
\end{ex}