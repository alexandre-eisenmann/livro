\begin{ex}
 	Dia 20/09/2009 foi um domingo. Vamos chamar de “anagrama da data” qualquer embaralhamento de seus dígitos,  sem levar em conta as barras de separação de dias, meses e anos.
    \begin{enumerate}[(a)]
    \item quantos anagramas tem a data?
    \item em quantos anagramas aparece o número 29 duas vezes?
    \end{enumerate}
      \begin{sol}
        \phantom{A}  
        \begin{enumerate} [(a)]
            \item $20.092.009\Longrightarrow \frac{8!}{2!4!2!}=420$
            \item $\frac{29}{\phantom{A}}\frac{29}{\phantom{A}}\frac{0}{\phantom{A}}\frac{0}{\phantom{A}}\frac{0}{\phantom{A}}\frac{0}{\phantom{A}}\longrightarrow\frac{6!}{4!}=30$
        \end{enumerate}
      \end{sol}
\end{ex}