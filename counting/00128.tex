\begin{ex}
(Ufscar) Um encontro científico conta com a participação de pesquisadores de três áreas, sendo eles: 7 químicos, 5 físicos e 4 matemáticos. No encerramento do encontro o grupo decidiu formar uma comissão de 2 cientistas para representá-los em um congresso. Tendo estabelecido que a dupla deveria ser formada por cientistas de áreas diferentes, qual é o total de duplas distintas que se poderia ter formado para representar o grupo no congresso?
  \begin{sol}
    \phantom{A} \\
    total de duplas (-) duplas de mesma área (QQ + FF + MM) \\
    $\mathrm{C}_{{16},2}-(\mathrm{C}_{7,2}+\mathrm{C}_{5,2}+\mathrm{C}_{4,2})=83$
  \end{sol}
\end{ex}