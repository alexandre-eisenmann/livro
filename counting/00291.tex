\begin{ex}
(FGV) Num certo país, 10\% das declarações de impostos de renda são suspeitas e submetidas a uma análise detalhada: entre estas verificou-se que 20\% são fraudulentas. Entre as não suspeitas, 2\% são fraudulentas.
   \begin{enumerate}[(a)]
   \item se uma declaração é escolhida ao acaso, qual é a probabilidade dela ser suspeita e fraudulenta?
   \item se uma declaração é fraudulenta, qual é a probabilidade de ela ter sido suspeita?
   \end{enumerate}
     \begin{sol}
       \phantom{A}
        \begin{enumerate} [(a)]
            \item $0,1\cdot0,2=0,02=2\%$
            \item $\frac{0,02}{0,038}=\frac{20}{38}=52,6\%$
            
        \end{enumerate}
     \end{sol}
\end{ex}