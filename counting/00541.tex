\begin{ex}
(Ufmg) Numa escola, há 10 professores de Matemática e 15 de Português. Pretende-se formar com esses professores, uma comissão de 7 membros.
   \begin{enumerate}[(a)]
   \item Quantas comissões distintas podem ser formadas?
   \item Quantas comissões distintas podem ser formadas com, pelo menos 1 professor de Matemática?
   \item Quantas comissões distintas podem ser formadas com, pelo menos 2 professores de Matemática e, pelo menos 3 professores de Português?
   \end{enumerate}
     \begin{sol}
      \phantom{A}
        \begin{enumerate}  [(a)]
            \item $\mathrm{C}_{{25},7}=480.700$
            \item total (-) sem professor de português: $\rightarrow 480.700 - \mathrm{C}_{{15},7}= 474.265$
            \item 2m e 5p ou 3m e 4p ou 4m e 3p: \\ $\mathrm{C}_{{10},2}\cdot \mathrm{C}_{{15},5}+\mathrm{C}_{{10},3}\cdot \mathrm{C}_{{15},4}+\mathrm{C}_{{10},4}\cdot \mathrm{C}_{{15},3}=45\cdot3003+120\cdot1365+210\cdot455=394.485$
        \end{enumerate}
     \end{sol}
\end{ex}