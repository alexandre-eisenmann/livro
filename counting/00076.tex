\begin{ex}
  (Unesp) Numa certa região, uma operadora telefônica utiliza 8 dígitos para designar seus números de telefones, sendo que o primeiro é sempre 3, o segundo não pode ser 0 e o terceiro número é diferente do quarto. Escolhido um número ao acaso, a probabilidade de os quatro últimos algarismos serem distintos entre si é:
    \begin{enumerate}  [(a)]
        \item $\frac{63}{125}$
        \item $\frac{567}{1250}$
        \item $\frac{189}{1250}$
        \item $\frac{63}{1250}$
        \item $\frac{7}{125}$
    \end{enumerate}
    \begin{sol}
     resposta: a \\
     probabilidade dos 4 últimos algarismos serem distintos: \\
     $p=\frac{10}{10}\cdot\frac{9}{10}\cdot\frac{8}{10}\cdot\frac{7}{10}=\frac{5040}{10000}=\frac{63}{125}$
    \end{sol}
 \end{ex}