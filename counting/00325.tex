\begin{ex}
(Vunesp) Os 500 estudantes de um colégio responderam a uma pergunta sobre qual a sua área de conhecimento preferida entre exatas, humanidades e biológicas. As respostas foram computadas e alguns dados foram colocados na tabela:
\begin{center}
\begin{tabular}{|c|c|c|c|} \hline
& Masc (M) & Fem (F) & Total \\  \hline
Exatas (E) & 120 &    & 200 \\  \hline
Humanidades (H) &   & 80 & 125 \\  \hline
Biológicas (B) & 100 &   & 175 \\  \hline
Total   &   &     &  500 \\  \hline
\end{tabular}
\end{center}
   \begin{enumerate}[(a)]
   \item sabendo que cada estudante escolheu uma única área, complete a tabela com os dados que estão faltando.
   \item um estudante é escolhido ao acaso. Sabendo que é do sexo feminino, determine a probabilidade de essa estudante preferir humanidades ou biológicas.
   \end{enumerate}   
     \begin{sol}
       \phantom{A} 
       \begin{enumerate} [(a)]
           \item \begin{tabular}{|c|c|c|c|} \hline
& Masc (M) & Fem (F) & Total \\  \hline
Exatas (E) & 120 & 80 & 200 \\  \hline
Humanidades (H) & 45 & 80 & 125 \\  \hline
Biológicas (B) & 100 & 75   & 175 \\  \hline
Total   & 265 & 235   &  500 \\  \hline
\end{tabular}
       \item $\frac{80+75}{235}=\frac{155}{235}=\frac{31}{47}$
       \end{enumerate}
     \end{sol}
\end{ex}