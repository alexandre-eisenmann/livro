\begin{ex}
(Uespi) A um debate entre candidatos a governador de certo estado compareceram 7 candidatos, sendo 4 homens e 3 mulheres. A organização do evento resolveu que os candidatos ficariam lado a lado, numa disposição não circular, e que os homens não ficariam juntos um do outro, e sim em posição alternada com as mulheres.	Para isso, em cada um dos 7 locais a serem ocupados pelos candidatos, foi colocado o nome do respectivo ocupante. Nessas condições, é correto afirmar que o número de maneiras diferentes de esses candidatos serem arrumados em seus respectivos locais de debate é:
   \begin{enumerate}[(a)]
   \item 121
   \item 124
   \item 136
   \item 144
   \item 169
   \end{enumerate}
    \begin{sol}
    resposta: d \\
    H M H M H M H $\rightarrow 4!\cdot3!=144$
    \end{sol}
\end{ex}