\begin{ex}
Numa enquete foram entrevistadas 80 pessoas sobre os meios de transporte que utilizavam para ir ao trabalho e/ou à escola. Quarenta e duas pessoas responderam ônibus, 28 responderam carro, 30 responderam moto, 12 utilizavam ônibus e carro, 14 carro e moto, 18 ônibus e moto e 5 utilizavam os três: carro, ônibus e moto. Qual é a probabilidade de que uma dessas pessoas escolhida ao acaso, utilize:
   \begin{enumerate}[(a)]
   \item somente carro?
   \item moto e ônibus, mas não carro?
   \item apenas um dos 3 veículos?
   \item sabendo que se escolheu uma pessoa que usa moto, qual é a probabilidade dela ser usuária de carro?
   \end{enumerate}
     \begin{sol}
       \phantom{A} \\ \\
      \begin{venndiagram3sets} [labelA=\(O\),labelB=\(C\),labelC=\(M\),labelOnlyA=17,labelOnlyB=7,labelOnlyC=3,labelNotABC=19,labelABC=5,labelOnlyAB=7,labelOnlyAC=13,labelOnlyBC=9]
      \end{venndiagram3sets}
      80-(17+7+3+7+9+13+5)=19
        \begin{enumerate} [(a)]
            \item $\frac{7}{80}$
            \item $\frac{13}{80}$
            \item $\frac{27}{80}$
            \item $\frac{14}{30}$
        \end{enumerate}
     \end{sol}
\end{ex}