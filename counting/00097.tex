\begin{ex}
  (Fuvest) Cláudia, Paulo, Rodrigo e Ana brincam entre si de amigo secreto (ou amigo-oculto). Cada nome é escrito em um pedaço de papel, que é colocado em uma urna, e cada participante retira um deles ao acaso. A probabilidade de que nenhum participante retire seu próprio nome é:
    \begin{enumerate}  [(a)]
        \item $\frac{1}{4}$
        \item $\frac{7}{24}$
        \item $\frac{1}{3}$
        \item $\frac{3}{8}$
        \item $\frac{5}{12}$
    \end{enumerate}
      \begin{sol}
       resposta: d \\
       Permutação caótica = $(PC)_n = n!\cdot(\frac{1}{2!}-\frac{1}{3!}+\frac{1}{4!}-\frac{1}{5!}....)$ \\
       4 pessoas: $\Rightarrow n!=4!=24$\hspace{0,4cm} 
       $\therefore \hspace{0,4cm} 24\cdot(\frac{1}{2}-\frac{1}{6}+\frac{1}{24})=9 \Longrightarrow p=\frac{9}{24}=\frac{3}{8}$\\
       ou contar: (PARC, CARP, RAPC, PCRA, CRAP, RCAP, PRAC, CRPA, RCPA) = 9 maneiras 
       $ \Longrightarrow p=\frac{9}{4!}=\frac{9}{24}=\frac{3}{8}$
      \end{sol}
 \end{ex}