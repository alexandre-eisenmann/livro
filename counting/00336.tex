\begin{ex}
(Enem) A capa de uma revista de grande circulação trazia a seguinte informação, relativa a uma reportagem daquela edição:
“O brasileiro diz que é feliz na cama, mas debaixo dos lençóis 47\% não sentem vontade de fazer sexo”.
O texto abaixo, no entanto, adaptado da mesma reportagem, mostra que o dado acima está errado.
“Outro problema predominantemente feminino é a falta de desejo – 35\% das mulheres não sentem nenhuma vontade de ter relações. Já entre os homens, apenas 12\% se queixam de tal desejo”.
Considerando que o número de homens na população seja igual ao de mulheres, a porcentagem aproximada de brasileiros que não sentem vontade de fazer sexo, de acordo com a reportagem, é:
   \begin{enumerate}[(a)]
   \item 12\%
   \item 24\%
   \item 29\%
   \item 35\%
   \item 50\%
   \end{enumerate}
     \begin{sol}
      resposta: b \\
      sem vontade (mulher ou homem) : $0,35\cdot0,50+0,12\cdot0,50=0,235 \approx 24\%$ 
     \end{sol}
\end{ex}