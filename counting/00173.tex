\begin{ex}
 O número obtido ao se jogar um dado é o coeficiente
\textit{b} da equação  $ x^2 + bx + 1 = 0 $. Determine:
   \begin{enumerate}[(a)]
   \item a probabilidade de essa equação ter raízes reais.
   \item a probabilidade de essa equação ter raízes reais, sabendo que ocorreu um número ímpar.
   \end{enumerate}
     \begin{sol}
      \phantom{A} 
       \begin{enumerate} [(a)]
           \item $\Delta \geq 0 \hspace{0,5cm}\therefore\hspace{0,5cm}  \text{b}=(2, 3, 4, 5, 6) \Longrightarrow p=\frac{5}{6}$
           \item 2 ímpares: 3 e 5 $\Longrightarrow p=\frac{2}{3}$ 
       \end{enumerate}
     \end{sol}
\end{ex}