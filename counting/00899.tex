\begin{ex}
 (Ufv) Considere 9 barras de metal que medem, respectivamente: 1, 2,  3, 4, 5, 6, 7, 8 e 9 metros. Quantas combinações de 5 barras, ordenadas em ordem crescente de comprimento, podem ser feitas de tal forma que a barra de 5 metros ocupe sempre a quarta posição?
    \begin{enumerate}[(a)]
    \item 32
    \item 16
    \item 20
    \item 18
    \item 120
    \end{enumerate}
      \begin{sol}
       resposta: b \\
       na quarta posição temos a  barra de 5 metros e na quinta posição podemos ter 4 opções ( barras: 6, 7, 8 ,9 ) \\
       $\frac{1}{\phantom{A}}\frac{2}{\phantom{A}}\frac{3}{\phantom{A}}\frac{5}{\phantom{A}}\frac{\phantom{A}}{\phantom{A}}\hspace{0.5cm}\frac{1}{\phantom{A}}\frac{2}{\phantom{A}}\frac{4}{\phantom{A}}\frac{5}{\phantom{A}}\frac{\phantom{A}}{\phantom{A}}\hspace{0.5cm}\frac{1}{\phantom{A}}\frac{3}{\phantom{A}}\frac{4}{\phantom{A}}\frac{5}{\phantom{A}}\frac{\phantom{A}}{\phantom{A}}\hspace{0.5cm}\frac{2}{\phantom{A}}\frac{3}{\phantom{A}}\frac{4}{\phantom{A}}\frac{5}{\phantom{A}}\frac{\phantom{A}}{\phantom{A}}\hspace{0.5cm}\Longrightarrow 4\cdot4=16$
      \end{sol}
\end{ex}