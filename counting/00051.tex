\begin{ex}
   (Ita) Lançando três dados de 6 faces, numeradas de 1 a 6, sem ver o resultado, você é informado de que a soma dos números observados na face superior de cada dado é igual a 9. Determine a probabilidade de o número observado em cada uma dessas faces ser um número ímpar. 
     \begin{sol}
     \phantom{A} \\
     possibilidades de soma 9 no lançamento de 3 dados: \\
     - (1 2 6) (1 6 2) (2 1 6) (2 6 1) (6 1 2) (6 2 1) = 6\\
     - (1 3 5) (1 5 3) (3 1 5) (3 5 1) (5 1 3) (5 3 1) = 6\\
     - (1 4 4) (4 1 4) (4 4 1) = 3 \\
     - (2 2 5) (2 5 2) (5 2 2) = 3 \\
     - (2 3 4) (2 4 3) (3 2 4) (3 4 2) (4 2 3) (4 3 2) = 6 \\
     - (3 3 3) = 6 \hspace{0,4cm} total: 25 possibilidades \\
     nº observado em cada uma das faces ser ímpar: 135, 153, 315, 351, 513, 531 e 333\\
     $\Longrightarrow p=\frac{7}{25}$
     \end{sol}
  \end{ex}