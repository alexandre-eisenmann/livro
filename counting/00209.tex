\begin{ex}
(FGV) Um viajante, partindo de A, deve chegar à cidade D, passando obrigatoriamente pelas cidades B e C. Para viajar de A para B existem 3 meios de transporte: avião, navio e trem; de B para C, 2 meios de transporte: táxi e ônibus e de C para D, 3 meios: carroça, moto e bicicleta. Quantas maneiras diferentes existem para viajar de A para D?
   \begin{enumerate}[(a)]
   \item 8
   \item 3
   \item mais de 15
   \item menos de 10
   \item nda
   \end{enumerate}
     \begin{sol}
       resposta: c \\
       $3\cdot2\cdot3=18$
     \end{sol}
\end{ex}