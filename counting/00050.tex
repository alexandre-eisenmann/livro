\begin{ex}
  (Enem) Uma empresa confecciona e comercializa um brinquedo formado por uma locomotiva, pintada na cor preta, mais 12 vagões de iguais formato e tamanho, numerados de 1 a 12. Dos 12 vagões, 4 são pintados na cor vermelha, 3 na cor azul, 3 na cor verde e 2 na cor amarela. O trem é montado utilizando-se uma locomotiva e 12 vagões, ordenados crescentemente segundo suas numerações, conforme ilustrado na figura:
    \begin{center}
        \includegraphics[width=8cm]{imagens/enem_ex_1050.png}
    \end{center}
  De acordo com as possíveis variações nas colorações dos vagões, a quantidade de trens que podem ser montados, expressa por meio de combinações, é dada por:
    \begin{enumerate}   [(a)]
        \item $C_{12}^4\times C_{12}^3\times C_{12}^3\times C_{12}^2$
        \item  $C_{12}^4+C_8^3+C_5^3+C_2^2$
        \item $C_{12}^4\times2\times C_8^3\times C_5^2$
        \item $C_{12}^4+2\times C_{12}^3+ C_{12}^2$
        \item $C_{12}^4\times C_8^3\times C_5^3\times C_2^2$
    \end{enumerate}
      \begin{sol}
      resposta: e \\
      Dos 12 vagões escolhe-se 4 para pintar de azul. Depois, escolhe-se 3 vagões dos 8 restantes para pintar de vermelho. Em seguida escolhe-se 3 dos 5 vagões para pintar de verde e os 2 últimos são pintados de amarelo. \\
      $\mathrm{C}_{{12},4}\cdot\mathrm{C}_{8,3}\cdot\mathrm{C}_{5,3}\cdot\mathrm{C}_{2,2}$
      \end{sol}
  \end{ex}