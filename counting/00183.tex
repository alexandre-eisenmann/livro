\begin{ex}
Numa urna há 10 bolas numeradas de 1 a 10. As bolas 1, 2 e 3 são brancas; 4, 5 e 6 são pretas e as demais vermelhas. Retirando ao acaso duas bolas da urna, calcule a probabilidade de que:
   \begin{enumerate}[(a)]
   \item a primeira bola seja branca e a segunda não seja branca.
   \item  a primeira bola seja vermelha e a segunda não seja vermelha.
   \item as duas bolas tenham cores diferentes uma da outra.
   \end{enumerate}
     \begin{sol}
      \phantom{A} \\
      B= branca;  P= preta;  V= vermelha
       \begin{enumerate} [(a)]
           \item $\frac{3}{10}\cdot\frac{7}{9}=\frac{7}{70}$
           \item $\frac{4}{10}\cdot\frac{6}{9}=\frac{4}{15}$
           \item B e P ou B e V ou V e P $\rightarrow (\frac{3}{10}\cdot\frac{3}{9}+\frac{3}{10}\cdot\frac{4}{9}+\frac{4}{10}\cdot\frac{3}{9})\cdot2=\frac{11}{15}$
       \end{enumerate}
     \end{sol}
\end{ex}