\begin{ex}
(Epcar-Afa) Num acampamento militar, serão instaladas três barracas: I, II e III. Nelas, serão alojados 10 soldados, dentre eles o soldado A e o soldado B, de tal maneira que fiquem 4 soldados na barraca I, 3 na barraca II e 3 na barraca III.
Se o soldado A deve ficar na barraca I e o soldado B não deve ficar na barraca III, então o número de maneiras distintas de distribuí-los é igual a:
   \begin{enumerate}[(a)]
   \item 560
   \item 1120
   \item 1680
   \item 2240
   \end{enumerate}
     \begin{sol}
      resposta: b \\
      temos duas situações: soldado A e soldado B na barraca I \textbf{ou} soldado A na barraca I e soldado B na barraca II \\
      $\mathrm{C}_{8,2}\cdot\mathrm{C}_{6,3}\cdot\mathrm{C}_{3,3}+\mathrm{C}_{8,3}\cdot\mathrm{C}_{5,2}\cdot\mathrm{C}_{3,3}=560+560=1120$
     \end{sol}
\end{ex}