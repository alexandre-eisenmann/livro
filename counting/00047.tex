\begin{ex}
   (Fatec) Admita que, na FATEC-SP, há uma turma de 40 alunos de Logística, sendo 18 rapazes; e uma turma de 36 alunos de Análise de Sistemas, sendo 24 moças. Para participar de um debate serão escolhidos aleatoriamente dois alunos, um de cada turma. Nessas condições, a probabilidade de que sejam escolhidos uma moça e um rapaz é :
      \begin{enumerate}  [(a)]
          \item $\frac{29}{60}$
          \item $\frac{47}{96}$
          \item $\frac{73}{144}$
          \item $\frac{81}{160}$
          \item $\frac{183}{360}$
      \end{enumerate}
        \begin{sol}
        resposta: a \\
        moça da logística e rapaz de sistemas ou rapaz da logística e moça de sistemas.\\ \\
         \begin{tabular}{|c|c|c|c|} \hline
        turma & moças & rapazes & total  \\  \hline
        logística & 22 & 18 & 40  \\ \hline
        sistemas & 24 & 12 & 36 \\ \hline
         \end{tabular}
         $\Longrightarrow p = \frac{22}{40}\cdot\frac{12}{36}+\frac{18}{40}\cdot\frac{24}{36}=\frac{29}{60}$
        \end{sol}
 \end{ex}