\begin{ex}
(UFES) Um homem encontra-se na origem do sistema cartesiano ortogonal de eixos Ox e Oy. Ele pode dar um passo de cada vez, para N ou para L. Se ele der exatamente 10 passos, o número de trajetórias que ele pode percorrer é:
   \begin{enumerate}[(a)]
   \item $10!$
   \item $\frac{10!}{(10-2)!}$
   \item ${10}^2$
   \item $2^{10}$
   \item $\frac{10!}{2!(10-2)!}$
   \end{enumerate}
     \begin{sol}
       resposta: d \\
       $2^{10}$
     \end{sol}
\end{ex}