\begin{ex}
 (UfAL) Desde o fim da última era glacial até hoje, a humanidade desenvolveu a agricultura, a indústria, construiu cidades e, por fim, com a advento da internet, experimentou um avanço comercial sem precedentes. Quase todos os produtos vendidos no planeta atravessam alguma fronteira antes de chegar ao consumidor. No esquema adiante, suponha que os países: A, B, C e D sejam inseridos na logística do transporte de mercadorias com o menor custo e no mesmo tempo.

\begin{center}    
\begin{tikzpicture}

\tikzset{
vertex/.style={circle,draw,minimum size=1em},
edge/.style={->,> = latex'}
}
% vertices
\node[vertex] (1) at (0,0) {$B$};
\node[vertex] (2) at (0,-4) {$A$};
\node[vertex] (3) at (2,-2) {$C$};
\node[vertex] (4) at (5,-2) {$D$};

%edges
\draw[edge] (2) -- (1) node[midway, left] {$4$};
\draw[edge] (1) -- (3) node[pos=.4, left, sloped, below] {$3$};
\draw[edge] (1.5) -- (4.145) node[midway, left, sloped, above] {$6$};
\draw[edge] (2) -- (3) node[pos=.4, left, sloped, above] {$5$};
\draw[edge] (2.350) -- (4.575) node[midway, sloped, below] {$7$};
\draw[edge] (3) -- (4) node[pos=.4, sloped, above] {$2$};

\end{tikzpicture}
\end{center}



Os números indicados representam o número de rotas distintas de transporte aéreo disponíveis, nos sentidos indicados. Por exemplo, de A  até B são 4 rotas; de C até D são 2 rotas, e assim por diante.\\
Nessas condições, o número total de rotas distintas, de A até D é igual a:
    \begin{enumerate}[(a)]
    \item 66
    \item 65
    \item 64
    \item 63
    \item 62
    \end{enumerate}
      \begin{sol}
      resposta: b  \\
      $\mathrm{AB}\rightarrow\mathrm{BC}\rightarrow\mathrm{CD}$\hspace{0.2cm}ou\hspace{0.2cm}$\mathrm{AB}\rightarrow\mathrm{BD}$\hspace{0.2cm}ou\hspace{0.2cm}$\mathrm{AC}\rightarrow\mathrm{CD}$\hspace{0.2cm}ou\hspace{0.2cm} AD \\
      $4\cdot3\cdot2+4\cdot6+5\cdot2+7=65$
      \end{sol}
\end{ex}