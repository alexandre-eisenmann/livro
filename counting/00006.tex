\begin{ex}
 (Ita) Considere uma prova com 10 questões de múltipla escolha, cada questão com 5 alternativas. Sabendo que cada questão admite uma única alternativa correta, então o número de formas possíveis para que um candidato acerte somente uma das 10 questões é:
    \begin{enumerate}[(a)]
    \item $ 4^4 \cdot 30$
    \item $ 4^3 \cdot 60$
    \item $ 5^3 \cdot 60$
    \item $\binom{7}{3}\cdot4^3$
    \item $\binom{10}{7}$
    \end{enumerate}
      \begin{sol}
        resposta: a \\
        $\mathrm{C}_{{10},7}$ é o número de possibilidades de acertar exatamente 7 testes em 10. \\
        $4\cdot4\cdot4$  é o número de maneiras de se escolher as 3 alternativas erradas. \\
        $\Longrightarrow \mathrm{C}_{{10},7}\cdot4^3=120\cdot4^3=30\cdot4^4$
      \end{sol}
\end{ex}