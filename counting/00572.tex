\begin{ex}
Num freezer de um supermercado, há somente sorvetes de chocolate ou sorvetes da marca Ice, num total de 60 unidades, sendo 40 de chocolate. Retirando-se um sorvete aleatoriamente desse freezer, a probabilidade de que ele seja de chocolate e da marca Ice é de $\frac{1}{5}$. Determine o número de sorvetes da marca Ice nesse freezer.
 \begin{sol}
  \phantom{A} \\
  p(chocolate)=$\frac{40}{60}=\frac{2}{3}$ \hspace{0,7cm}
  p(chocolate e marca Ice) = $\frac{\frac{1}{5}}{\frac{2}{3}}=\frac{3}{10}$ \\
  sorvete (chocolate) marca Ice:  \hspace{0,6cm} $\frac{3}{10}\cdot40=12 $\\ sorvete Ice (não de chocolate)\hspace{0,6cm} $60-40=20$  \\
  $\Longrightarrow 20+12=32$
 \end{sol}
\end{ex}