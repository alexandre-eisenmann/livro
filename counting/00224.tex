\begin{ex}
 (Fuvest) Numa urna há:
   \begin{itemize}
   \item uma bola numerada com o número 1
   \item 2 bolas com o número 2
   \item 3 bolas com o número 3, e assim por diante, até \textit{n} bolas com o número \textit{n}.
   \end{itemize}
Uma bola é retirada ao acaso dessa urna. Admitindo-se que todas as bolas tem a mesma probabilidade de serem escolhidas, qual é , em função de \textit{n}, a probabilidade de que o número da bola retirada seja par ?
 \begin{sol}
   \phantom{A} \\
   número de bolas na urna: $1+2+3+4+...+n$. Temos uma PA cuja soma é: $\frac{(n+1).n}{2}$\\
   com n par: $2+4+6+8+...+n$. Temos outra PA com soma: $\frac{(2+n).n}{4}$ \\
   $p=\frac{\frac{(2+n).n}{4}}{\frac{(n+1).n}{2}}=\frac{2+n}{2(1+n)}$
   
 \end{sol}
\end{ex}