\begin{ex}
  (Ita) São dadas duas caixas, uma delas contém três bolas brancas e duas pretas e a outra contém duas bolas brancas e uma preta. Retira-se, ao acaso, uma bola de cada caixa. Se $P_1$ é a probabilidade de que pelo menos uma bola seja preta e $P_2$ a probabilidade de as duas bolas serem da mesma cor, então $P_1 +P_2$ vale: 
    \begin{enumerate} [(a)]
        \item $\frac{8}{15}$
        \item $\frac{7}{15}$
        \item $\frac{6}{15}$
        \item 1
        \item $\frac{17}{15}$
    \end{enumerate}
      \begin{sol}
       resposta: e \\
       caixa 1: 3 brancas e 2 pretas \hspace{0,5cm} caixa 2 : 2 brancas e 1 preta \\
    $P_1$ = branca cx 1 e preta cx 2 ou preta cx 1 e branca cx 2 ou preta cx 1 e preta cx 2\\
    $P_1= \frac{3}{5}\cdot\frac{1}{3}+\frac{2}{5}\cdot\frac{2}{3}+\frac{2}{5}\cdot\frac{1}{3}=\frac{9}{15}$\\
    $P_2$ = branca cx 1 e preta cx 2 ou preta cx 1 e preta cx 2 \\
    $P_2=\frac{3}{5}\cdot\frac{2}{3}+\frac{2}{5}\cdot\frac{1}{3}=\frac{8}{15}$ \hspace{0,6cm}
    $\Longrightarrow P_1+P_2= \frac{9}{15}+\frac{8}{15}=\frac{17}{15}$
    
      \end{sol}
 \end{ex}