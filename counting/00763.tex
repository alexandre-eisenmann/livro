\begin{ex}
 (Unifesp) Três dados honestos são lançados. A probabilidade de que os três números sorteados possam ser posicionados para formar progressões aritméticas  de razão 1 ou 2 é:
    \begin{enumerate}[(a)]
    \item $\frac{1}{36}$
    \item $\frac{1}{9}$
    \item $\frac{1}{6}$
    \item $\frac{7}{36}$
    \item $\frac{5}{18}$
    \end{enumerate}
      \begin{sol}
       resposta: c  \\
       possíveis progressões aritméticas de razão 1 ou 2 com números obtidos no lançamentos de 3 dados honestos: (1,2,3) (2,3,4) (3,4,5) (4,5,6) (1,3,5) e (2,4,6)  \\
       probabilidade de ocorrer cada uma delas: $\frac{1}{6}\cdot\frac{1}{6}\cdot\frac{1}{6}\cdot3!=\frac{1}{36}\Longrightarrow p=6.\frac{1}{36}=\frac{1}{6}$
      \end{sol}
\end{ex}