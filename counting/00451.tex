\begin{ex}
(Uff) Seiscentos estudantes de uma escola foram entrevistados sobre suas preferências quanto aos esportes vôlei e futebol. O resultado foi o seguinte: 204 estudantes gostam somente de futebol; 252 gostam somente de vôlei e 48 disseram que não gostam de nenhum dos dois esportes.
   \begin{enumerate}[(a)]
   \item determine o número de estudantes entrevistados que gostam dos dois esportes;
   \item  um dos estudantes entrevistados é escolhido ao acaso. Qual é a probabilidade de que ele goste de vôlei? 
   \end{enumerate}
    \begin{sol}
     \phantom{A} \\ \\
     \begin{venndiagram2sets} [labelA=\(F\),labelB=\(V\),labelOnlyA=204,labelOnlyB=252,labelNotAB=48,labelAB=\textit{x}]
     \end{venndiagram2sets}
       \begin{enumerate} [(a)]
           \item $204+x+252+48=600 \rightarrow x=96$
           \item $\frac{252+96}{600}=\frac{58}{100}=58\%$
       \end{enumerate}
    \end{sol}
\end{ex}