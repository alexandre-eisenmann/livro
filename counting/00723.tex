\begin{ex}
 (Fgv-SP) Uma fatia de pão com manteiga pode cair no chão de duas maneiras apenas:
    \begin{itemize}
    \item[--] com a manteiga para cima (evento A)
    \item[--] com a manteiga para baixo (evento B)
    \end{itemize}
Uma possível distribuição de probabilidade para esses eventos é:
    \begin{enumerate}[(a)]
    \item P(A) = P(B) = $\frac{3}{7}$
    \item P(A) = 0 e P(B) = $\frac{5}{7}$
    \item P(A) = - 0,3 e P(B) = 1,3
    \item P(A) = 0,4 e P(B) = 0,6
    \item P(A) = $\frac{6}{7}$ e P(B) = 0
    \end{enumerate}
     \begin{sol}
      resposta: d 
       \begin{enumerate} [(a)]
           \item $\frac{3}{7}+\frac{3}{7}=\frac{6}{7}$ falso
           \item $0+\frac{5}{7}=\frac{5}{7}$ falso
           \item P(A) $<0$ impossível
           \item $0,4 +0,6 =1$ \textbf{verdadeiro}
           \item $\frac{6}{7}+0=\frac{6}{7}$ falso
       \end{enumerate}
     \end{sol}
\end{ex}