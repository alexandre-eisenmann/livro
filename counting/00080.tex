\begin{ex}
 (Fuvest) Maria deve criar uma senha de 4 dígitos para sua conta bancária. Nessa senha, somente os algarismos 1, 2, 3, 4, 5, podem ser usados e um mesmo algarismo podem aparecer mais de uma vez. Contudo, supersticiosa, Maria não quer que sua senha contenha o número 13, isto é, o algarismo 1 seguido imediatamente pelo algarismo 3. De quantas maneiras distintas Maria pode escolher sua senha? 
    \begin{enumerate} [(a)]
        \item 551
        \item 552
        \item 553
        \item 554
        \item 555
    \end{enumerate}
     \begin{sol}
      resposta: a \\
      - total de senhas: $5\cdot5\cdot5\cdot5=625$ \\
      - senhas em que aparece o "13":\hspace{0,4cm} $5\cdot5\cdot3=75$\\
      - a senha 1313 foi contada 2 vezes, então tira-se 1 vez, logo o número de senhas possíveis é:\hspace{0,3cm}
      $625 - 74 =551$
     \end{sol}
 \end{ex}