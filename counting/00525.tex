\begin{ex}
Trinta pacientes hipertensos submeteram-se a um teste de esforço. Ao final do teste, o médico assinalou ao lado do nome de cada um, em uma lista, a letra S ou a letra D ou a sequência SD, conforme o paciente tenha apresentado variação acentuada na pressão sistólica ou diastólica ou nas duas, respectivamente. Sabendo que todos os pacientes tiveram uma dessas classificações e que foram assinaladas 18 letras S e 20 letras D, quantos pacientes tiveram a classificação SD?
  \begin{sol}
  \phantom{A} \\
  $A\cup B=A+B-(A\cap B) \rightarrow 30=18+20-x \Longrightarrow x=8 \therefore SD=8$
  \end{sol}
\end{ex}