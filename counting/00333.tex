\begin{ex}
(Enem) Um apostador tem 3 opções para participar de certa modalidade de jogo, que consiste no sorteio aleatório de um número dentre dez.
   \begin{itemize}
   \item 	1ª opção: comprar 3 números para um único sorteio;
   \item 	2ª opção: comprar 2 números para um sorteio e um número para um segundo sorteio;
   \item 	3ª opção: comprar um número para cada sorteio no total de 3 sorteios.
   \end{itemize}
Se $X, Y, Z$ representam as possibilidades de o apostador ganhar algum prêmio, escolhendo respectivamente a 1ª, a 2ª ou 3ª opção, é correto afirmar que:
   \begin{enumerate}[(a)]
   \item $ X < Y < Z $
   \item $ X = Y = Z $
   \item $ X > Y = Z $
   \item $ X = Y + Z $
   \item $ X > Y > Z $
   \end{enumerate}
    \begin{sol}
     resposta: e \\
     1ª opção: um sorteio $\Longrightarrow$  
     p(ganhar)(X) = $\frac{3}{10}$ \hspace{0,5cm} p(não ganhar)(X)= $\frac{7}{10}$ \\
     2ª opção: 2 sorteios $\Longrightarrow $p(não ganhar)(Y)= $\frac{8}{10}\cdot\frac{9}{10}=0,72$ \hspace{0,5cm} p(ganhar)(Y)=0,28 \\
     3ª opção: 3 sorteios $\Longrightarrow $ 
     p(não ganhar)(Z)=$(\frac{9}{10})^3$\hspace{0,5cm} p(ganhar) Z=$1-(\frac{9}{10})^3=0,271$ \\
     $\mathrm{X} > \mathrm{Y} > \mathrm{Z}$
    \end{sol}
\end{ex}