\begin{ex}
 (UFRJ) Um sítio da internet gera uma senha de 6 caracteres para cada usuário, alternando letras e algarismos. A senha é gerada  com as seguintes regras:
    \begin{itemize}
    \item[--] não há repetição de caracteres;
    \item[--] começar sempre por uma letra;
    \item[--] o algarismo que segue uma vogal corresponde a um número primo;
    \item[--] o algarismo que segue uma consoante corresponde a um número par.
    \end{itemize}
Quantas senhas podem ser geradas de forma que as três letras sejam A, M e R em qualquer ordem?
   \begin{sol}
      \phantom{A}  \\
    -número de senhas com número primo não par após a primeira letra:\\
     $\frac{\mathrm{A}}{\phantom{A}}\frac{\phantom{A}}{3}\frac{\mathrm{M}}{\phantom{A}}\frac{\phantom{A}}{5}\frac{\mathrm{R}}{\phantom{A}}\frac{\phantom{A}}{4}=3\cdot3\cdot5\cdot4\cdot3!=360$ \\
   - número de senhas como número primo par 2 após a primeira letra: \\
    $\frac{\mathrm{A}}{\phantom{A}}\frac{\phantom{A}}{1}\frac{\mathrm{M}}{\phantom{A}}\frac{\phantom{A}}{4}\frac{\mathrm{R}}{\phantom{A}}\frac{\phantom{A}}{3}=1\cdot4\cdot3\cdot3!=72$\\
   $ \Longrightarrow 360+72=432$
  
   \end{sol}
\end{ex}