\begin{ex}
 (Puc-MG) Em um código binário, utilizam-se dois símbolos: o algarismo 0 (zero) e o algarismo 1 (um). Considerando esses símbolos como letras, são formadas palavras. Assim, por exemplo as palavras 0, 10 e 111 tem respectivamente , uma, duas ou três letras. O número máximo de palavras, com até 6 letras que podem ser formadas com esse código, é:
    \begin{enumerate}[(a)]
    \item 42
    \item 62
    \item 86
    \item 126
    \end{enumerate}
      \begin{sol} 
      resposta: d  \\
      $2^6+2^5+2^4+2^3+2^2+2=126$
        
      \end{sol}
\end{ex}