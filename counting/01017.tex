\begin{ex}
 	(Unesp)  Jogando três dados de tamanhos diferentes, a probabilidade  de dar números que correspondem, em grandeza, ao tamanho dos dados, ou seja, o número maior que ocorre deve estar no dado maior, o médio no médio e o menor no menor, é:
    \begin{enumerate}[(a)]
    \item $\frac{25}{216}$
    \item $\frac{5}{54}$
    \item $\frac{19}{216}$
    \item $\frac{1}{6}$
    \item $\frac{1}{3}$
    \end{enumerate}
       \begin{sol}
         resposta: b \\
         total: $6^3 = 216$ possibilidades\\
         dado menor: (1,2,3) (1,2,4) (1,2,5) (1,2,6) (1,3,4) (1,3,5) (1,3,6) (1,4,5) (1,4,6) (1,5,6) = 10 \\
         dado médio: (2,3,4) (2,3,5) (2,3,6) (3,4,5) (3,4,6) (4,5,6) = 6 \\
         dado maior: (3,4,5) (3,4,6) (3,5,6) (4,5,6)= 4 \\
         $\Longrightarrow p= \frac{10+6+4}{216}=\frac{5}{54}$
       \end{sol}
\end{ex}