\begin{ex}
 	Existem  duas urnas. A primeira com quatro bolas numeradas de 1 a 4 e a segunda com três bolas numeradas de 7 a 9. Duas bolas são extraídas da primeira urna, sucessivamente e sem reposição, e em seguida duas bolas são extraídas  da segunda urna, sucessivamente e sem reposição. Quantos números de quatro algarismos são possíveis serem formados nestas condições?
 	   \begin{sol}
 	     \phantom{A} \\
 	     $\frac{\phantom{A}}{4}\frac{\phantom{A}}{3}\frac{\phantom{A}}{3}\frac{\phantom{A}}{2}= 72$
 	   \end{sol}
\end{ex}