\begin{ex}
 (Unirio) Um grupo de 9 pessoas, dentre elas os irmãos João e Pedro, foi acampar. Na  hora de dormir montaram 3 barracas diferentes, sendo que, na primeira, dormiram duas pessoas, na segunda 3 pessoas e na terceira as 4 restantes. De quantos modos diferentes eles podem se organizar, sabendo que a única restrição é a de que os irmãos João e Pedro não podem dormir na mesma barraca?
    \begin{enumerate}[(a)]
    \item 1260
    \item 1225
    \item 1155
    \item 1050
    \item 910
    \end{enumerate}
      \begin{sol}
       resposta: e \\
       Total - João e Pedro juntos.
       Total= $\mathrm{C}_{9,2}\cdot\mathrm{C}_{7,3}\cdot\mathrm{C}_{4,4}= 1260$ \\
       João e Pedro na primeira barraca: $1\cdot\mathrm{C}_{7,3}\cdot\mathrm{C}_{4,4}=35$\\
       Joâo e Pedro na segunda barraca: $\mathrm{C}_{7,1}\cdot\mathrm{C}_{6,2}\cdot\mathrm{C}_{4,4}=105$ \\
       João e Pedro na terceira barraca: $\mathrm{C}_{2,2}\cdot\mathrm{C}_{7,2}\cdot\mathrm{C}_{5,3}\cdot\mathrm{C}_{2,2}= 210$ \\
       $\Longrightarrow 1260 - (35 +105+210)=910$
      \end{sol}
\end{ex}