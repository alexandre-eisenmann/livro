\begin{ex}
(Puc – RJ) No jogo denominado “zerinho – ou – um”, cada uma das três pessoas indica ao mesmo tempo com a mão uma escolha de 0 (mão fechada) ou 1 ( o indicador apontado), e ganha a pessoa que escolher a opção que diverge da maioria. Se as 3 pessoas escolheram a mesma opção, faz-se então uma nova tentativa. Qual é a probabilidade de não haver um ganhador definido depois de 3 rodadas?
  \begin{sol}
   \phantom{A} \\
   não haver ganhador: números iguais nas 3 rodadas
   $\rightarrow (\frac{2}{2}\cdot\frac{1}{2}\cdot\frac{1}{2})^3=\frac{1}{64}$
  \end{sol}
\end{ex}