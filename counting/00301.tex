\begin{ex}
(UMC –SP) A tabela a seguir fornece, por sexo e área escolhida, o número de inscritos em um vestibular para ingresso no curso superior.
\begin{center}
\begin{tabular}{|c|c|c|c|} \hline
& Biomédicas & Exatas & Humanas \\  \hline 
Masc.  & 2500 & 1500 & 1500  \\  \hline
Fem.   & 1500 & 1000 & 2000  \\ \hline
\end{tabular}
\end{center}

Escolhidos ao acaso, um dos inscritos  e representada por $p_1$ a probabilidade de o escolhido ser do sexo masculino e ter optado por Exatas e por $p_2$ a probabilidade de o escolhido ser do sexo feminino sabendo que optou por Biomédicas, pode-se concluir que:
   \begin{enumerate}[(a)]
   \item $p_1=0,6$ e $p_2=0,375$
   \item $p_1=0,6$ e $p_2=0,15$
   \item $p_1=0,15$ e $p_2=0,15$
   \item $p_1=0,15$ e $p_2=0,375$
   \item $p_1=0,375$ e $p_2=0,15$
   \end{enumerate}
    \begin{sol}
      resposta: d \\
      total de inscritos = 10000 \hspace{0,3cm}e \hspace{0,3cm}total de biomédicas = 4000 \\
      $\Longrightarrow p_1=\frac{1500}{10000}=0,15$ \hspace{0,3cm}e \hspace{0,3cm}$p_2=\frac{1500}{4000}=0,375$
    \end{sol}
\end{ex}