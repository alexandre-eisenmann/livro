\begin{ex}
 (UFJF) Um casal planeja ter exatamente três crianças. A probabilidade de que pelo menos uma criança seja menino é de :
    \begin{enumerate}[(a)]
    \item $\frac{36}{50}$
    \item $\frac{3}{4}$
    \item $\frac{7}{8}$
    \item $\frac{5}{8}$
    \item $\frac{1}{4}$
    \end{enumerate}
      \begin{sol}
        resposta: c \\
        1 menino e 2 meninas ou 2 meninos e 1 menina ou 3 meninos \\
        $\frac{1}{2}\cdot(\frac{1}{2})^2\cdot\mathrm{C}_{3,1}+(\frac{1}{2})^2\cdot\frac{1}{2}\cdot\mathrm{C}_{3,2}+(\frac{1}{2})^3=\frac{7}{8}$   \\
        ou (1 - só meninas)= $1-(\frac{1}{2})^3=\frac{7}{8}$
        
      \end{sol}
\end{ex}