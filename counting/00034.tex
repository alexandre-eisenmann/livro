\begin{ex}
   (Ita) As faces de dez moedas são numeradas de modo que: a primeira moeda tem faces 1 e 2; a segunda, 2 e 3; a terceira, 3 e 4, e assim sucessivamente até a décima moeda, com faces 10 e 11. As dez moedas são lançadas aleatoriamente e os números exibidos são somados. Então, a probabilidade de que essa soma seja igual a 60 é:
     \begin{enumerate} [(a)]
         \item $\frac{63}{128}$
         \item $\frac{63}{256}$
         \item $\frac{63}{512}$
         \item $\frac{189}{512}$
         \item $\frac{189}{1024}$
     \end{enumerate}
       \begin{sol}
       resposta: b \\
       menor soma das faces das 10 moedas: $1+2+3+4+5+6+7+8+9+10=55$\\
       ao "virar" uma das moedas da menor soma obtém-se a soma 56 , logo $\mathrm{C}_{{10},1}= 10$ vezes acontece a soma 56 \\
       ao "virar" duas das moedas da menor soma obtém-se a soma 57, 
       logo $\mathrm{C}_{{10},2}= 45$ vezes acontece a soma 57 \\
       ao "virar" três das moedas da menor soma obtém-se a soma 58,  logo $\mathrm{C}_{{10},3}= 45$ \\
       ao "virar" cinco das moedas da menor soma obtém-se a soma 60,  logo $\mathrm{C}_{{10},5}= 252$ \\
       espaço Amostral= número total de resultados possíveis: $\mathrm{C}_{{10},0}+\mathrm{C}_{{10},1}+\mathrm{C}_{{10},2}+\mathrm{C}_{{10},3}+....+\mathrm{C}_{{10},10}= 2^{10}=1024$
       $\Longrightarrow p=\frac{\mathrm{C}_{{10},5}}{2^{10}}=\frac{252}{1024}=\frac{63}{256}$
       \end{sol}
  \end{ex}