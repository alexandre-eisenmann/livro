\begin{ex}
Numa enquete foram entrevistados 100 estudantes: 70 deles responderam que frequentavam um curso de microcomputadores; 28 que frequentavam um curso de inglês e 10 que frequentavam os dois cursos. Qual é a probabilidade de um desses estudantes selecionados ao acaso:
   \begin{enumerate}[(a)]
   \item estar frequentando somente o curso de microcomputadores?
   \item não estar frequentando nenhum desses cursos?
   \end{enumerate}
    \begin{sol}
      \phantom{A} \\
       \begin{venndiagram2sets} [labelA=\(MC\),labelB=\(I\),labelOnlyA=60,labelOnlyB=18,labelAB=10] \\
        \end{venndiagram2sets}
        \begin{enumerate} [(a)]
            \item $\frac{60}{100}=60\%$
            \item $70+18=88\rightarrow100-88=12\%$
        \end{enumerate}
    \end{sol}
\end{ex}