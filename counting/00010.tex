\begin{ex}
   (Enem) Uma caixa contém uma cédula de R\$ 5,00, uma de R\$ 20,00 e duas de R\$ 50,00 de modelos diferentes. Retira-se aleatoriamente uma cédula dessa caixa, anota-se o seu valor e devolve-se a cédula à caixa. Em seguida, repete-se o procedimento anterior. A probabilidade de que a soma dos valores anotado seja pelo menos igual a R\$ 55,00 é:
     \begin{enumerate} [(a)]
         \item $\frac{1}{2}$
         \item $\frac{1}{4}$
         \item $\frac{3}{4}$
         \item $\frac{2}{9}$
         \item $\frac{5}{9}$
     \end{enumerate}
       \begin{sol}
        resposta: c \\ 
        montamos um quadro :  \\
        \begin{tabular}{c|c|c|c|c} \hline 
         &  5 & 20 & 50 &  50 \\  \hline
       5 & 10 & 25 & 55 &  55  \\  \hline
      20 & 25 & 40 & 70 &  70  \\  \hline
      50 & 55 & 70 &100 & 100  \\  \hline
      50 & 55 & 70 &100 & 100  \\  \hline
        \end{tabular}  
       total de somas encontradas = 16  \\
       12 somas iguais ou maiores que 55 \hspace{0,4cm}
       $\Longrightarrow  p = \frac{12}{16}=\frac{3}{4}$
       
       \end{sol}
  \end{ex}