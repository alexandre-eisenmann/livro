\begin{ex}
 	(Ufscar)Em uma comissão formada por 24 deputados e deputadas federais, 16 votaram a favor do encaminhamento de um projeto ao Congresso e 8 votaram contra. Do total de membros da comissão, 25\% são mulheres, e todas elas votaram a favor do encaminhamento do projeto.
    \begin{enumerate}[(a)]
    \item sorteando-se um homem da comissão, qual a probabilidade dele ser um dos que votou contra o encaminhamento do projeto?
    \item se um jornalista sortear aleatoriamente para uma entrevista 6 membros da comissão, qual é a probabilidade de que exatamente 4 dos sorteados tenham votado contra o encaminhamento do projeto ao Congresso?
    \end{enumerate}
      \begin{sol}
        \phantom{A}  \\  \\
         \begin{tabular}{c|c|c|c}  \hline
              & mulher & homem & total \\ \hline
            a favor & 6 & 10 & 16  \\  \hline
            contra & 0 & 8 & 8 \\ \hline
            total & 6 & 18 & 24  \\  \hline
         \end{tabular}
            \begin{enumerate}  [(a)]
                \item $\frac{8}{18}=\frac{4}{9}$
                \item 4 votaram contra e 2 a favor: \\
                $\frac{8}{24}\cdot\frac{7}{23}\cdot\frac{6}{22}\cdot\frac{5}{21}\cdot\frac{16}{20}\cdot\frac{15}{19}\cdot\mathrm{C}_{6,4}=\frac{300}{4807}$
            \end{enumerate}
      \end{sol}
\end{ex}