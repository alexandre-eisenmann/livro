\begin{ex}
(Vunesp) Numa comunidade formada de 1000 pessoas, foi feito um teste para detectar a presença de uma doença. Como o teste não é totalmente eficaz  existem pessoas doentes cujo resultado do teste foi negativo e existem pessoas saudáveis com o resultado do teste positivo. Sabe-se que 200 pessoas da comunidade são portadoras dessa doença. Essa informação e alguns dos dados obtidos com o teste foram colocados na tabela:\\
\begin{center}
\begin{tabular} {|c|c|c|c|}  \hline
   & Positivo (P) & Negativo (N) & Total \\  \hline
Saudável (S) &   80 &    &    800 \\  \hline
Doente (D)   &      &  40   &  200 \\  \hline
Total  &     &    & 1000 \\ \hline
\end{tabular}
\end{center}
   \begin{enumerate}[(a)]
   \item  complete a tabela com os dados que estão faltando.
   \item  uma pessoa da comunidade é escolhida ao acaso e verifica-se que o resultado do teste foi positivo. Determine a probabilidade de essa pessoa ser saudável.
   \end{enumerate}
    \begin{sol}
     \phantom{A}
       \begin{enumerate} [(a)]
       \item
       \begin{tabular} {|c|c|c|c|}  \hline
   & Positivo (P) & Negativo (N) & Total \\  \hline
Saudável (S) &   80 & 720   &    800 \\  \hline
Doente (D)   &  160    &  40   &  200 \\  \hline
Total  & 240 &  760  & 1000 \\ \hline
\end{tabular}
           \item $p=\frac{80}{240}=\frac{1}{3}$
       \end{enumerate}
    \end{sol}
\end{ex}