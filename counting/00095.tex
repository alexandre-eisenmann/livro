\begin{ex}
  (Unicamp) Lançando-se determinada moeda tendenciosa, a probabilidade de sair cara é o dobro da probabilidade de sair coroa. Em dois lançamentos dessa moeda, a probabilidade de sair o mesmo resultado é igual a:
    \begin{enumerate}   [(a)]
        \item $\frac{1}{2}$
        \item $\frac{5}{9}$
        \item $\frac{2}{3}$
        \item $\frac{3}{5}$
    \end{enumerate}
     \begin{sol}
      resposta: b \\
     \textit{x}=coroa;\hspace{0,2cm} \textit{2x}=cara \hspace{0,4cm}
       $\Longrightarrow x+2x=1  \therefore x=\frac{1}{3}$ \\
       $ p \text{(cara)} =\frac{2}{3} $\hspace{0,4cm} $p\text{(coroa)}=\frac{1}{3}$ \\
       2 resultados iguais no lançamento desta moeda duas vezes é: $\frac{1}{3}\cdot\frac{1}{3}+\frac{2}{3}\cdot\frac{2}{3}=\frac{5}{9}$
       \end{sol}
 \end{ex}