\begin{ex}
 A probabilidade de uma casa estar trancada é $\frac{5}{7}$. Uma pessoa escolhe ao acaso uma das chaves de um molho de 5 chaves das quais só uma abre a porta da casa. A probabilidade de a pessoa conseguir entrar na casa é:
    \begin{enumerate}[(a)]
    \item $\frac{3}{7}$
    \item $\frac{1}{7}$
    \item $\frac{1}{5}$
    \item $\frac{26}{35}$
    \item $\frac{2}{7}$
    \end{enumerate}
      \begin{sol}
      resposta: a\\
        p(casa trancada)=$\frac{5}{7}$\hspace{1cm}p(casa não trancada)=$\frac{2}{7}$\\
        p(chave certa)=$\frac{1}{5}$\hspace{1.45cm}p(chave errada)=$\frac{4}{5}$
          \begin{enumerate} [--]
              \item usando a chave certa: $\frac{1}{5}\cdot\frac{5}{7}+\frac{2}{7}=\frac{3}{7}$\hspace{0.7cm}      ou
              \item usando a chave errada:  $\frac{4}{5}\cdot\frac{5}{7}=\frac{4}{7}\longrightarrow 1-\frac{4}{7}=\frac{3}{7}$
          \end{enumerate}
      \end{sol}
\end{ex}