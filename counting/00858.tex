\begin{ex}
 Os pais da secretária de Renata geraram 12 filhos. Renata contou essa história na aula e ninguém acreditou muito, pois ela disse que dos 12 filhos, apenas 2 eram do sexo feminino. Carolina, que chegara atrasada à aula, afirmou que era mais fácil jogar um dado 5 vezes e sair 4 vezes a face “6” do que acontecer o que Renata contou. A polêmica foi formada e cabe a você decidir. Carolina estava certa? Para justificar sua resposta você deve mostrar com clareza os cálculos pertinentes.
   \begin{sol}
    \phantom{A} \\
    $(\frac{1}{2})^{10}\cdot(\frac{1}{2})^2\cdot\mathrm{C}_{{12},2}=\frac{66}{2^{12}}=\frac{33}{2^{11}} \approx 0,016$ \\
    $(\frac{1}{6})^4\cdot(\frac{5}{6})^1\mathrm{C}_{5,1}=(\frac{2}{6})^5=\frac{1}{3^5}\approx0,004$\hspace{0,2cm}logo, Carolina está errada.
   \end{sol}
\end{ex}