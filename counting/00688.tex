\begin{ex}
 Em uma sala de aula estavam presentes 12 alunos, sendo 7 meninas e 5 meninos, quando a professora falou:
- Pessoal: vamos formar grupos de 3 pessoas por grupo.
    \begin{enumerate}[(a)]
    \item quantos grupos diferentes poderão ser formados se Arlindo não quiser, de modo algum, ficar num grupo junto com Lucinda ou junto com Marilda?
    \item se os grupos forem formados ao acaso, qual a probabilidade de que Arlindo acabe formando grupo com Lucinda ou com Marilda?
    \end{enumerate}
      \begin{sol}
          \phantom{A} 
        \begin{enumerate} [(a)]
            \item (todos os grupos possíveis) - (grupos em que aparecem A,L,M) \\
            $\mathrm{C}_{{12},3}-(9\cdot3!\cdot2+3!)=106$
            
            \item $\frac{19\cdot3!}{\mathrm{C}_{{12},3}}=\frac{57}{110}$
        \end{enumerate}
      \end{sol}
\end{ex}