\begin{ex}
 (UFMG) Leandro e Heloísa participam de um jogo em que se utilizam dois cubos. Algumas faces desses cubos são brancas e as demais, pretas. O jogo consiste em lançar, simultaneamente, os dois cubos e em observar as faces superiores de cada um deles quando param:\\
- se as faces superiores forem da mesma cor, Leandro vencerá; e\\
- se as faces superiores forem de cores diferentes, Heloísa vencerá.\\
Sabe-se que  um dos cubos possui 5 faces brancas e uma preta e que a probabilidade de Leandro vencer o jogo é de $\frac{11}{18}$.\\
Então é CORRETO afirmar que o outro cubo tem:
    \begin{enumerate}[(a)]
    \item 4 faces brancas
    \item 1 face branca
    \item 2 faces brancas
    \item 3 faces brancas
    \end{enumerate}
      \begin{sol}
       resposta: a  \\
       Leandro vence se tirar 2 faces brancas ou 2 faces pretas: B e B ou P e P\\ $\frac{5}{6}\cdot\frac{x}{6}+\frac{1}{6}\cdot\frac{(6-x)}{6}=\frac{11}{18}\Longrightarrow x=4$
      \end{sol}
\end{ex}