\begin{ex}
(Ufal) Considere o conjunto A, formado pelos algarismos de 0 a 9, e analise as afirmações que seguem, colocando V ou F.
   \begin{enumerate}[(a)]
   \item Com os elementos de A, é possível escrever 32542 números de 5 algarismos distintos entre si.
   \item De todos os números com 4 algarismos distintos entre si que podem ser escritos com os elementos de A, 3120 são pares.
   \item De todos os números de 3 algarismos distintos entre si que podem ser escritos com os elementos de A, 176 são menores que 350.
   \item Com os algarismos de valor ímpar de A, é possível escrever exatamente 60 números de 3 algarismos distintos entre si. 
   \item De todos os números de 3 algarismos distintos entre si que podem ser escritos com os elementos de A, 150 são divisíveis por 5.
   \end{enumerate}
     \begin{sol}
      \phantom{A} 
      \begin{enumerate} [(a)]
          \item$ \frac{\phantom{A}}{9}\frac{\phantom{A}}{9}\frac{\phantom{A}}{8}\frac{\phantom{A}}{7}\frac{\phantom{A}}{6}= 27216 \rightarrow F$
          \item $\frac{\phantom{A}}{9}\frac{\phantom{A}}{8}\frac{\phantom{A}}{7}\frac{0}{1}+\frac{\phantom{A}}{8}\frac{\phantom{A}}{8}\frac{\phantom{A}}{7}\frac{\phantom{A}}{4}= 2296 \rightarrow F$
          \item $\frac{1}{1}\frac{\phantom{A}}{9}\frac{\phantom{A}}{8}+\frac{2}{1}\frac{\phantom{A}}{9}\frac{\phantom{A}}{8}+\frac{3}{1}\frac{\phantom{A}}{4}\frac{\phantom{A}}{8} = 176 \rightarrow V$
          \item $\frac{\phantom{A}}{5}\frac{\phantom{A}}{4}\frac{\phantom{A}}{3} = 60 \rightarrow V$
          \item $\frac{\phantom{A}}{9}\frac{\phantom{A}}{8}\frac{0}{1}+\frac{\phantom{A}}{8}\frac{\phantom{A}}{8}\frac{5}{1}= 156  \rightarrow F$
      \end{enumerate}
     \end{sol}
\end{ex}