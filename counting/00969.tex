\begin{ex}
 Uma urna contém 9 bolas azuis e 6 brancas. Sorteiam-se, sucessivamente e sem reposição, 4 bolas. Qual a probabilidade de sair:
    \begin{enumerate}[(a)]
    \item branca, azul, branca, nesta ordem?
    \item cores alternadas?
    \item 2 brancas e 2 azuis, não necessariamente nesta ordem?
    \item pelo menos uma branca?
    \end{enumerate}	
      \begin{sol}
      \phantom{A}
        \begin{enumerate}  [(a)]
            \item $\frac{6}{15}\cdot\frac{9}{14}\cdot\frac{5}{13}\cdot\frac{8}{12}=\frac{6}{91}$
            \item
            $\frac{6}{15}\cdot\frac{9}{14}\cdot\frac{5}{13}\cdot\frac{8}{12}\cdot2=\frac{12}{91}$
            \item
            $\frac{9}{15}\cdot\frac{8}{14}\cdot\frac{6}{13}\cdot\frac{5}{12}\cdot\mathrm{C}_{4,2}=\frac{36}{91}$
            \item total - nenhuma branca \\
            $1-\frac{9}{15}\cdot\frac{8}{14}\cdot\frac{7}{13}\cdot\frac{6}{12}=\frac{59}{65}$
        \end{enumerate}
      \end{sol}
\end{ex}