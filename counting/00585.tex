\begin{ex}
(Ufpe) Num programa de televisão, existem duas urnas, A e B, contendo bolas destinadas a um sorteio de brindes. Na urna A, existem 10 bolas amarelas e 2 azuis, e na urna B, 9 bolas amarelas e 6 azuis. Um participante é convidado a retirar uma bola de cada urna, sabendo que será premiado caso retire bolas da mesma cor. Qual é a probabilidade de esse participante ser premiado? 
  \begin{sol}
   \phantom{A}\\
   amarela(A) e amarela(B) ou azul(A) e azul(B) $\Rightarrow \frac{10}{12}\cdot\frac{9}{15}+\frac{2}{12}\cdot\frac{6}{15}=\frac{17}{30}$
  \end{sol}
\end{ex}