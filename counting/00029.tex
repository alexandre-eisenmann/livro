\begin{ex}
   (Enem) Uma loja acompanhou o número de compradores de dois produtos, A e B, durante os meses de janeiro, fevereiro e março de 2012. Com isso, obteve este gráfico.
     \begin{center}
       \includegraphics[width=8cm]{imagens/enem_ex_1029.png}
     \end{center}
   A loja sorteará um brinde entre os compadores do produto A e outro brinde entre os compradores do produto B. Qual a probabilidade de que os dois sorteados tenham feito suas compras em fevereiro de 2012?
  
    \begin{enumerate} [(a)]
        \item $\frac{1}{20}$
        \item $\frac{3}{242}$
        \item $\frac{5}{22}$
        \item $\frac{6}{25}$
        \item $\frac{7}{15}$
    \end{enumerate}
     \begin{sol}
     resposta: a \\
     $\frac{30}{10+30+60}\cdot\frac{20}{20+20+80}=\frac{30}{100}\cdot\frac{20}{120}=\frac{1}{2}$
     \end{sol}
 \end{ex}