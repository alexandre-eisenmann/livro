\begin{ex}
  (Enem) Um casal, ambos com 30 anos de idade, pretende fazer um plano de previdência. A seguradora pesquisada, para definir o valor do recolhimento mensal, estima a probabilidade de que pelo menos um deles esteja vivo daqui a 50 anos, tomando por base dados da população, que indicam que 20\% dos homens e 30\% das mulheres de hoje alcançarão a idade de 80 anos. Qual é essa probabilidade?
     \begin{enumerate}  [(a)]
         \item 50\%
         \item 44\%
         \item 38\%
         \item 25\%
         \item 6\%
     \end{enumerate}
       \begin{sol}
        resposta: b \\
        - probabilidade (homem morrer daqui a 50 anos) = 80\%  \\
        - probabilidade ( mulher morrer daqui a 50 anos)= 70\% \\
        - probabilidade (dos 2 estarem mortos daqui a 50 anos) = $80\%\cdot70\%=56\%$ \\
        - probabilidade ( um deles estar vivo daqui a 50 anos) = $1-0,56=0,44=44\%$
       \end{sol}
  \end{ex}