\begin{ex}
(FGV) Um fundo de investimento disponibiliza números inteiros de cotas aos interessados nessa aplicação financeira. No primeiro dia de negociação desse fundo, verifica-se que 5 investidores compraram cotas, e que foi vendido um total de 9 cotas. Em tais condições, o número de maneiras diferentes de alocação das 9 cotas entre os 5 investidores é igual a:
   \begin{enumerate}[(a)]
   \item 56
   \item 70
   \item 86
   \item 120
   \item 126
   \end{enumerate}
     \begin{sol}
      \phantom{A} \\
É uma combinação com repetição: 
$ C_R(n,k)=
\left (
\begin{array}{c}
n+k-1 \\
k
\end{array}
\right )
$ \\
9 cotas e 5 investidores compraram  pelo menos uma, logo sobram 4 cotas para dividir entre os 5. \\
$C_R(5,4)=
\left(
\begin{array}  {c}
5+4-1 \\
4
\end{array}
\right)
$ = $\mathrm{C}_{8,4}=70$
      
      

     \end{sol}
\end{ex}