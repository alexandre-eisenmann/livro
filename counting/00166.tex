\begin{ex}
(UF Sta Maria – RS) Numa câmara de vereadores, trabalham 6 vereadores do partido A, 5 vereadores do partido B e 4 vereadores do partido C. O número de comissões de 7 vereadores que podem ser formadas, devendo cada comissão ser constituída de 3 vereadores do partido A, 2 do partido B e 2 do partido C, é igual a:
   \begin{enumerate}[(a)]
   \item 7
   \item 36
   \item 152
   \item 1200
   \item 28800
   \end{enumerate}
     \begin{sol}
       resposta: d \\
       $\mathrm{C}_{6,3}\cdot\mathrm{C}_{5,2}\cdot\mathrm{C}_{4,2}=1200$
     \end{sol}
\end{ex}