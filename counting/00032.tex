\begin{ex}
 (Enem) Em um determinado ano, os computadores da receita federal de um país identificaram como inconsistentes 20\% das declarações de imposto de renda que lhe foram encaminhadas. Uma declaração é classificada como inconsistente quando apresenta algum tipo de erro ou conflito nas informações prestadas. Essas declarações consideradas inconsistentes foram analisadas pelos auditores, que constataram que 25\% delas eram fraudulentas. Constatou-se ainda que, dentre as declarações que não apresentaram inconsistências, 6,25\% eram fraudulentas. Qual é a probabilidade de, nesse ano, a declaração de um contribuinte ser considerada inconsistente, dado que ela era fraudulenta? 
   \begin{enumerate} [(a)]
       \item 0,0500
       \item 0,1000
       \item 0,1125
       \item 0,3125
       \item 0,5000
   \end{enumerate}
      \begin{sol}
      resposta: e \\
      $20\%$ são inconsistentes e $80\%$ são consistentes. \\
      $25\%$ de $20\%=5\%$ são inconsistentes fraudulentas.\\
      $6,25\%$ de $80\%= 5\%$ são consistentes fraudulentas\\
      Monta-se uma tabela:  \\ \\
      \begin{tabular}{c|c|c|c} \hline
           & fraudulenta & não fraudulenta & total \\   \hline
     inconsistente  &  5\% & 15\% & 20\%  \\  \hline
     consistente &5\% &75\% &80\% \\  \hline
     total & 10\%  & 90\% & 100\% \\   \hline
      \end{tabular}
      $\Longrightarrow p(\frac{inconsistente}{fraudulenta})=\frac{5\%}{10\%}=0,5000$
      \end{sol}
 \end{ex}