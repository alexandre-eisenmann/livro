\begin{ex}
(UF –RN) Um jogo consiste em um prisma triangular reto com uma lâmpada em cada vértice e um quadro de interruptores para acender essas lâmpadas. Sabendo que qualquer 3 lâmpadas podem ser acesas por um único interruptor e que cada interruptor acende precisamente 3 lâmpadas, calcule:
   \begin{enumerate}[(a)]
   \item quantos interruptores existem nesse quadro;
   \item a probabilidade de, ao se escolher um interruptor aleatoriamente, este acender 3 lâmpadas numa mesma face. 
   \end{enumerate}
     \begin{sol}
       \phantom{A} 
        \begin{enumerate} [(a)]
            \item $\mathrm{C}_{6,3}=20$
            \item $\frac{3\cdot\mathrm{C}_{4,3}+2}{20}=\frac{7}{10}$
        \end{enumerate}
     \end{sol}
\end{ex}