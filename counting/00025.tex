\begin{ex}
 (Enem) Eduardo deseja criar um e-mail utilizando um anagrama exclusivamente com as sete letras que compõem o seu nome, antes do símbolo @. O e-mail terá a forma *******@site.com.br e será de tal modo que as três letras “edu” apareçam sempre juntas e exatamente nessa ordem. Ele sabe que o e-mail eduardo@site.com.br já foi criado por outro usuário e que qualquer outro agrupamento das letras do seu nome forma um e-mail que ainda não foi cadastrado. De quantas maneiras Eduardo pode criar um e-mail desejado? \\ \\
 (a) 59\hspace{0,7cm}  (b) 60\hspace{0,7cm}   (c) 118\hspace{0,7cm}   (d) 119\hspace{0,7cm}   (e) 120
      \begin{sol}
       resposta: d \\
     considerando as letras EDU juntas e nessa ordem, temos $P_5=5!=120$ possibilidades. Tirando o caso EDUARDO, temos $120-1=119$ e-mails possíveis.
      \end{sol}
 \end{ex}