\begin{ex}
(Puc – SP) O novo sistema de placas de veículos utiliza um grupo de 3 letras (dentre 26 letras) e um grupo de 4 algarismos. Uma placa dessas será “palíndroma” se os dois grupos que as constituem forem “palíndromos”. O grupo ABA é “palíndromo” pois as leituras da esquerda para a direita e da direita para a esquerda são iguais; da mesma forma, o grupo 1331 é “palíndromo”. Quantas placas “palíndromas” distintas podem ser construídas?
  \begin{sol}
    \phantom{A} \\
    $\frac{\phantom{A}}{26}\frac{\phantom{A}}{26}\frac{\phantom{A}}{1}\frac{\phantom{A}}{10}\frac{\phantom{A}}{10}\frac{\phantom{A}}{1}\frac{\phantom{A}}{1} \Longrightarrow  26\cdot26\cdot100=67600$
  \end{sol}
\end{ex}