\begin{ex}
 Um sistema automático de alarme contra incêndio utiliza três células sensíveis ao calor que agem independentemente uma da outra. Cada célula entra em funcionamento com probabilidade 0,8 quando a temperatura atinge 60$^{\circ}$C. Se pelo menos duas das células entrarem em funcionamento, o alarme soa. A probabilidade do alarme soar quando a temperatura atingir 60$^{\circ}$C é:
    \begin{enumerate}[(a)]
    \item $(0,8)^3$ + $(0,8)^2$
    \item 1$- (0,2)^3$ - $(0,2)^2$
    \item $(0,8)^3 + (0,8)^2\cdot 0,2 $ 
    \item $(0,8)^3 + 3\cdot(0,8)^2\cdot0,2$
    \item $(0,8)^2$
    \end{enumerate}
      \begin{sol}
      resposta: d\\
        $0,8^2\cdot0,2\cdot\mathrm{C}_{3,2}+0,8^3=3.0,8^2\cdot0,2+0,8^3$
      \end{sol}
\end{ex}