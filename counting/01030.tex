\begin{ex}
 	(Fuvest) Uma urna contém 5 bolas brancas e 3 bolas pretas. Três bolas são retiradas ao acaso, sucessivamente , sem reposição. Determine:
    \begin{enumerate}[(a)]
    \item  a probabilidade de que tenham sido retiradas 2 bolas pretas e 1 bola branca .
    \item  a probabilidade de que tenham sido retiradas 2 bolas pretas e uma branca, sabendo-se que as 3 bolas retiradas não são da mesma cor.
    \end{enumerate}
      \begin{sol}
        \phantom{A} 
          \begin{enumerate} [(a)]
              \item  $3\cdot(\frac{3}{8}\cdot\frac{2}{7}\cdot\frac{5}{6})=\frac{15}{56}$
              \item 2 brancas e 1 preta  \\
              $3\cdot(\frac{5}{8}\cdot\frac{4}{7}\cdot\frac{5}{6})=\frac{30}{56} \Longrightarrow p= \frac{\frac{15}{56}}{\frac{15}{56}+\frac{30}{56}}=\frac{1}{3}$
              
          \end{enumerate}
      \end{sol}
\end{ex}