\begin{ex}
Um indivíduo esqueceu a senha de seu cartão bancário. Sabia que havia utilizado, sem repetição, todos os algarismos de sua data de nascimento – vinte e cinco de agosto de mil novecentos e setenta e três – e recorda-se que os algarismos dois e cinco estavam juntos. Se um dia ele consegue testar 144 senhas, em quanto tempo, no máximo, ele terá acesso à conta?
 \begin{sol}
   \phantom{A} \\
   25 e 52 $\rightarrow 7!\cdot2=10080\hspace{0,6cm}\therefore \hspace{0,6cm}10080\div144=70$ dias
 \end{sol}
\end{ex}