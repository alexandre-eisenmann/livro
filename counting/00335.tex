\begin{ex}
(Enem) Um município de 628 $km^2$ é atendido por duas emissoras de rádio cujas antenas A e B alcançam um raio de 10 km do município, conforme mostra a figura abaixo.\\
Para orçar um contrato publicitário, uma agência precisa avaliar a probabilidade que um morador tem de, circulando livremente pelo município, encontrar-se na área de alcance de pelo menos uma das emissoras.
   \begin{figure} [!htb]
       \centering
       \includegraphics[scale=0.7]{imagens/ex_729.png}
   \end{figure}
Essa probabilidade é de, aproximadamente:
   \begin{enumerate}[(a)]
   \item 20\%
   \item 25\%
   \item 30\%
   \item 35\%
   \item 40\%
   \end{enumerate}
    \begin{sol}
     resposta: b \\
    ângulo A + ângulo B = $180^{\circ}$ \hspace{0,6 cm} área = $\frac{(\pi {10}^2)}{2}=50\pi \Longrightarrow p=\frac{50\pi}{628}=\frac{157}{628}=25\%$
    \end{sol}
\end{ex}