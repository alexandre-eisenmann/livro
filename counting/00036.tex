\begin{ex} 
  (Enem) Torneios de tênis, em geral, são disputados em sistema de eliminatória simples. Nesse sistema, são disputadas partidas entre dois competidores, com a eliminação do perdedor e promoção do vencedor para a fase seguinte. Dessa forma, se na 1ª fase o torneio conta com 2n competidores, então na 2ª fase restarão n competidores, e assim sucessivamente até a partida final. Em um torneio de tênis, disputado nesse sistema, participam 128 tenistas. Para se definir o campeão desse torneio, o número de partidas necessárias é dado por:
    \begin{enumerate} [(a)]
        \item 2 x 128
        \item 64 + 32 + 16 + 8 + 4 + 2
        \item 128 + 64 + 32 + 16 + 8 + 4 + 2 + 1
        \item 128 + 64 + 32 + 16 + 8 + 4 + 2
        \item 64 + 32 + 16 + 8 + 4 + 2 + 1
    \end{enumerate}
      \begin{sol}
      resposta: e \\
      total de partidas: $64+32+16+8+4+2+1=127$
      \end{sol}  
 \end{ex}