\begin{ex}
Das 180 pessoas que trabalham em uma empresa, sabe-se que 40\% tem nível universitário e 60\% são do sexo masculino. Se 25\% do número de mulheres tem nível universitário, a probabilidade de selecionar-se um funcionário dessa empresa que seja do sexo masculino e não tenha nível universitário é:
   \begin{enumerate}[(a)]
   \item $\frac{5}{12}$
   \item $\frac{3}{10}$
   \item $\frac{2}{9}$
   \item $\frac{1}{5}$
   \item $\frac{5}{36}$
   \end{enumerate}
    \begin{sol}
     resposta: b \\
     \begin{tabular}{|c|c|c|c|} \hline
          & universitário & não universitário & total \\  \hline
   masculino  & 72-18=54 & 108-54=54 & 60\%=108 \\  \hline
   feminino & 25\% de 72=18 & 72-18=54 & 40\% 72  \\ \hline
   total & 40\% = 72 & 60\%=108 & 180 \\ \hline
     \end{tabular}
     $ \Longrightarrow \frac{54}{180}=\frac{6}{20}=\frac{3}{10}$
    \end{sol}
   
\end{ex}