\begin{ex}
   (Fatec) Em um supermercado, a probabilidade de que um produto da marca A e um produto da marca B estejam a dez dias, ou mais, do vencimento do prazo de validade é de 95\% e 98\%, respectivamente. Um consumidor escolhe, aleatoriamente, dois produtos, um produto da marca A e outro da marca B. Admitindo eventos independentes, a probabilidade de que ambos os produtos escolhidos estejam a menos de dez dias do vencimento do prazo de validade é:
     \begin{enumerate} [(a)]
         \item 0,001\%
         \item 0,01\%
         \item 0,1\%
         \item 1\%
         \item 10\%
     \end{enumerate}
      \begin{sol}
       resposta: c \\
       probabilidade dos produtos A e B estarem a menos de 10 dias do vencimento do prazo de validade:  (A) = $1-0,95=0,05$\hspace{0,5cm} (B) = $1-0,98=0,02$\\
       probabilidade de ambos estarem a menos de 10 dias do vencimento do prazo de validade é: \hspace{0,3cm} $0,05\cdot0,02=0,001=0,1\%$
      \end{sol}
  \end{ex}