\begin{ex}
 Considere um baralho de 12 cartas, formado por 3 figuras ( valete, dama e rei) dos 4 naipes. São retiradas simultaneamente 5 cartas. A probabilidade de sair uma quadra é:
    \begin{enumerate}[(a)]
    \item $\frac{1}{33}$
    \item $\frac{1}{165}$
    \item $\frac{1}{264}$
    \item $\frac{1}{495}$
    \item $\frac{3}{5}$
    \end{enumerate}
     \begin{sol}
       resposta: a\\
       $\frac{12}{12}\cdot\frac{3}{11}\cdot\frac{2}{10}\cdot\frac{1}{9}\cdot\frac{1}{8}\cdot\mathrm{C}_{5,1}=\frac{1}{33}$\hspace{0.5cm} ou \\
       Temos 3 quadras e 8 cartas quaisquer, então:
       $p= \frac{3\cdot8}{\mathrm{C}_{{12},5}}=\frac{1}{33}$
     \end{sol}
\end{ex}