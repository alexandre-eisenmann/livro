\begin{ex}
 (Ufrj) Uma caixa contém bombons de nozes e bombons de passas. O número de bombons de nozes é superior ao número de bombons de passas em duas unidades. Se retirarmos, ao acaso dois bombons dessa caixa, a probabilidade de que ambos sejam de nozes é $\frac{2}{7}$.
   \begin{enumerate}[(a)]
   \item determine o número total de bombons;
   \item se retirarmos, ao acaso, dois bombons da caixa, determine a probabilidade de que sejam de sabores distintos. 
   \end{enumerate}
      \begin{sol}
       \phantom{A}
       \begin{enumerate} [(a)]
           \item bombom de nozes = $x+2$\hspace{0,4cm} bombom de passas = $x$\\
          $\frac{x+2}{2x+2}\cdot\frac{x+1}{2x+1}=\frac{2}{7} \rightarrow x=10$ \hspace{0,4cm}
          total de bombons: $2x+2= 22$
          \item 12 bombons de nozes e 10 bombons de passas \\
          $\frac{12}{22}\cdot\frac{10}{21}+\frac{10}{22}\cdot\frac{12}{21}=\frac{40}{77}$
           
       \end{enumerate}
      \end{sol}
\end{ex}