\begin{ex}
 (Unesp) O corpo de enfermeiros plantonistas de uma clínica compõe-se de 6 homens e 4 mulheres. Isso posto, calcule: 
    \begin{enumerate}[(a)]
    \item quantas equipes de 6 plantonistas é possível formar com os 10 enfermeiros, levando em conta que em nenhuma delas deve haver mais homens que mulheres?
    \item a probabilidade de que, escolhendo-se aleatoriamente uma dessas equipes, ela tenha número igual de homens e mulheres.
    \end{enumerate}
      \begin{sol}
        \phantom{A}
        \begin{enumerate} [(a)]
            \item 3H e 3M ou 2H e 4M \\
            $\mathrm{C}_{6,3}\cdot\mathrm{C}_{4,3}+\mathrm{C}_{6,2}\cdot\mathrm{C}_{4,4}=80+15= 95$
            \item 3H e 3M \hspace{0,4cm} $\Longrightarrow \frac{80}{95}=\frac{16}{19}$
        \end{enumerate}
      \end{sol}
\end{ex}