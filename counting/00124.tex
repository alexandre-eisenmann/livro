\begin{ex}
Dentre os 200 alunos dos colégios A e B que foram aprovados no vestibular, apenas um será sorteado para receber uma bolsa de estudos. Sabe-se que:
   \begin{itemize}
   \item [--] 40\% estudaram no colégio A
   \item [--] 60\% são rapazes
   \item [--] 25\% das moças estudaram no colégio A
   \end{itemize}
Calcule a probabilidade de que o sorteado seja um rapaz do colégio B.
 \begin{sol}
   \phantom{A} 
   montar uma tabela: \\   \\
   \begin{tabular}{|c|c|c|c|}   \hline
          & A & B & Total  \\  \hline
    rapazes&60 & 60 & 120  \\  \hline
    moças & 20 & 60 & 80   \\  \hline
    Total & 80 & 120 & 200  \\   \hline
     \end{tabular}
     $ \Longrightarrow  p = \frac{60}{200}=\frac{3}{10}=0,3$  
 \end{sol}
\end{ex}