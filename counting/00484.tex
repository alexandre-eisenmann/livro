\begin{ex}
(Unifesp) Os alunos quartanistas do curso diurno e do curso noturno de uma faculdade se submeteram a uma prova de seleção, visando a participação numa olimpíada internacional. Dentre os que tiraram nota 9,5 ou 10,0 será escolhido um aluno por sorteio.
 \begin{center}
     \begin{tabular}{|c|c|c|} \hline 
     \multirow {2} {*}{NOTA} & \multicolumn{2}{c|}{CURSO} \\
     \cline{2-3}
      & Diurno & Noturno  \\ \hline
       9,5 & 6 & 7 \\ \hline
       10,0 & 5 & 8 \\  \hline
     \end{tabular}
 \end{center}
Com base na tabela, a probabilidade de que o aluno sorteado tenha tirado nota 10,0 e seja do curso noturno é:
   \begin{enumerate}[(a)]
   \item $\frac{12}{26}$
   \item $\frac{6}{14}$ 
   \item $\frac{4}{13}$
   \item $\frac{12}{52}$
   \item $\frac{1}{6}$
   \end{enumerate}
      \begin{sol}
        resposta: c \\  \\
        \begin{tabular}{|c|c|c|c|} \hline
        nota   & diurno& noturno & total \\  \hline
           9,5 & 6 & 7 & 13  \\  \hline
           10  & 5 & 8 & 13  \\  \hline
           total & 11 & 15 & 26 \\  \hline
        \end{tabular}
        $\Longrightarrow \frac{8}{26}=\frac{4}{13}$
       \end{sol}
\end{ex}