\begin{ex}
 Uma urna contém 18 cartelas numeradas. Sabe-se que 10 delas marcam "zero" e 8 marcam "1". Sorteiam-se sucessivamente e sem reposição 5 cartelas, formando um número de 5 dígitos. Qual é a probabilidade de sair:
    \begin{enumerate}[(a)]
    \item o número 10101?
    \item um número com exatamente três "uns"?
    \item um número com pelo menos um"zero"?
    \end{enumerate}
      \begin{sol}
          \phantom{A} 
       \begin{enumerate} [(a)]
           \item $\frac{\mathrm{A}_{8,3}\cdot\mathrm{A}_{{10},2}}{\mathrm{A}_{{18},5}} =  \frac{1}{34}$ \hspace{0.2cm}  ou \hspace{0.2 cm} $\frac{8}{18}\cdot\frac{10}{17}\cdot\frac{7}{16}\cdot\frac{9}{15}\cdot\frac{6}{14}= \frac{1}{34}$
           \item $\frac{\mathrm{C}_{8,3}\cdot\mathrm{C}_{{10},2}}{\mathrm{C}_{{18},5}}=\frac{5}{17}$\hspace{0.2cm} ou \hspace{0.2cm} $\frac{8}{18}\cdot\frac{7}{17}\cdot\frac{6}{16}\cdot\frac{10}{15}\cdot\frac{9}{14}\cdot\mathrm{C}_{5,3} = \frac{5}{17}$
           \item Todos - (só com 1 ou só com um zero) $\Longrightarrow 1-\mathrm{P}(11111)-\mathrm{P}(10111)$\\
           $1-(\frac{8}{18}\cdot\frac{7}{17}\cdot\frac{6}{16}\cdot\frac{5}{15}\cdot\frac{4}{14})-\binom{5}{1}(\frac{8}{18}\cdot\frac{7}{17}\cdot\frac{6}{16}\cdot\frac{5}{15}\cdot\frac{4}{14})=\frac{31}{34}$\\
           ou \hspace{0.2cm}
           $1-(\frac{\mathrm{A}_{8,5}-\mathrm{A}_{{10},0}}{\mathrm{A}_{{18},5}}-\frac{\mathrm{A}_{8,4}.\mathrm{A}_{{10},1}}{\mathrm{A}_{{18},5}}\cdot \mathrm{C}_{5,1})=\frac{31}{34}$
       \end{enumerate}
      \end{sol}
\end{ex}