\begin{ex}
 (UFMG) Num grupo constituído de 15 pessoas, 5 vestem camisas amarelas, 5 vestem camisas vermelhas e 5 vestem camisas verdes. Deseja-se formar uma fila com essas pessoas de forma que as três primeiras vistam camisas de cores diferentes e que as seguintes mantenham a sequência de cores dadas pelas três primeiras. Nessa situação de quantas maneiras distintas se pode formar tal fila?
    \begin{enumerate}[(a)]
    \item $3\cdot (5!) ^3 $
    \item $(5!)^3$
    \item $(5!)^3\cdot(3!)$
    \item $\frac{15!}{3!\cdot5!}$
    \end{enumerate}
      \begin{sol}
        resposta: c \\
        $\frac{\mathrm{Am}}{5}\frac{\mathrm{Vm}}{5}\frac{\mathrm{Vd}}{5}
        \frac{\mathrm{Am}}{4}\frac{\mathrm{Vm}}{4}\frac{\mathrm{Vd}}{4}
        \frac{\mathrm{Am}}{3}\frac{\mathrm{Vm}}{3}\frac{\mathrm{Vd}}{3}
        \frac{\mathrm{Am}}{2}\frac{\mathrm{Vm}}{2}\frac{\mathrm{Vd}}{2}
        \frac{\mathrm{Am}}{1}\frac{\mathrm{Vm}}{1}\frac{\mathrm{Vd}}{1}\Longrightarrow (5!)^3\cdot3!$
        
      \end{sol}
\end{ex}