\begin{ex}
 (Puc-SP) Uma urna contém apenas cartões marcados com números de três algarismos distintos, escolhido de 1 a 9. Se, nessa urna, não há cartões com números repetidos, a probabilidade de ser sorteado um cartão com um número menor que 500 é:
    \begin{enumerate}[(a)]
    \item $\frac{3}{4}$
    \item $\frac{1}{2}$
    \item $\frac{8}{21}$
    \item $\frac{4}{9}$
    \item $\frac{1}{3}$
    \end{enumerate}
      \begin{sol}
        resposta: d \\
        espaço amostral = $9\cdot8\cdot7=504$ \\
        números menores que 500 = $4\cdot8\cdot7=224 \Longrightarrow  p=\frac{224}{504}=\frac{4}{9}$
        
      \end{sol}
\end{ex}