\begin{ex}
(Ita adaptado) Uma caixa branca contém 6 bolas verdes e 3 azuis e uma caixa preta contém 3 bolas verdes e 8 azuis. Pretende-se retirar uma bola de uma das caixas. Para tanto dois dados são lançados. Se a soma resultante dos 2 dados for menor que 4, retira-se uma bola da caixa branca. Nos demais casos, retira-se uma bola da caixa preta. Qual a probabilidade de se retirar uma bola verde?
  \begin{sol}
   \phantom{A}  \\
   soma menor que 4: $(1,1) (1,2) (2,1) \rightarrow \frac{3}{36}=\frac{1}{12} \therefore$ soma maior que 4 = $\frac{11}{12}$\\
   probabilidade de bola verde: soma menor que 4 e bola verde da caixa branca \textbf{ou} soma maior que 4 e bola verde da caixa preta $\Longrightarrow \frac{1}{12}\cdot\frac{6}{9} + \frac{11}{12}\cdot\frac{3}{11}=\frac{11}{36}$ 
   
  \end{sol}
\end{ex}