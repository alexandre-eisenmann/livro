\begin{ex}
 	(Vunesp) Em um colégio foi realizada uma pesquisa sobre as atividades extracurriculares de seus alunos. Dos 500 alunos entrevistados, 240 praticavam algum tipo de esporte, 180 frequentavam um curso de idiomas e 120 realizavam essas duas atividades, ou seja, praticavam um tipo de esporte e frequentavam um curso de idiomas. Se nesse grupo de 500 estudantes um é escolhido ao acaso, a probabilidade de que ele realize pelo menos uma dessas atividades, isto é, pratique um tipo de esporte ou frequente um curso de idiomas , é:
    \begin{enumerate}[(a)]
    \item $\frac{18}{25}$
    \item $\frac{3}{5}$
    \item $\frac{12}{25}$
    \item $\frac{6}{25}$
    \item $\frac{2}{5}$
    \end{enumerate}
      \begin{sol}
        resposta: b \\
        \begin{venndiagram2sets} [labelA=\(E\),labelB=\(I\),labelOnlyA=120,labelOnlyB=60,labelNotAB=200,labelAB=120]
        \end{venndiagram2sets}
        \\
        pelo menos uma atividade: esporte ou idiomas ou os 2 = 120 + 60 + 120= 300 \\
        $\Longrightarrow \frac{300}{500}=\frac{3}{5}$
      \end{sol}
\end{ex}