\begin{ex}
 (Fuvest) Dois triângulos congruentes, com lados coloridos, são indistinguíveis se podem ser sobrepostos de tal modo que as cores dos lados coincidentes sejam os mesmas. 
Dados dois triângulos equiláteros congruentes, cada um de seus lados é pintado com uma cor escolhida dentre duas possíveis, com igual probabilidade.  
A probabilidade de que esses triângulos sejam indistinguíveis é de: 
    \begin{enumerate}[(a)]
    \item $\frac{1}{3}$
    \item $\frac{3}{4}$
    \item $\frac{9}{16}$
    \item $\frac{5}{16}$
    \item $\frac{15}{32}$
    \end{enumerate}
     \begin{sol}
     resposta: d \\
     cores dos lados do triângulo: A e B \\
     pintura do triângulo e respectiva probabilidade:\\
      - 3 lados numa cor só (AAA ou BBB): $p=\frac{1}{2}\cdot\frac{1}{2}\cdot\frac{1}{2}=\frac{1}{8}$ \\
      - 2 lados de uma cor e um lado da outra cor (ABB ou BAA): $p=3\cdot\frac{1}{2}\cdot\frac{1}{2}\cdot\frac{1}{2}=\frac{3}{8}$ \\
     probabilidade de 2 triângulos indistinguíveis: \\ $p=\frac{1}{8}\cdot\frac{1}{8}+\frac{1}{8}\cdot\frac{1}{8}+\frac{3}{8}\cdot\frac{3}{8}+\frac{3}{8}\cdot\frac{3}{8}=\frac{20}{64}=\frac{5}{16}$
     \end{sol}
\end{ex}