\begin{ex}
 	(FGV) Um viajante diante de uma bifurcação da estrada, dirige-se ao posto de combustível mais próximo para saber que direção deve tomar, para chegar ao seu destino. Ocorre que nesse posto há três funcionários: Franco, que sempre fala a verdade; Hilário, que sempre mente e Dúbio, que diz a verdade duas em cada três vezes.
    \begin{enumerate}[(a)]
    \item se os 3 funcionários estiverem trabalhando no posto quando o viajante pedir a informação a um deles, qual a probabilidade de ele chegar ao seu destino corretamente?
    \item suponha, agora, que um único funcionário trabalhe em cada turno, que Franco trabalhe o dobro de turnos de Dúbio e que este último trabalhe uma vez e meia o número de turnos de Hilário. Nesse caso, qual a probabilidade de a informação ser correta?
    \end{enumerate}
       \begin{sol}
         \phantom{A} 
          \begin{enumerate} [(a)]
              \item $\frac{1}{3}\cdot1+\frac{1}{3}\cdot\frac{2}{3}=\frac{5}{9}$
              \item turnos: Hilário = h,  Dúbio = $\frac{3}{2} $ h,  Franco = $2\cdot\frac{3}{2}$ h. \\ Supondo h = 2 : Hilário = 2 turnos, Dúbio = 3 turnos e Franco = 6 turnos, num total de 11 turnos.\\
              Franco: $\frac{6}{11}\cdot1=\frac{6}{11}$;\hspace{0,3cm} Hilário: $\frac{2}{11}\cdot0=0 $;\hspace{0,3cm} Dúbio: $\frac{3}{11}\cdot\frac{2}{3}\Longrightarrow \frac{6}{11}+\frac{6}{33}=\frac{8}{11}$
          \end{enumerate}
       \end{sol}
\end{ex}