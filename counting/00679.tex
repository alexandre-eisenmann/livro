\begin{ex}
 Uma urna A contém 6 bolas brancas e 4 pretas; uma outra urna B, contém 3 bolas brancas e 6 pretas. Será retirada, aleatoriamente uma bola da urna A e colocada em B. Depois disso, qual será a probabilidade de se retirar uma bola da urna B e ela ser branca?
    \begin{sol}
        \phantom{A} \\
    Temos 2 alternativas: retirar uma bola branca da urna A e colocar na urna B ou  retirar uma bola preta da urna A e colocar na urna B.\\
    p(bola branca de A)=$\frac{6}{10}$\hspace{0.15cm} e\hspace{0.2cm}p(bola branca de B)=$\frac{4}{10}$ \hspace{0.2cm} ou \hspace{0.2cm}
    p(bola preta de A)=$\frac{4}{10}$\hspace{0.2cm} e \hspace{0.2cm} p(bola branca de B)=$\frac{3}{10}$\\
    $\frac{6}{10}\cdot\frac{4}{10}+\frac{4}{10}\cdot\frac{3}{10}= \frac{9}{25}$
   
    \end{sol}
 
\end{ex}