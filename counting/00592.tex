\begin{ex}
(Unopar –PR) Cada uma das dez questões de uma prova apresenta uma única afirmação, que deve ser classificada como verdadeira (V) ou falsa (F). Um aluno, que nada sabe sobre a matéria, vai responder todas as questões ao acaso. A probabilidade de ele não tirar zero é:
   \begin{enumerate}[(a)]
   \item $\frac{1}{256}$
   \item $\frac{511}{512}$
   \item $\frac{3}{512}$
   \item $\frac{1}{1024}$
   \item $\frac{1023}{1024}$
   \end{enumerate}
     \begin{sol}
      resposta: e \\
      $1-(\frac{1}{2})^{10}=\frac{1023}{1024}$
     \end{sol}
\end{ex}