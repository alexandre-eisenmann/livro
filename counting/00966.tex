\begin{ex}
 	(Fei) Uma urna contém em seu interior 6 moedas de 10 unidades monetárias (u.m.) e uma de 5 u.m. Uma segunda urna contém 4 moedas de 10 u.m. Retiram-se ao acaso 5 moedas da primeira urna, colocam-se as mesmas na segunda urna e, em seguida, sorteiam-se duas moedas da segunda, colocando-as na primeira urna. A probabilidade de, estar a moeda de 5 u.m. na primeira urna, após essas operações, é de:
    \begin{enumerate}[(a)]
    \item $\frac{1}{4}$
    \item $\frac{2}{11}$
    \item $\frac{5}{7}$
    \item $\frac{4}{7}$
    \item $\frac{4}{9}$
    \end{enumerate}
      \begin{sol}
      reposta: e \\
      moeda de 5 u.m. da 1ª urna: fica na 1ª urna ou sai e volta para a 1ª urna. \\
      $\frac{\mathrm{C}_{6,5}}{\mathrm{C}_{7,5}}+(1-\frac{\mathrm{C}_{6,5}}{\mathrm{C}_{7,5}})\cdot\frac{\mathrm{C}_{8,1}\cdot\mathrm{C}_{1,1}}{\mathrm{C}_{9,2}}=\frac{4}{9}$
      \end{sol}
\end{ex}