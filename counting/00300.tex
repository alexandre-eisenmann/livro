\begin{ex}
(Enem) Num determinado bairro há duas empresas de ônibus: Andabem e Bompasseio, que fazem o trajeto levando e trazendo passageiros do subúrbio ao centro da cidade. Um ônibus de cada uma dessas empresas parte do terminal a cada 30 minutos, nos horários indicados na tabela: 
\begin{center}
\begin{tabular}{|c|c|}  \hline
\multicolumn {2} {|c|} {HORÁRIO DOS ÔNIBUS} \\  \hline
\cline{1-2}
Andabem & Bompassseio  \\  \hline
........ & ........ \\ \hline
6h 00min & 6h 10min \\   \hline
6h 30min & 6h 40min \\   \hline
7h 00min & 7h 10min \\  \hline
7h 30min & 7h 40min  \\  \hline
........ & .......  \\  \hline
\end{tabular}
\end{center}
Carlos mora próximo ao terminal de ônibus e trabalha na cidade. Como não tem hora certa para chegar ao trabalho nem preferência por qualquer das empresas, toma sempre o primeiro ônibus que sai do terminal. Nessa situação, pode-se afirmar que a probabilidade de Carlos viajar num ônibus da empresa Andabem é:
   \begin{enumerate}[(a)]
   \item $\frac{1}{4}$  da probabilidade de viajar num ônibus da empresa Bompasseio;
   \item $\frac{1}{3}$ da probabilidade de viajar num ônibus da empresa Bompasseio;
   \item $\frac{1}{2}$ da probabilidade de viajar num ônibus da empresa Bompasseio;
   \item duas vezes maior do que a probabilidade de viajar num ônibus da empresa Bom-passeio;
   \item três vezes maior do que a probabilidade de viajar num ônibus da empresa Bom-passeio;
   \end{enumerate}
     \begin{sol}
       resposta: d \\
         Se Carlos chegar entre 6h e 6h10min, pega o Bompasseio e se Carlos chegar entre 6h10min e 6h30min, pega o Andabem. Esse padrão se repete ao longo do dia. \\
         Conclui-se que em 20 minutos de cada meia hora Carlos vai de Andabem e em 10 minutos de cada meia hora vai de Bompasseio. Logo a chance de Carlos ir de Andabem é duas vezes maior que a de ir de Bompasseio.
        
     \end{sol}
\end{ex}