\begin{ex}
Na convenção de um partido político, devem ser escolhidos dois candidatos para formar a chapa que irá disputar as próximas eleições presidenciais. A escolha deve ser feita entre 3 homens e 2 mulheres, candidatos à presidência e 2 homens e 4 mulheres, candidatos à vice-presidência. Admitindo que todos os candidatos tenham a mesma probabilidade de serem escolhidos,  a probabilidade de que a chapa vencedora tenha um homem como candidato à presidência, e uma mulher como candidata à vice- presidência é:
   \begin{enumerate}[(a)]
   \item 40\%
   \item 36\%
   \item 46\%
   \item 28\%
   \item 25\%
   \end{enumerate}
     \begin{sol}
      resposta: a \\
      $\frac{3}{5}\cdot\frac{4}{6}=\frac{12}{30}=40\%$
     \end{sol}
\end{ex}