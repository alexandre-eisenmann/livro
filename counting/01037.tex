\begin{ex}
(Unesp) Paulo deve enfrentar em um torneio dois outros jogadores , João e Mário.\\ Considere os eventos:	(A) Paulo vence João e (B) Paulo vence Mário. \\
Os resultados dos jogos são eventos independentes. Sabendo que a probabilidade de Paulo vencer ambos os jogadores é $\frac{2}{5}$ e a probabilidade dele ganhar de João é  $\frac{3}{5}$, determine a probabilidade de Paulo perder dos dois jogadores: João e Mário.
  \begin{sol}
    \phantom{A} 
      \begin{enumerate} [(a)]
          \item Paulo vencer ambos = Paulo vencer de João e vencer de Mário (\textit{x}) \\
          $\frac{2}{5}=\frac{3}{5}.x \Longrightarrow x=\frac{2}{3}$
           \item Paulo perder de Mário =$\frac{1}{3}$; logo Paulo perder de ambos é: $\longrightarrow\frac{2}{5}\cdot\frac{1}{3}=\frac{2}{15}$
           
      \end{enumerate}
  \end{sol}
\end{ex}