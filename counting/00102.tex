\begin{ex}
  (Unesp) Um dado viciado, que será lançado uma única vez, possui seis faces, numeradas de 1 a 6. A tabela a seguir fornece a probabilidade de ocorrência de cada face.
   \begin{center}
       \begin{tabular}{|c|c|c|c|c|c|c|}  \hline
           número na face & 1 & 2 & 3 & 4 & 5 & 6 \\
 \hline            
probabilidade de ocorrência da face & $\frac{1}{5}$ & $\frac{3}{10}$ & $\frac{3}{10}$ & $\frac{1}{10}$ & $\frac{1}{20}$ & $\frac{1}{20}$ \\ \hline
       \end{tabular}
   \end{center}
  Sendo X o evento “sair um número ímpar” e Y um evento cuja probabilidade de ocorrência seja 90\%, calcule a probabilidade de ocorrência de X e escreva uma possível descrição do evento Y. 
    \begin{sol}
     \phantom{A} \\
     $\text{X}= \frac{1}{5}+\frac{3}{10}+\frac{1}{20}=\frac{11}{20}$ \\
     Y pode ser: sair um número menor ou igual a 4 \hspace{0,3cm} $\Rightarrow \text{Y}=\frac{1}{5}+\frac{3}{10}+\frac{3}{10}+\frac{1}{10}=\frac{9}{10}=90\%$
    \end{sol}
 \end{ex}