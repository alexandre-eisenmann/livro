\begin{ex}
Uma urna contém \textit{x} bolas brancas, $\textit{x}^2$ bolas vermelhas e duas bolas pretas. Uma bola é escolhida ao acaso e sabe-se que a probabilidade de ela ser branca é maior que 20\%. Quantas bolas brancas essa urna pode conter?
  \begin{sol}
    \phantom{A}\\
    probabilidade de ocorrer bola branca: $\frac{x}{x+x^2+2} > 20\% \rightarrow \frac{x}{x+x^2+2} > 0,2$ \\
    resolvendo a inequação temos: $ x=3\hspace{0,3cm} \text{ou}\hspace{0,3cm} x=2\hspace{0,3cm} \text{ou}\hspace{0,3cm} x=1$
  \end{sol}
\end{ex}