\begin{ex}
(Fatec) No lançamento de um dado, seja $p_k$ a probabilidade de se obter um número $k$, com:  $ p_1=p_3=p_5=x $  e $ p_2=p_4=p_6=y $. Se, num único lançamento, a probabilidade de se obter um número menor ou igual a três é $\frac{3}{5}$, então $ x-y $ é igual a :
   \begin{enumerate}[(a)]
   \item $\frac{1}{15}$
   \item $\frac{2}{15}$
   \item $\frac{1}{5}$
   \item $\frac{4}{15}$
   \item $\frac{1}{3}$
   \end{enumerate}
    \begin{sol}
     resposta: c \\
     $
     \left\{
     \begin{array}{cl}
     x+x+x+y+y+y=1\\
     x+x+y=\frac{3}{5}
     \end{array}
     \right.
     $
     resolvendo o sistema , temos: $x=\frac{4}{15}\hspace{0,1cm}y=\frac{1}{15} \therefore x-y=\frac{1}{5}$
    \end{sol}
\end{ex}