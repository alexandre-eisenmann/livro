\begin{ex}
Numa eleição para representante de uma classe, todos os 30 alunos votaram em um dos candidatos A ou B. O candidato A venceu com o total de 20 votos. 
Escolhendo-se ao acaso dois dos alunos que votaram nessa eleição, qual é a probabilidade de que pelo menos um deles tenha votado no vencedor?
 \begin{sol}
  \phantom{A} \\
  ninguém votando em A $\hspace{0,3cm} \rightarrow \hspace{0,2cm} p=\frac{10}{30}\cdot\frac{9}{29}=\frac{3}{29}$\\
  pelo menos 1 votou em A $ \hspace{0,3cm} \Longrightarrow \hspace{0,2cm} p=1-\frac{3}{29}=\frac{26}{29}$
\end{sol}
\end{ex}