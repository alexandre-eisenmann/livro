\begin{ex}
(Uerj) Numa cidade, os números telefônicos não podem começar por zero e possuem 8 algarismos, dos quais os quatro primeiros constituem o prefixo.	Considere que os quatro últimos dígitos de todas as farmácias são 0000 e que o prefixo da farmácia Vivavida é formado pelos dígitos 2, 4, 5, e 6, não repetidos e não necessariamente nessa ordem.
O número máximo de tentativas a serem feitas para identificar o número telefônico completo dessa farmácia equivale a:
   \begin{enumerate}[(a)]
   \item 6
   \item 24
   \item 64
   \item 168
   \end{enumerate}
     \begin{sol}
     resposta: b \\
     $\frac{\phantom{A}}{4}\frac{\phantom{A}}{3}\frac{\phantom{A}}{2}\frac{\phantom{A}}{1}\frac{0}{1}\frac{0}{1}\frac{0}{1}\frac{0}{1} = 24 $
     \end{sol}
\end{ex}