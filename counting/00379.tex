\begin{ex}
Uma caixa de joias contém exatamente 5 pérolas falsas e 6 pérolas verdadeiras. Retirando simultaneamente 4 pérolas dessa urna, calcule a probabilidade de obter:
   \begin{enumerate}[(a)]
   \item 3 verdadeiras e uma falsa.
   \item todas verdadeiras.
   \item pelo menos uma falsa.
   \end{enumerate}
     \begin{sol}
      \phantom{A} 
       \begin{enumerate} [(a)]
           \item $\frac{\mathrm{C}_{6,3}\cdot \mathrm{C}_{5,1}}{\mathrm{C}_{{11},4}}=\frac{10}{33}$\hspace{0,4cm} \textbf{ou}\hspace{0,4cm} $\frac{6}{11}\cdot\frac{5}{10}\cdot\frac{4}{9}\cdot\frac{5}{8}\cdot \mathrm{C}_{4,1}=\frac{10}{33}$
           \item $\frac{\mathrm{C}_{6,4}}{\mathrm{C}_{{11},4}}=\frac{1}{22}$\hspace{0,4cm}\textbf{ou}\hspace{0,4cm}$\frac{6}{11}\cdot\frac{5}{10}\cdot\frac{4}{9}\cdot\frac{3}{8}=\frac{1}{22}$
           \item (1-todas verdadeiras) $\Longrightarrow 1-\frac{\mathrm{C}_{6,4}}{\mathrm{C}_{{11},4}}=\frac{21}{22}$
           
       \end{enumerate}
     \end{sol}
\end{ex}