\begin{ex}
	(Ufg) Um grupo de 150 pessoas é formado por 28\% de crianças, enquanto o restante é composto de adultos. Classificando esse grupo por sexo, sabe-se que $\frac{1}{3}$ dentre os de sexo masculino é formado por crianças e que $\frac{1}{5}$ dentre os do sexo feminino também é formado por crianças. Escolhendo ao  acaso uma pessoa nesse grupo, calcule a probabilidade dessa pessoa ser uma criança do sexo feminino.
	  \begin{sol}
	    \phantom{A} \\  
	   $ 28\% = 42 $ crianças, sendo um terço do sexo masculino e um quinto são do sexo feminino. Constrói-se um sistema:
	      $
	      \left\{
	      \begin{array}{cl}
	        m+f=150 \\
	        \frac{m}{3}+\frac{f}{5}=42\\
	      \end{array}
	      \right.
	      \longrightarrow m=90\hspace{0,4 cm} f=60 $ \\ $\frac{1}{5}\cdot60=12\Longrightarrow p=\frac{12}{150}= 8\% $
	  \end{sol}
\end{ex}