\begin{ex}
  (Unesp) Uma pesquisa publicada pela revista Veja em 07/06/2006 sobre os hábitos alimentares dos brasileiros mostra que no almoço, aproximadamente 70\% dos brasileiros comem carne bovina e que, no jantar, esse índice cai para 50\%. Supondo que a probabilidade condicional de uma pessoa comer carne bovina no jantar, dado que ela comeu carne bovina no almoço, seja $\frac{6}{10}$ , determine a probabilidade de a pessoa comer carne bovina no almoço ou no jantar.
    \begin{sol}
      \phantom{A}\\
      almoço e jantar: $\frac{6}{10}\cdot70 = 42$ \\
      só almoço: $70-42=28$ \\
      só jantar $50-42=8$ \\
      almoço ou jantar: $28+42+8=78\%$ \\
    
    \end{sol}
\end{ex}