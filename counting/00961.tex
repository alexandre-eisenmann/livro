\begin{ex}
 (Puc – RJ) Em uma amostra de 20 peças, existem exatamente 4 defeituosas.
    \begin{enumerate}[(a)]
    \item calcule o número de maneiras diferentes de escolher, sem reposição, uma peça perfeita e uma defeituosa.
    \item retirando-se, ao acaso, sem reposição, 3 peças, calcule a probabilidade de exatamente duas serem perfeitas. Escreva a resposta em fração.                 
    \end{enumerate}
      \begin{sol}
        \phantom{A}  
          \begin{enumerate}  [(a)]
              \item $16\cdot4=64$
              \item $\frac{16}{20}\cdot\frac{15}{19}\cdot\frac{4}{18}\cdot3=\frac{8}{19}$
          \end{enumerate}
      \end{sol}
\end{ex}