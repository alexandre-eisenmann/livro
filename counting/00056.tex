\begin{ex}
 (Unicamp) O sistema de segurança de um aeroporto consiste de duas inspeções. Na primeira delas, a probabilidade de um passageiro ser inspecionado é de 3/5. Na segunda, a probabilidade se reduz para 1/4. A probabilidade de um passageiro ser inspecionado pelo menos uma vez é igual a:
   \begin{enumerate} [(a)]
       \item $\frac{17}{20}$
       \item $\frac{7}{10}$
       \item $\frac{3}{10}$
       \item $\frac{3}{20}$
   \end{enumerate}
     \begin{sol}
     resposta: b \\
     A probabilidade do passageiro não ser inspecionado nas 2 inspeções é: $\frac{2}{5}\cdot\frac{3}{4}=\frac{3}{10}$ \\
     A probabilidade dele ser inspecionado é: $1-\frac{3}{10}=\frac{7}{10}$
     \end{sol}
 \end{ex}