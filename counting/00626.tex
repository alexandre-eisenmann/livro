\begin{ex}
 Uma urna contém apenas cartões marcados com números de três algarismos distintos, escolhidos de 1 a 9. Se, nessa urna, não há cartões com números repetidos, a probabilidade de ser sorteado um cartão com número menor que 500, é:
    \begin{enumerate}[(a)]
    \item $\frac{3}{4}$
    \item $\frac{1}{2}$
    \item $\frac{8}{21}$
    \item $\frac{4}{9}$
    \item $\frac{1}{3}$
    \end{enumerate}
    \begin{sol}
    resposta: d\\ 
 $\frac{\phantom{A}}{9}\frac{\phantom{A}}{8}\frac{\phantom{A}}{7}=504$\phantom{AA} (menor que 500)  $\frac{\phantom{10}}{4}\frac{\phantom{10}}{8}\frac{\phantom{10}}{7}= 224$\\ \\
 $\frac{224}{504}=\frac{4}{9}$

    \end{sol}
\end{ex}