\begin{ex}
  (Fuvest) Um aplicativo de videoconferências estabelece, para cada reunião , um código de 10 letras, usando um alfabeto completo de 26 letras. A quantidade de códigos distintos possíveis está entre:
  \begin{enumerate}  [(a)]
      \item 10 bilhões e 100 bilhões
      \item 100 bilhões e 1 trilhão
      \item 1 trilhão e 10 trilhões
      \item 10 trilhões e 100 trilhões
      \item 100 trilhões e 1 quatrilhão
  \end{enumerate}
  Note e adote: $\log_{10} 13 \cong 1,114; \hspace{0.4cm} 1bi = 10^9$
    \begin{sol}
     resposta: e\\
     $26^{10}=(2\cdot13)^{10}=2^{10}\cdot 13^{10}=1024\cdot13^{10}$ \\
     $log_{10}13 \cong1,114 \Rightarrow 10^{1,114}=13\Rightarrow (10^{1,114})^{10}=13^{10}\Longrightarrow 10^{11,14}=13^{10}$\\
     $1024\cdot13^{10}=1024\cdot10^{11,14}=1,024\cdot10^3\cdot10^{11,14}=1,024\cdot10^{14,14}$\\
     $10^{14} < 1,02\cdot10^{14,14} <10^{15}$\hspace{0,6cm} $\Longrightarrow 1\cdot10^{14}<1,02\cdot10^{14,14}<1\cdot10^{15}$\\
     $100\cdot10^{12}<102,4\cdot10^{12,14}<1\cdot10^{15}$\hspace{0,6cm} $\Longrightarrow 100 \hspace{0,1cm} \text{trilhões}< 102,4\cdot10^{12,14}< 1 \hspace{0,1cm} \text{trilhão}$
    \end{sol}
 \end{ex}