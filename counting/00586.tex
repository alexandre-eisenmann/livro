\begin{ex}
Uma dona de casa tem o hábito de guardar na caixa de fósforos os palitos já queimados. Em determinado dia, havia na caixa exatamente 18 palitos queimados e 22 perfeitos. Calcule a probabilidade de:
   \begin{enumerate}[(a)]
   \item retirando um palito da caixa, ao acaso, este seja perfeito.
   \item retirando um palito da caixa, ao acaso, este seja queimado.
   \item retirando três palitos da caixa, sucessivamente e sem reposição, apenas o terceiro seja perfeito.
   \end{enumerate}
     \begin{sol}
      \phantom{A}
       \begin{enumerate} [(a)]
           \item $\frac{22}{40}=\frac{11}{20}$
           \item $\frac{18}{40}=\frac{9}{20}$
           \item $\frac{18}{40}\cdot\frac{17}{39}\cdot\frac{22}{38}=\frac{561}{4940}$
       \end{enumerate}
     \end{sol}
\end{ex}