\begin{ex}
  (Unicamp) Uma empresa tem 5000 funcionários. Desses, 48\% tem mais de 30 anos, 36\%  são especializados e 1400 tem mais de 30 anos e são especializados. Com base nesses dados, pergunta-se:
    \begin{enumerate}[(a)]
    \item quantos funcionários tem mais de 30 anos e não são especializados?
    \item escolhendo um funcionário ao acaso,  qual a probabilidade de ele ter até 30 anos e ser especializado?
    \end{enumerate}
      \begin{sol}
       Diagrama \\
        \begin{venndiagram2sets}[labelA=\(>30\),labelB=espec.,labelOnlyA=1000,labelOnlyB=400,labelAB=1400,labelNotAB=2200,radius=1.4cm]
        \end{venndiagram2sets}
        \begin{enumerate} [(a)]
            \item $5000-1000-1400-400=2200$
            \item $\frac{400}{5000}=8\%$
        \end{enumerate}
      \end{sol}
      
\end{ex}