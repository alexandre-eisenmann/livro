\begin{ex}
 Uma cidade tem 50.000 habitantes e 3 jornais: A, B e C. Sabe-se que: 
    \begin{itemize}
    \item[--] 15.000 leem o jornal A
    \item[--] 10.000 leem o jornal B
    \item[--] 8.000 leem o jornal C
    \item[--] 6.000 leem os jornais A e B
    \item[--] 4.000 leem os jornais A e C
    \item[--] 3.000 leem os jornais B e C
    \item[--] 1.000 leem os três jornais.
    \end{itemize}
Uma pessoa é selecionada ao acaso. Qual a probabilidade de que:
    \begin{enumerate}[(a)]
    \item ela leia pelo menos um jornal?
    \item ela leia um só jornal?
    \end{enumerate}
      \begin{sol}
       Diagrama\\  \\
        \begin{venndiagram3sets}
        [labelOnlyA=6000,labelOnlyB=2000,labelOnlyC=2000,labelNotABC=29000,labelABC=1000,labelOnlyAB=5000,labelOnlyAC=3000,labelOnlyBC=2000,radius=1.6cm,tikzoptions={scale=1.3}]
        \end{venndiagram3sets}
          \begin{enumerate}  [(a)]
              \item $\frac{21000}{50000}=\frac{21}{50}$
              \item $\frac{6000+2000+2000}{50000}=\frac{10000}{50000}=\frac{1}{5}$
          \end{enumerate}
      \end{sol}
\end{ex}