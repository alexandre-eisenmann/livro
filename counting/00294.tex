\begin{ex}
(UF – BA adaptado) Em uma escola o 3º ano colegial tem duas turmas: A e B. A tabela mostra a distribuição, por sexo, dos alunos dessas turmas:
\begin{center}
\begin{tabular} {|c|c|c|} \hline
TURMA & HOMENS & MULHERES \\ \hline
A & 20 & 35 \\  \hline
B & 25 & 20 \\  \hline 
\end{tabular}
\end{center}
Com base nesses dados, assinale V ou F nas afirmações seguintes, justificando as falsas:
   \begin{enumerate}[(a)]
   \item 	Escolhendo-se ao acaso, um aluno do 3º ano, a probabilidade de ser homem é igual a 0,45.
   \item 	Escolhendo-se ao acaso, um aluno do 3º ano B, a probabilidade de ser mulher é de 20\%.
   \item 	Escolhendo-se ao acaso, simultaneamente dois alunos, um de cada turma, a probabilidade de serem os dois do mesmo sexo é igual a $\frac{16}{33}$.
   \item	Escolhendo-se ao acaso, um aluno do 3º ano , a probabilidade de ser mulher ou de ser da turma B é igual a 80\%.
   \item Reunindo-se as mulheres das duas turmas e escolhendo-se uma ao acaso, a probabilidade de ser da turma A é igual a 35\%.
   \end{enumerate}
    \begin{sol}
      \phantom{A}
       \begin{enumerate} [(a)]
           \item V
           \item F $(44,4\%)$
           \item V $(\frac{20}{55}\cdot\frac{25}{45 }+\frac{35}{55}\cdot\frac{20}{45}=\frac{16}{33})$
           \item V $(\frac{55}{100}+\frac{45}{100}-\frac{20}{100}=\frac{80}{100})$
           \item F $(\frac{35}{55}=\frac{7}{11})$
       \end{enumerate}
    \end{sol}
\end{ex}