\begin{ex}
  (Enem) O psicólogo de uma empresa aplica um teste para analisar a aptidão de um candidato a determinado cargo. O teste consiste em uma série de perguntas cujas respostas devem ser verdadeiro ou falso e termina quando o psicólogo fizer a décima pergunta ou quando o candidato der a segunda resposta errada. Com base em testes anteriores, o psicólogo sabe que a probabilidade de o candidato errar uma resposta é 0,20. A probabilidade de o teste terminar na quinta pergunta é:
    \begin{enumerate} [(a)]
        \item 0,02048
        \item 0,08192
        \item 0,24000
        \item 0,40960
        \item 0,49152
    \end{enumerate}
      \begin{sol}
       resposta: b \\
       Errar uma das 4 primeiras perguntas e errar a 5ª pergunta.\\
       E = errada C = certa \hspace{0,5cm} 4 possibilidades: CCCEE, CCECE, CECCE, ECCCE\\ $\Longrightarrow p=\frac{8}{10}\cdot\frac{8}{10}\cdot\frac{8}{10}\cdot\frac{2}{10}\cdot\frac{2}{10}\cdot4=\frac{2^{13}}{10^5}=\frac{8192}{10000}=0,08192$
      
      \end{sol}
    \end{ex}