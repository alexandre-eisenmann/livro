\begin{ex}
  Numa urna há 12 bolas numeradas . Uma delas tem o número 1, a outra o número 2, a seguinte o número 3 e assim por diante. As bolas de 1 a 5 são da cor preta e as demais são brancas. Serão retiradas 5 bolas dessa urna, uma de cada vez e sem reposição, e queremos que sobre isso você calcule:
    \begin{enumerate}[(a)]
    \item quantos grupos diferentes poderão ser formados sem que todas  as bolas sejam de uma única cor?
    \item quantos grupos diferentes poderão ser formados com mais números pares do que ímpares?
    \item qual é  a probabilidade que saiam apenas bolas com números primos?
    \item qual é a probabilidade que saia um grupo de bolas com números seguidos, como 1, 2, 3, 4 e 5 ou 4, 5, 6, 7 e 8, etc...?
    \end{enumerate}
       \begin{sol}
         \phantom{A}
           \begin{enumerate} [(a)]
               \item $\mathrm{C}_{{12},5}=\frac{12!}{5!\cdot7!}=792$
               \item 3 pares e 2 ímpares ou 4 pares e 1 ímpar ou 5 pares: \\
               $\mathrm{C}_{6,3}\cdot\mathrm{C}_{6,2}+\mathrm{C}_{6,4}\cdot\mathrm{C}_{6,1}+\mathrm{C}_{6,5}=396$
               \item só tem 1 grupo de números primos $\Longrightarrow \frac{1}{792}$
               \item existem 8 grupos com números seguidos $\Longrightarrow \frac{8}{792}$
           \end{enumerate}
       \end{sol}
\end{ex}