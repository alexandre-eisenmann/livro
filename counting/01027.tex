\begin{ex}
	São lançados, simultaneamente dois dados dodecaédricos ( com 12 faces) , um vermelho e um amarelo. A probabilidade de ocorrer no dado amarelo, número maior que o número do dado vermelho, é
    \begin{enumerate}[(a)]
    \item $\frac{13}{24}$
    \item $\frac{11}{24}$
    \item $\frac{1}{2}$
    \item $\frac{7}{12}$
    \item $\frac{5}{8}$
    \end{enumerate}
      \begin{sol}
        resposta: b \\
        saindo 1 no vermelho, pode sair no amarelo: 2, 3, 4, 5, 6, 7, 8, 9, 10, 11, 12 = 11 números\\
        saindo 2 no vermelho, pode sair no amarelo: 3, 4, 5, 6, 7, 8, 9, 10, 11, 12 = 10 números  \\
        e assim sucessivamente teremos : 11+10+9+8+7+6+5+4+3+2+1= 66 \\
        espaço amostral = 144 $\Longrightarrow p = \frac{66}{144}=\frac{11}{24}$
      \end{sol}
\end{ex}