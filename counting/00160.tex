\begin{ex}
(Mack) Numa emergência, suponha que você precise ligar para a polícia. O número a ser ligado tem 3 dígitos. Você sabe que o primeiro dígito é 1 e o terceiro é 0 ou 2, mas você não sabe qual é o dígito do meio. A probabilidade de você acertar o número da polícia, em até duas tentativas é:
   \begin{enumerate}[(a)]
   \item $\frac{19}{49}$
   \item $\frac{1}{10}$
   \item $\frac{2}{5}$
   \item $\frac{19}{20}$
   \item $\frac{1}{19}$
   \end{enumerate}
     \begin{sol}
       resposta: b \\
       - probabilidade de acertar na 1ª tentativa: $\frac{1}{20}$  \\
       - probabailidade de acertar na 2ª tentativa: $\frac{19}{20}\cdot \frac{1}{19}=\frac{1}{20}$ \\
      - probabilidade de acertar em até 2 tentativas: $\frac{1}{20}+\frac{1}{20}=\frac{1}{10}$
     \end{sol}
\end{ex}