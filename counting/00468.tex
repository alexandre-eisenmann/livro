\begin{ex}
(Uel) De um total de 500 estudantes da área de exatas, 200 estudam Cálculo Diferencial  e 180 estudam Álgebra Linear. Esses dados incluem 130 estudantes que estudam ambas as disciplinas. Qual é a probabilidade de que um estudante escolhido aleatoriamente esteja estudando Cálculo Diferencial ou Álgebra Linear?
   \begin{enumerate}[(a)]
   \item 0,26
   \item 0,50
   \item 0,62
   \item 0,76
   \item 0,80
   \end{enumerate}
    \begin{sol}
     resposta: b \\ \\
       \begin{venndiagram2sets} [labelA=\(CD\),labelB=\(AL\),labelOnlyA=70,labelOnlyB=50,labelNotAB=250,labelAB=130]
       \end{venndiagram2sets}
       $70+130+50=250 \Longrightarrow \frac{250}{500}=\frac{1}{2}$
    \end{sol}
\end{ex}