\begin{ex}
 Um dado é constuído de tal forma  que num lançamento se tenha $p_1=p_3=p_5$ e $p_2=p_4=p_6$ .  Se $p_2= 2\cdot p_1$, calcule:
    \begin{enumerate}[(a)]
    \item $p_1$ e $p_2$.
    \item a probabilidade de obter mais de 3 pontos num lançamento.
    \end{enumerate}
      \begin{sol}
       \phantom{A}
         \begin{enumerate}  [(a)]
            \item $p_1=p_3=p_5$ e $p_2=p_4=p_6$ ;\hspace{0,3cm} $p_1+p_2+p_3+p_4+p_5+p_6=1$; \hspace{0,3cm} $p_2 =  2 \cdot p_1$\\ 
            $p_1+2p_1+p_1+2p_1+p_1+2p_1=1 \longrightarrow p_1=\frac{1}{9}\hspace{0.3cm}p_2=\frac{2}{9}$
            \item $p_4+p_5+p_6 = \frac{2}{9}+\frac{1}{9}+\frac{2}{9}=\frac{5}{9}$
         \end{enumerate}
      \end{sol}
\end{ex}