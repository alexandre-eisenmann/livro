\begin{ex}
  Dois tenistas A e B iam disputar um prêmio de US\$ 3200,00 e seria considerado vencedor aquele que ganhasse quatro partidas seguidas ou não. Em cada partida, ambos tinham chance igual de vencer. Após as três primeiras partidas, das quais duas foram vencidas por A e uma por B, um mau tempo impediu a continuação da disputa e, então, decidiu-se repartir o prêmio. Do ponto de vista probabilístico qual seria a divisão justa?
    \begin{sol}
      \phantom{A} \\
      A só precisa vencer 1 partida e B precisa vencer 3.\\
      A probabilidade de B vencer 3 partidas  é  $\frac{1}{2}\cdot\frac{1}{2}\cdot\frac{1}{2}=\frac{1}{8}$ então a probabilidade de A é $\frac{7}{8} \Longrightarrow3200 \div 8= 400$\\ A deve receber $7 \times 400=2800$ e B 400
      
    \end{sol}
\end{ex}