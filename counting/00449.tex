\begin{ex}
(Unicamp) Em matemática, um número natural \textit{a} é chamado palíndromo se seus algarismos, escritos em ordem inversa, produzem o mesmo número. Por exemplo: 8, 22 e 373 são palíndromos. Pergunta-se:
   \begin{enumerate}[(a)]
   \item quantos números naturais palíndromos existem entre 1 e 9.999?
   \item escolhendo-se ao acaso um número natural entre 1 e 9.999, qual é a probabilidade de que esse número seja palíndromo? Tal probabilidade é maior ou menor que 2\%? Justifique a resposta.
   \end{enumerate}
     \begin{sol}
      \phantom{A}
       \begin{enumerate} [(a)]
           \item - palíndromos com 1 algarismo : 9 \\
           - com 2 algarismos: $\frac{\phantom{A}}{9}\frac{\phantom{A}}{1}=9$\\
           - com 3 algarismos:$\frac{\phantom{A}}{9}\frac{\phantom{A}}{10}\frac{\phantom{A}}{1}=90$ \\
           - com 4 algarismos: $\frac{\phantom{A}}{9}\frac{\phantom{A}}{10}\frac{\phantom{A}}{1}\frac{\phantom{A}}{1}=90$ \hspace{0,3cm}
           total : $9+9+90+90=198$
           \item $\frac{198}{9999}=\frac{22}{1111}=\frac{2}{101}\therefore$ \hspace{0,2cm} menor que 2\%
       \end{enumerate}
     \end{sol}
\end{ex}