\begin{ex}
(Ime) Um grupo de 9 pessoas, sendo duas dela irmãos, deverá formar três equipes, com respectivamente dois, três e quatro integrantes. Sabendo que os dois irmãos não podem ficar na mesma equipe, o número de equipes que podem ser organizadas é:
   \begin{enumerate}[(a)]
   \item 288
   \item 455
   \item 480
   \item 910
   \item 960
   \end{enumerate}
    \begin{sol}
     resposta: d \\
     -sem restrição: $\mathrm{C}_{9,2}\cdot\mathrm{C}_{7,3}\cdot\mathrm{C}_{4,4}=1260$\\
     -2 irmãos nas equipes de 2:
     $\mathrm{C}_{2,2}\cdot\mathrm{C}_{7,3}\cdot\mathrm{C}_{4,4}=35$ \\
     -2 irmãos na equipe de 3: $\mathrm{C}_{7,1}\cdot\mathrm{C}_{6,2}\cdot\mathrm{C}_{4,4}=105$ \\
     -2 irmãos na equipe de 4:
     $\mathrm{C}_{7,2}\cdot\mathrm{C}_{5,2}\cdot\mathrm{C}_{4,4}=210$ \\
     Total - (os 2 irmãos nas equipes)= $1260-35-105-210=910$
    \end{sol}
\end{ex}