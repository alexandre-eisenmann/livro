\begin{ex}
(UF – RJ) Fernando e Cláudio foram pescar num lago onde só existem trutas e carpas. Fernando pescou no total, o triplo da quantidade pescada por Cláudio. Fernando pescou duas vezes mais trutas do que carpas, enquanto que Cláudio pescou quantidades iguais de carpas e trutas. Os peixes foram todos jogados num balaio e uma truta foi escolhida ao acaso desse balaio. Determine a probabilidade de que essa truta tenha sido pescada por Fernando.
  \begin{sol}
    \phantom{A} \\ 
    Fernando pescou 2\textit{x} trutas e \textit{x} carpas\hspace{0,3cm} Claudio pescou $\frac{x}{2}$ trutas e $\frac{x}{2}$ carpas\\
    total de peixes pescados : $2x+\frac{x}{2}=\frac{5x}{2} \Longrightarrow \frac{2x}{\frac{5x}{2}}=\frac{4}{5}=80\%$
  \end{sol}
\end{ex}