\begin{ex}
 (FGV) Um grupo de seis amigos (A, B, C, D, E, F) pretende realizar um passeio de barco onde só há três lugares. É feito então um sorteio para serem escolhidos os 3 amigos que ocuparão o barco. A probabilidade de que A seja escolhido e B não o seja, é:
    \begin{enumerate}[(a)]
    \item $\frac{6}{15}$
    \item $\frac{3}{10}$
    \item $\frac{4}{6}$
    \item $\frac{1}{2}$
    \item $\frac{4}{5}$
    \end{enumerate}
      \begin{sol}
        resposta: b \\
        A está no barco e B não $\longrightarrow \mathrm{C}_{4,2}$ \\
        espaço amostral: $\mathrm{C}_{6,3}$\\
        $\frac{\mathrm{C}_{4,2}}{\mathrm{C}{6,3}}=\frac{3}{10}$
      \end{sol}
\end{ex}