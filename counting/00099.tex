\begin{ex}
 (Unicamp) Um atleta participa de um torneio composto por três provas. Em cada prova, a probabilidade de ele ganhar é de $\frac{2}{3}$, independentemente do resultado das outras provas. Para vencer o torneio, é preciso ganhar pelo menos duas provas. A probabilidade de o atleta vencer o torneio é igual a:
   \begin{enumerate}   [(a)]
       \item $\frac{2}{3}$
       \item $\frac{4}{9}$
       \item $\frac{20}{27}$
       \item $\frac{16}{81}$
   \end{enumerate}
     \begin{sol}
      resposta: c \\
      ganhar pelo menos 2 vezes ou ganhar 3 vezes: $\frac{2}{3}\cdot\frac{2}{3}\cdot\frac{1}{3}\cdot3+\frac{2}{3}\cdot\frac{2}{3}\cdot\frac{2}{3}=\frac{20}{27}$
     \end{sol}
 \end{ex}