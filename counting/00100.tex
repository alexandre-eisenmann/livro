\begin{ex}
  (Fuvest) Doze pontos são assinalados sobre quatro segmentos de reta de forma que três pontos sobre três segmentos distintos nunca são colineares, como na figura.
    \begin{figure} [!htb]
        \centering
        \includegraphics[scale=0.5]{imagens/fuvest_2018.png}
    \end{figure}
  O número de triângulos distintos que podem ser desenhados com os vértices nos pontos assinalados é :
    \begin{enumerate}   [(a)]
        \item 200
        \item 204
        \item 208
        \item 212
        \item 220
    \end{enumerate}
      \begin{sol}
       resposta: d \\
       há 4 pontos alinhados na horizontal e mais 4 no segmento decrescente:\hspace{0,1cm} $2\cdot\mathrm{C}_{4,3}$ \\
       número de triângulos formados: $\mathrm{C}_{{12},3}-2\cdot\mathrm{C}_{4,3}=212$
      \end{sol}
 \end{ex}