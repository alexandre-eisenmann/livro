\begin{ex}
(Fuvest) Seis pessoas: A, B, C, D, E e F vão atravessar um rio em três barcos. Distribuindo-se ao acaso as pessoas de modo que fiquem duas em cada barco, a probabilidade de A atravessar junto com B, C junto com D e E junto com F é:
   \begin{enumerate}[(a)]
   \item $\frac{1}{5}$
   \item $\frac{1}{10}$
   \item $\frac{1}{15}$
   \item $\frac{1}{20}$
   \item $\frac{1}{25}$
   \end{enumerate}
    \begin{sol}
      resposta: c \\
    casos possíveis (espaço amostral): são 6 lugares disponíveis para A, 5 lugares para B, 4 lugares para C........ isto é: $6\cdot5\cdot4\cdot3\cdot2\cdot1=6!=720$\\
    casos favoráveis: A vai junto com B, C  com D e E com F, isto é: $6\cdot1\cdot4\cdot1\cdot2\cdot1=48$\\
    $\Longrightarrow p=\frac{48}{720}=\frac{1}{15}$
    \end{sol}
\end{ex}