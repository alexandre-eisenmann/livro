\begin{ex}
 (Enem) Estima-se que haja,  no Acre, 209 espécimes de mamíferos, distribuidas conforme a tabela a seguir:

\begin{table}
\centering
\begin{tabular}{|l|c|} \hline
\textbf{ Grupos taxonômicos } & \textbf{Número de espécies} \\  \hline   
Artiodáctilos & 4 \\  \hline
Carnívoros & 18    \\  \hline
Cetáceos & 2   \\  \hline
Quirópteros & 103  \\ \hline
Lagomorfos & 1 \\  \hline
Marsupiais  & 16 \\  \hline
Perissodáctilos & 1 \\  \hline
Primatas & 20 \\  \hline
Roedores & 33 \\  \hline
Sirênios & 1 \\  \hline
Edentados  & 10 \\ \hline
\ TOTAL & 209 \\ \hline
\end{tabular}
\caption*{T\&C Amazônia, ano 1, nº3, dez 2003}
\end{table}

Deseja-se realizar um estudo comparativo entre três dessas espécies de mamíferos: uma do grupo Cetáceos, outra do grupo Primaatas e a terceira do grupo Roedores. O número de conjuntos distintos que podem ser formados com essas espécies para esse estudo é igual a:
    \begin{enumerate}[(a)]
    \item 1320
    \item 2090
    \item 5845
    \item 6600
    \item 7245
    \end{enumerate}
      \begin{sol}
      resposta: a  \\
      2 cetáceos, 20 primatas e 33 roedores  
      $\Longrightarrow33\cdot20\cdot2=1320$
      
      \end{sol}
\end{ex}