\begin{ex}
(Ufg) Duas moedas diferentes foram lançadas simultaneamente, 4 vezes, e os resultados foram anotados no quadro a seguir:
(K= cara; C= coroa)
\begin{center}
\begin{tabular}{|c|c|c|}  \hline
LANÇAMENTO  &  MOEDA 1 & MOEDA 2\\  \hline
1  &  K  &  K   \\  \hline
2  &  K  &  C   \\  \hline
3  &  C  &  K   \\  \hline
4  &  C  &  C  \\ \hline
\end{tabular}
\end{center}
Nos próximos 4 lançamentos, a probabilidade de se obter os 4 resultados obtidos anteriormente, em qualquer ordem, é:
   \begin{enumerate}[(a)]
   \item 1
   \item $\frac{1}{2^5}$
   \item $\frac{3}{2^5}$
   \item $\frac{1}{2^8}$
   \item $\frac{3}{2^8}$
   \end{enumerate}
   \begin{sol}
    resposta: c \\
    $(\frac{1}{4})^4.4!= \frac{1}{2^8}\cdot4!=\frac{3}{2^5}$
   \end{sol}
\end{ex}