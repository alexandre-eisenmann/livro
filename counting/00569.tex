\begin{ex}
A cobertura de um evento internacional é feita por repórteres do jornal A sendo 8 homens e 4 mulheres; por repórteres do jornal B sendo 6 homens e 9 mulheres; e por repórteres do jornal C sendo 7 homens e algumas mulheres. Em uma entrevista coletiva da qual participam todos esses repórteres será sorteado um deles para a primeira pergunta. Sabendo que $\frac{2}{3}$ é a probabilidade de que o profissional sorteado seja uma mulher ou seja do jornal A, calcule o número de repórteres mulheres do jornal C que participam da cobertura do evento.
  \begin{sol}
   \phantom{A} \\
   jornal A=12;\hspace{0,2cm} jornal B =15;\hspace{0,2cm} jornal C =7+\textit{x} \\
   $\frac{2}{3}=\frac{4+9+x}{12+15+7+x}+\frac{12}{12+15+7+x}-\frac{4}{12+15+7+x} \Longrightarrow x=5$
  \end{sol}
\end{ex}