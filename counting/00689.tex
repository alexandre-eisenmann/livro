\begin{ex}
 Nove pessoas estão no ponto esperando o ônibus Vila Nhocuné. Entre essas pessoas 4 são mulheres: Kátia, Laura, Marta e Nádia. Cinco são homens: Álvaro, Beto, Carlão, Duda e Ernesto. Ao chegar ao ônibus, a situação vira a maior muvuca, com todos querendo entrar primeiro. No final, tudo acabará bem, com todos entrando, um de cada vez, no tal ônibus.
    \begin{enumerate}[(a)]
    \item de quantas maneiras diferentes poderá ser formada a ordem de entrada das pessoas no ônibus se o Carlão não vai entrar em primeiro lugar e nem em último?
    \item Supondo que todos tenham a mesma chance e capacidade de entrar no ônibus, e que, dessa forma, a fila seja formada ao acaso, qual será a probabilidade de que não entrem dois homens em sequência?
    \end{enumerate}
      \begin{sol}
          \phantom{A} 
        \begin{enumerate} [(a)]
            \item (todos) - (Carlão entrar em primeiro ou último lugar) = $ 9!-2\cdot8!=282240$
            \item $\frac{5!\cdot4!}{9!}$
        \end{enumerate}
      \end{sol}
\end{ex}