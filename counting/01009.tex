\begin{ex}
 Dois dados perfeitos e distinguíveis são lançados ao acaso. A probabilidade de que a soma dos resultados obtidos seja 3 ou 6 é:
    \begin{enumerate}[(a)]
    \item $\frac{7}{18}$
    \item $\frac{1}{18}$
    \item $\frac{7}{36}$
    \item $\frac{7}{12}$
    \item $\frac{4}{9}$
    \end{enumerate}
      \begin{sol}
        resposta: c \\
         - possibilidades: (1,2) (2,1) (1,5) (2,4) (3,3) (4,2) (5,1) $\Longrightarrow 7\cdot\frac{1}{36}=\frac{7}{36}$
      \end{sol}
\end{ex}