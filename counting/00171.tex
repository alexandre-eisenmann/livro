\begin{ex}
   (Unesp) Numa festa de aniversário infantil, 5 crianças comeram um alimento contaminado por uma bactéria. Sabe-se que, uma vez em contato com esta bactéria, a probabilidade de que a criança manifeste problemas intestinais é de $\frac{2}{3}$. Sabendo que
\(\binom{n}{k} = \frac{n!}{k!(n-k)!}\) determine: 
   \begin{enumerate}[(a)]
   \item \(\binom{5}{2}\) e a probabilidade de manifestação de problemas intestinais em exatamente duas crianças.
   \item \(\binom{5}{0}\),\(\binom{5}{1}\) e a probabilidade de manifestação de problemas intestinais no máximo
   em uma criança.
   \end{enumerate}
     \begin{sol}
       \phantom{A}
         \begin{enumerate} [(a)]
             \item $\binom{5}{2}=\frac{5!}{2!\cdot3!}=10$ \hspace{0,5cm} e \hspace{0,5cm}$\binom{5}{2}(\frac{2}{3})^2 (\frac{1}{3})^3=\frac{40}{243}$
            \item $\binom{5}{0}=\frac{5!}{0!5!}=1$ \hspace{0,5cm} e \hspace{0,5cm} $\binom{5}{1}=\frac{5!}{1!\cdot4!}= 5$ \hspace{0,5cm}
             e \hspace{0,5cm} $\binom{5}{0}(\frac{2}{3})^0(\frac{1}{3})^5+\binom{5}{1}(\frac{2}{3})^1(\frac{1}{3})^4=\frac{11}{243}$
             
         \end{enumerate}
     \end{sol}

\end{ex}