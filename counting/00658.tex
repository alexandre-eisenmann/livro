\begin{ex}
 Com os algarismos 1, 2, 3, 4, 5, 6 e 7 são formados números inteiros de 3 algarismos distintos.
    \begin{enumerate}[(a)]
    \item quantos são divisiveis por 5?
    \item e se considerarmos os algarismos 0, 1, 2, 3, 4, 5, 6 e 7?
    \end{enumerate}
    \begin{sol}
        \phantom{A} 
     \begin{enumerate} [(a)]
         \item $\frac{\phantom{A}}{6}\frac{\phantom{A}}{5}\frac{5}{1} = 30$
         \item $\frac{\phantom{A}}{6}\frac{\phantom{A}}{6}\frac{5}{\phantom{A}}+\frac{\phantom{A}}{7}\frac{\phantom{A}}{6}\frac{0}{\phantom{A}}= 36+42=78$
     \end{enumerate}
    \end{sol}
\end{ex}