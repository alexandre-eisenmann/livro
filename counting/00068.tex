\begin{ex}
 (Unesp) Joga-se um dado honesto. O número que ocorreu (isto é, da face voltada para cima) é o coeficiente b da equação $x^2 +bx +1 = 0$. Determine:
   \begin{enumerate} [(a)]
       \item a probabilidade de essa equação ter raízes reais.
       \item a probabilidade de essa equação ter raízes reais, sabendo-se que ocorreu um número ímpar. 
   \end{enumerate}
     \begin{sol}
     \phantom{A}
       \begin{enumerate} [(a)]
           \item para ter raízes reais: $\Delta = b^2-4 \geq 0 \Rightarrow b=\{2, 3, 4, 5, 6\}
           \Longrightarrow p=\frac{5}{6}$
           \item se \textit{b} é ímpar, \textit{b} pode ser 1, 3 ou 5 $\Longrightarrow p=\frac{2}{3}$
       \end{enumerate}
     \end{sol}
\end{ex}