\begin{ex}
 (Unesp) Dois rapazes e duas moças irão viajar de ônibus, ocupando as poltronas de números 1 a 4, com 1 e 2 juntas e 3 e 4 juntas, conforme o esquema:
 \begin{center}
     \begin{tikzpicture}
      \draw (0,0)--(2,0)--(2,1)--(0,1)--(0,0);
      \draw (0,2)--(2,2)--(2,3)--(0,3)--(0,2);
      \draw (1,2)--(1,3);  \draw (1,0)--(1,1);
      \draw [densely dotted] (2,-1)--(2,4);
      \draw [densely dotted] (3.5,-1)--(3.5,4);
      \draw [->] [very thick] (4,.8)--(3,1.5);
      \draw node [right] at (4,.8) {corredor do ônibus};
      \draw node at (.5,2.5) {1};\draw node at (1.5,2.5) {2};
      \draw node at (.5,.5) {3}; \draw node at (1.5,.5) {4};
     \end{tikzpicture}
\end{center}
O número de maneiras de ocupação dessas poltronas, garantindo que, em duas poltronas juntas, ao lado de uma moça sempre viaje um rapaz, é:
    \begin{enumerate}[(a)]
    \item 4
    \item 6
    \item 8
    \item 12
    \item 16
    \end{enumerate}
       \begin{sol}
        resposta: e \\
        nas poltronas 1 e 2 : M1R1, M1R2, M2R1,  M2R2,  R1M1,  R2M1,  R1M2,  R2M2 = 8 \\
        idem para as poltronas 3 e 4, portanto 8+8=16 maneiras
        
       \end{sol}
\end{ex}