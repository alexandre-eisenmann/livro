\begin{ex}
(Fuvest) Em um jogo entre Pedro e José, cada um deles lança, em cada rodada, um mesmo dado honesto uma única vez. O dado é cúbico e, cada uma das suas 6 faces estampa um único algarismo de maneira que todos os algarismos de 1 a 6 estejam representados nas faces do dado. Um participante vence, em uma certa rodada, se a diferença entre seus pontos e os pontos de seu  adversário for, no mínimo, de duas unidades. Se nenhum dos participantes vencer, passa-se a uma nova rodada. Dessa forma, determine a probabilidade de:
   \begin{enumerate}[(a)]
   \item Pedro vencer na primeira rodada;
   \item nenhum dos dois participantes vencer na primeira rodada.
   \item um dos participantes vencer até a quarta rodada.
   \end{enumerate}
     \begin{sol}
      \phantom{A}
       \begin{enumerate} [(a)]
           \item Pedro vence se obter os seguintes resultados: (1,3) (1,4) (1,5) (1,6) (2,4) (2,5) (2,6) (3,5) (3,6) (4,6) $\Longrightarrow p=\frac{10}{36}=\frac{5}{18}$
           \item a probabilidade de José vencer na 1ª rodada  também é $\frac{5}{18}$, logo a probabilidade de nenhum dos 2 vencer é : $1-\frac{5}{18}-\frac{5}{18}= \frac{4}{9}$
           \item a possibilidade de nenhum dos 2 vencer nas 4 rodadas é: $\frac{4}{9}\cdot\frac{4}{9}\cdot\frac{4}{9}\cdot\frac{4}{9}=\frac{256}{6561}$ logo a probabilidade de 1 deles vencer até a 4ª rodada é :
          $1-(\frac{4}{9})^4=\frac{6305}{6561}$
       \end{enumerate}
     \end{sol}
\end{ex}