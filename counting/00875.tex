\begin{ex}
 (Cesgranrio) Uma turma tem 25 alunos, dos quais 40\% são meninas. Escolhendo-se, ao acaso, um dentre todos os grupos de dois alunos que se pode formar com os alunos da turma, a probabilidade de que este seja composto por um menino e uma menina é de:
    \begin{enumerate}[(a)]
    \item $\frac{1}{6}$
    \item $\frac{1}{5}$
    \item $\frac{1}{4}$
    \item $\frac{1}{3}$
    \item $\frac{1}{2}$
    \end{enumerate}
      \begin{sol}
       resposta: e \\
       25 alunos sendo 10 meninas e 15 meninos 
       $\Longrightarrow\frac{\mathrm{C}_{{10},1}\cdot \mathrm{C}_{{15},1}}{\mathrm{C}_{{25},2}}=\frac{1}{2}$
      \end{sol}
\end{ex}