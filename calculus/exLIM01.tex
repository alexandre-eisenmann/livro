%Exercicio01
%exercicio118 pag 82
\begin{ex}
Provar que $\lim_{x\rightarrow 2} (3x+1) = 7$ considerando a definição de limite.
%resposta
Seja $\epsilon>0,$ um número arbitrário dado. Para que a desigualdade $\left|(3x+1)-7\right|<\epsilon$ seja satisfeita, é necessário que sejam satisfeitas as desigualdades:$\left|3x-6\right|<\epsilon$, $\left|x-2\right|<\frac{\epsilon}{3}$, $-\frac{\epsilon}{3}<x-2<\frac{\epsilon}{3}$. Para $\epsilon$ arbitrário e para todos os valores da variável, verifica-se a desigualdade $\left|x-2\right|<\frac{\epsilon}{3}=\delta$,o valor da função: $3x+1$ difere de 7 por um valor menor do que $\epsilon$ isto significa, que para $x\rightarrow 2 $ o limite da função é 7.
\end{ex}