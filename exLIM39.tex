%Exercicio39
%exercicio 203
\begin{ex}
\begin{align}
&\lim_{x\rightarrow \infty} [\sqrt{x(x-a)}-x]=\nonumber\\
%resposta
&\lim_{x\rightarrow \infty} \frac{[\sqrt{x(x-a)}-x][\sqrt{x(x-a)}+x]}{\sqrt{x(x-a)}+x}=\lim_{x\rightarrow \infty}\frac{x(x-a)-x^2}{\sqrt{x(x-a)}+x}\nonumber\\
&\text{dividindo-se o numerador e o denominador por}\quad\left|x\right|=\sqrt{x^2},\quad\text{temos:}\nonumber\\
&\lim_{x\rightarrow \infty} \frac{\frac{-ax}{\left|x\right|}}{\frac{\sqrt{x^2-ax}}{\left|x\right|}+\frac{x}{\left|x\right|}}=\nonumber\\
<<<<<<< HEAD
&\text{Introduzindo}\quad\left|x\right|\quad\text{no radical, no radicando, ele aparece como}\quad{x^2},\quad\text{isto �,}\nonumber\\
&\lim_{x\rightarrow \infty} \frac{\frac{-ax}{\left|x\right|}}{{\sqrt{1-\frac{a}{x}}}+\frac{x}{\left|x\right|}}=\nonumber\\
&\text{Para}\quad{x\rightarrow +\infty},\quad\text{o limite �:}\quad{\frac{-a}{2}}\nonumber\\
&\text{Para}\quad{x\rightarrow -\infty},\quad\text{o limite �:}\quad{\frac{a}{1-1}=\frac{a}{0}}=\infty\nonumber
=======
&\text{Introduzindo}\quad\left|x\right|\quad\text{no radical, no radicando, ele aparece como}\quad{x^2},\quad\text{isto �,}\nonumber\\
&\lim_{x\rightarrow \infty} \frac{\frac{-ax}{\left|x\right|}}{{\sqrt{1-\frac{a}{x}}}+\frac{x}{\left|x\right|}}=\nonumber\\
&\text{Para}\quad{x\rightarrow +\infty},\quad\text{o limite �:}\quad{\frac{-a}{2}}\nonumber\\
&\text{Para}\quad{x\rightarrow -\infty},\quad\text{o limite �:}\quad{\frac{a}{1-1}=\frac{a}{0}}=\infty\nonumber
>>>>>>> 2edfa61f09aa52eb0e5cbaaba256a294c2d4af48
\end{align}
\end{ex}
