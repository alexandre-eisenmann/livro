%Exercicio351
%exercicio 507
\begin{ex}
\begin{align}
&\text{Calcular a n'�sima derivada de} \quad y=\frac{ax+b}{ax-b}\nonumber\\
%resposta
&y'=\frac{(ax-b)a-(ax+b)a}{(ax-b)^2}=\frac{-2ab}{(ax-b)^2}\nonumber\\
&y''=\frac{2ab.2(ax-b).a}{(ax-b)^4}=\frac{4a^2(ax-b)}{(ax-b)^4}=\frac{4a^2b}{(ax-b)^3}\nonumber\\
&y'''=\frac{-4a^2b.3(ax-b)^2.a}{(ax-b)^6}=\frac{-12a^3b(ax-b)^2}{(ax-b)^6}=\frac{-12a^3b}{(ax-b)^4}\nonumber\\
&y^{n}=\frac{(-1)^n.2n!a^{n}b}{(ax-b)^{n+1}}\nonumber
\end{align}
\end{ex}



